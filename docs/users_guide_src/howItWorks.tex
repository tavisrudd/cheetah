\section{How Cheetah Works}
\label{howWorks}

%%%%%%%%%%%%%%%%%%%%%%%%%%%%%%%%%%%%%%%%%%%%%%%%%%%%%%%%%%%%%%%%%%%%%%%%%%%%%%%%
\subsection{The Template Class}
\label{howWorks.templateClass}

The heart of Cheetah is the \code{Template} class in the \code{Cheetah.Template}
module. It serves two purposes. First, its constructor accepts a
\code{template definition} in a string or a file (filename or file object).
Cheetah compiles the template definition into a Python class (the 'generated
class'). Second, it is used as the base class for the generated classes.
\code{Template} subclasses Webware's \code{HTTPServlet} class when it is
available. Thus, the generated classes can be used as Webware servlets.

Generated template classes can either be used immediately or written to a Python
module file for future use. In the former case, the methods and attributes of
the generated class are dynamically bound to the instance of Template that did
the compiling. They are available as soon as compilation is complete. In the
latter case, Cheetah will wrap the generated class definition in some
boilerplate code to make a complete module definition.

Templates objects exist to be {\bf filled}, that is, to have their 
placeholders substituted and directives executed to produce a finished string.
To fill a template, you call its {\bf core method}.  The core method is
normally \code{.respond()} because that's the core method of a Webware
servlet.  You call this method without arguments and it returns the
finished string.  For convenience, Cheetah makes the
\code{.\_\_str\_\_} method an alias to the core method. Thus, \code{``print
myTemplateObj''} or \code{``print Template(sourceString)''} will print the
output of the template.  (Under certain circumstances, Cheetah will name the
core method \code{.writeBody}, and you can also explicitly name the core
method yourself.  See the \code{\#implements} directive (section
\ref{inheritanceEtc.implements}) for more information.)

To improve performance, Cheetah does "lazy compilation".  The compiling step is
delayed until the first time it is needed; e.g., the first time the template is
filled.  You can also precompile your templates to gain a (slight) increase
in speed, using the {\bf cheetah compile} command described below.

%%%%%%%%%%%%%%%%%%%%%%%%%%%%%%%%%%%%%%%%%%%%%%%%%%%%%%%%%%%%%%%%%%%%%%%%%%%%%%%%
\subsection{Object-Oriented Documents}
\label{howWorks.objoriented}

As Cheetah documents are actually class definitions, templates may inherit from
one another in a natural way, using regular Python semantics. For instance,
consider this template:

\begin{verbatim}
#def title
This document has not defined its title
#end def
#def htTitle
$title
#end def
<HTML><HEAD>
<TITLE>$title</TITLE>
</HEAD><BODY>
<H1>$htTitle</H1>
$body
</BODY></HTML>
\end{verbatim}

And its subclassed document:
\begin{verbatim}
#from SomeModule import TheTemplateAbove
#extends TheTemplateAbove
#def title
The Frog Page
#end def
#def htTitle
The <IMG SRC="Frog.png"> page
#end def
\end{verbatim}

This is a classic use of inheritance. What we consider a "template" is simply an
abstract superclass. Each subdocument both defines its own behaviour and
specializes the output. For instance, in this document we redefine
\code{\$htTitle} so that we can distinguish between the plain title that appears
in the HEAD and the potentially more complicated title that can appear in
\code{H1}.  This allows flexibility, while allowing most documents to ignore the
distinction (since by default \code{\$htTitle} is defined as \code{\$title}).

In many other templating systems, the solution to this sort of problem is
implemented with case statements (in whatever form they take) often placed in
many different sections of code. This is the classic problem that arises with
purely procedural programming languages, and Cheetah uses a classical approach
to solve it.

This is also a good way to mix Cheetah and Python code. Cheetah is not and does
not try to be the right language for everything you want to do (in contrast to
PHP, for instance). Instead it tries to focus on display logic and integrate
easily with Python code.

Consider the first example template: while we show another Cheetah document
inheriting from it, a Python class could inherit from it just as easily. This
Python class could define its programmatically-driven value for \code{\$body}
and \code{\$title}, simply by defining body() and title() methods.

Similarly, the Cheetah document can inherit from an arbitrary class. This
technique is used when combining Cheetah with Webware for Python
(chapter \ref{webware}): the base template for your site inherits (indirectly)
from the Webware HTTPServlet class.  The classes are sufficiently generic that
similar techniques should be possible for other systems.

%%%%%%%%%%%%%%%%%%%%%%%%%%%%%%%%%%%%%%%%%%%%%%%%%%%%%%%%%%%%%%%%%%%%%%%%%%%%%%%%
\subsection{Constructing Template Objects}
\label{howWorks.constructing}

Here are typical ways to create a template object:
\begin{description}
\item{\code{templateObj = Template("The king is a \$placeholder1.")}}
     \\ Pass the Template Definition as a string.
\item{\code{templateObj = Template(file="fink.tmpl")}}
     \\ Read the Template Definition from a file named "fink.tmpl".  
\item{\code{templateObj = Template(file=f)}}
     \\ Read the Template Definition from file-like object 'f'.
\item{\code{templateObj = Template("The king is a \$placeholder1.", searchList=[dict, obj])}}
     \\ Pass the Template Definition as a string.  Also pass two Namespaces for
     the searchList: a dictionary 'dict' and an instance 'obj'.
\item{\code{templateObj = Template(file="fink.txt", searchList=[dict, obj])}}
     \\ Same, but pass a filename instead of a string.  The \code{None} is
     required here to represent the missing Template Definition string -- this
     due to Python's rules for positional parameters.
\item{\code{templateObj = Template(file=f, searchList=[dict, obj])}}
     \\ Same with a file object.
\end{description}

The constructor accepts the following keyword arguments:

\begin{description}
\item{{\bf source}}
     The template definition as a string.  
\item{{\bf file}}
     A filename or file object containing the template definition.
     A filename must be a string, and a file object must be open for reading.
\item{{\bf searchList}}
     A list of objects to search for \code{\$placeholder} values.
\item{{\bf filter}}
     A class that will format every \code{\$placeholder} value.  You may
     specify a class object or string.  If a class object,
     it must be a subclass of \code{Cheetah.Filters.Filter}.  If a string, it
     must be the name of one of the filters in filtersLib module (see next 
     item).
     (You may also use the \code{\#filter} directive (section
     \label{output.filter}) to switch filters at runtime.)
\item{{\bf filtersLib}}
     A module containing the filters Cheetah should look up by name.  The
     default is \code{Cheetah.Filters}.  All classes in this module that are
     subclasses of \code{Cheetah.Filters.Filter} are considered filters.
\item{{\bf errorCatcher}}
     A class to handle \code{\$placeholder} errors.  You may
     specify a class object or string.  If a class object,
     it must be a subclass of \code{Cheetah.ErrorCatchers.ErrorCatcher}.  
     If a string, it must be the name of one of the error catchers in
     \code{Cheetah.ErrorCatchers}.  This is similar to the 
     \code{\#errorCatcher} directive 
     (section \ref{errorHandling.errorCatcher}).
\item{{\bf compilerSettings}}
     A dictionary (or dictionary hierarchy) of settings that change Cheetah's
     behavior.  Not yet documented.
\end{description}

You must specify either {\bf source} or {\bf file}, but not both.  All other
arguments are optional.  If you are using {\bf source}, you may omit the
\code{source=} prefix {\em if it is the first argument}, as in all the examples
above.  

If you spell a keyword wrong, it will be ignored rather than causing an error.
So check your spelling!

EXCEPTION: When using a precompiled template class created by
\code{cheetah compile}, you do {\em not} need to specify a {\bf source} or {\bf
file} argument since that has already been provided.  This allows you to
instantiate a precompiled template object without any constructor arguments if
you wish:
\begin{verbatim}
from MyPrecompiledTemplate import MyPrecompiledTemplate
t = MyPrecompiledTemplate()
t.name = "Fred Flintstone"
t.city = "Bedrock City"
print t
\end{verbatim}

%%%%%%%%%%%%%%%%%%%%%%%%%%%%%%%%%%%%%%%%%%%%%%%%%%%%%%%%%%%%%%%%%%%%%%%%%%%%%%%%
\subsection{cheetah compile: converting .tmpl files into .py modules}
\label{howWorks.tmpl2py}
\label{howWorks.cheetah-compile}

If your application requires only a few short template definitions, you can
just put them inline in your modules, and create Template objects like the
examples in section \ref{gettingStarted.tutorial}.  But if your application
has large templates or many templates, you will find it more convenient to
put each template definition in a separate *.tmpl file, and use the
\code{cheetah compile} program to convert it into a *.py template module; that
is, a Python module named after the template, containing the generated class
which is also named after the template.  

\code{cheetah compile} parses template definition
files and creates equivalent Python modules.  On Unix systems, it is installed
into a system directory like \code{/usr/local/bin}, so you can use it without
specifying the absolute path of the script.  On Windows systems, you must
specify the full path (\code{<cheetahRoot>/bin/cheetah compile}). Type
``\code{cheetah compile --help}'' from the command line after installing
Cheetah to get usage information.  The most common usage is
``\code{cheetah-compile -R}'', which will convert all the *.tmpl files in the
current directory and its subdirectories.

For backward compatibility, a separate program \code{cheetah-compile} is also
installed.  This program does exactly the same thing as \code{cheetah compile}.

If you run \code{cheetah-compile} on a file FILENAME.tmpl, it will
overwrite FILENAME.py if it exists in the same directory, no matter what
FILENAME.py contains.  For this reason, you should make changes to the
\code{.tmpl} version of the template rather than to the \code{.py} version.  Any
\code{.py servlet files} that are about to be overwritten will are
automatically backed up with the extension \code{.py\_bak}.

Because FILENAME will be used as a class and module name, it must be a valid
Python identifier.  For instance, \code{cheetah compile spam-eggs.tmpl} is 
illegal because of the hyphen ("-").

One of the advantages of \code{cheetah compile} is that you don't lose any
flexibility.  The generated class contains all \code{\#attr} values and
\code{\#def}/\code{\#block} values as ordinary attributes and methods of the
template class, so you can read the values individually from other Python
tools for any kind of custom processing you want.  For instance, if you want to
put the titles of all your servlets into a database.

For the same reason, if your template requires custom Python methods or
other Python code, don't put it in the \code{FILENAME.py} file.  Instead, put
it in a separate base class and use the \code{\#extends} directive to
inherit from it.  Or put the Python code in the calling routine (the routine
that instantiates and uses the template object).

%%%%%%%%%%%%%%%%%%%%%%%%%%%%%%%%%%%%%%%%%%%%%%%%%%%%%%%%%%%%%%%%%%%%%%%%%%%%%%%%
\subsection{Some trivia about .py template modules}
\label{howWorks.pyTrivia}

Note how .py template modules are {\em different} from a module you write
yourself in the spirit of:

\begin{verbatim}
from Cheetah.Template import Template

TEMPLATE_DEF = """\
#attr $what = "beautiful"
Here's my $what template.
... lots of other fancy stuff ...
"""

t = Template(TEMPLATE_DEF)
# ... lots of other fancy stuff ...
print t
\end{verbatim}

This module contains an ordinary template definition, and the compilation is
all done behind the scenes.  A ".py template module", in contrast, contains
lines like:

\begin{verbatim}
write("The number is ")  
write(filter(VFN(VFS(SL,"Test.unittest",1),"main",0)
write(".")
\end{verbatim}

which is the compiled equivalent of:

\begin{verbatim}
The number is $Test.unittest.main.
\end{verbatim}

Clearly you don't want to edit a .py template module directly--who wants
to try to keep all those function calls straight?

We won't speak further about the internals of .py template module in this
Guide.  A closer analysis of them will be in the forthcoming {\em Cheetah
Developers' Guide}.  But inspecting at a few .py template modules is a good way
to see how Cheetah works under the hood, because it's ordinary object-oriented
Python code.  Looking at its .py template module may also give clues as to
why a particular template doesn't work.  You can see what the compiler 
thought you meant by a particular construct, and what it converted it to.

%%%%%%%%%%%%%%%%%%%%%%%%%%%%%%%%%%%%%%%%%%%%%%%%%%%%%%%%%%%%%%%%%%%%%%%%%%%%%%%%
\subsection{Running your template as a standalone script}
\label{howWorks.standalone}

In addition to importing your cheetah-compile'd \code{.py} file into a Python
program or using it as a Webware servlet, you can also run it as a standalone
program, just like any Python script.  The program will print the filled
template on standard output.  This is useful while debugging the template.
It's also useful as a filter in certain production situations such as shell
scripts.  

When running the template as a program, you cannot provide a searchList or
set \code{self.} attributes in the normal way.  So you must take
alternative measures to ensure that every placeholder has a value.
Otherwise, you will get the usual \code{NameMapper.NotFound} exception at
the first missing value.

Fortunately, there are three ways to supply values to cheetah-compile'd
templates running directly from the command line:

\begin{description}
\item{{\bf Default values}}  You can set default values in the template itself
     (via the \code{\#attr} directive) or in a Python superclass.
\item{{\bf Environment variables}} If you use the \code{-e} command-line option,
     Cheetah will look in the environment for values.  This is the easiest way
     to pass values from a shell script.
\item{{\bf A pickle file}}  If you use the \code{-p PICKLE\_FILE} option, 
     Cheetah will unpickle the file and place the resulting data structure in
     the searchList.    See the standard Python modules \code{pickle} and
     \code{cPickle} for more information.  (The truly masochistic who love
     filters can even use \code{-p -} to read the pickle data from standard
     input.)
\end{description}

You can always run \code{python FILENAME.py --help} to see all the command-line
options your template program accepts.  (That's a double hyphen before the
"help", even if LaTeX misformats it to look like a single hyphen.)

Cheetah .py templates that will be used as Webware servlets can also be 
debugged this way.  The only caveat is that if they do any processing that
tries to call back into a live web transaction (such as looking for form
input data), they will raise an exception since there is no web transaction
in progress.  These servlets must be debugged by calling them through the
web.

% Local Variables:
% TeX-master: "users_guide"
% End:      

