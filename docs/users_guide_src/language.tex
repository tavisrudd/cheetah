\section{Language Overview}
\label{language}


Cheetah's basic syntax was inspired by the Java-based template engines Velocity
and WebMacro. It has two types of tags: {\bf placeholders} and {\bf directives}.
Both types are case-sensitive.  

Placeholder tags begin with a dollar sign (\code{\$varName}) and are similar to
data fields in a form letter or the \code{\%(key)s} fields Python's \code{\%}
operator uses. When a template is processed, placeholders are replaced with the
value they refer to.

Directive tags begin with a hash character (\#) and are used for comments, loops,
conditional blocks, includes, and all other advanced features. Cheetah uses a
Python-like syntax inside directive tags and understands any valid Python
expression.  Some of the main differences from pure Python syntax are that
variable names are prefaced with a dollar sign (\$), that colons (:) are not
used to mark the beginning of a code block, and indentation is not significant.
See section \ref{language.namemapper} below for details of another important
difference. Most of the directive tags listed below are direct mirrors of Python
statements.

Cheetah does not use HTML/XML-style tags like some other template languages for
the following reasons:
\begin{itemize}
\item Cheetah is not limited to HTML
\item HTML-style tags are hard to distinguish from real HTML tags
\item HTML-style tags are not visible in rendered HTML when something goes wrong
\item HTML-style tags often lead to invalid HTML, as with
     \code{<img src=''<template-directive>''>}
\item Cheetah tags are less verbose and easier to understand than
     HTML-style tags
\item HTML-style tags aren't compatible with most WYSIWYG editors
\end{itemize}

Besides being much more compact, Cheetah also has some advantages over template
systems, such as Zope Page Templates, that use the structure of the HTML and put
information inside the HTML tags:

\begin{itemize}
\item HTML or XML-bound languages do not work well with other languages.
\item While ZPT-like syntaxes work well in many ways with WYSIWYG HTML editors,
     they also give up a significant advantage of those editors -- concrete
     editing of the document.  When logic is hidden away in (largely
     inaccessible) tags it is hard to understand a page simply by viewing it,
     and it is hard to confirm or modify that logic.
\end{itemize}

%%%%%%%%%%%%%%%%%%%%%%%%%%%%%%%%%%%%%%%%%%%%%%%%%%%%%%%%%%%%%%%%%%%%%%%%%%%%%%%%
\subsection{Language Constructs}
\label{language.constructs}

\begin{enumerate}
\item Comments and documentation strings
     \begin{enumerate}
     \item \code{\#\# single line}
     \item \code{\#* multi line *\#}
     \end{enumerate}

\item Generation, caching and filtering of output
     \begin{enumerate}
     \item plain text
     \item output from simple expressions: \code{\$placeholders}
     \item output from more complex expressions: \code{\#echo} \ldots
     \item silencing output from expressions: \code{\#silent} \ldots
     \item gobble the EOL: \code{\#slurp}
     \item parsed file includes: \code{\#include} \ldots
     \item raw file includes: \code{\#include raw} \ldots
     \item verbatim output of Cheetah code: \code{\#raw} \ldots \code{\#end raw}
     \item cached placeholders: \code{\$*var}, \code{\$*<interval>*var}
     \item cached regions: \code{\#cache} \ldots \code{\#cache}
     \item set the output filter: \code{\#filter} \ldots
     \end{enumerate}
          
\item Importing other modules and from other modules: \code{\#import} \ldots,
     \code{\#from} \ldots

\item Inheritance 
     \begin{enumerate}
     \item set the base classes to inherit from: \code{\#extends}
     \item set the name of the main method to implement: \code{\#implements} \ldots
     \end{enumerate}

\item Compile-time declaration
     \begin{enumerate}
     \item define class attributes: \code{\#attr} \ldots
     \item define class methods: \code{\#def} \ldots \code{\#end def}
     \item \code{\#block} \ldots \code{\#end block} provides a simplified
          interface to \code{\#def} \ldots \code{\#end def}
     \item define class 'settings': \code{\#settings} \ldots \code{\#end settings}
     \end{enumerate}

\item Run-time assignment
     \begin{enumerate}
     \item local vars: \code{\#set} \ldots
     \item global vars: \code{\#set global} \ldots
     \end{enumerate}

\item Flow control
     \begin{enumerate}
     \item \code{\#if} \ldots \code{\#else} \ldots \code{\#else if} (aka
          \code{\#elif}) \ldots \code{\#end if}
     \item \code{\#for} \ldots \code{\#end for}
     \item \code{\#while} \ldots \code{\#end while}
     \item \code{\#break}
     \item \code{\#continue}
     \item \code{\#pass}
     \item \code{\#stop}
     \end{enumerate}

\item error/exception handling
     \begin{enumerate}
     \item \code{\#assert}
     \item \code{\#raise}
     \item \code{\#try} \ldots \code{\#except} \ldots \code{\#else} \ldots
          \code{\#end try} and  \code{\#finally}
     \item \code{\#errorCatcher} \ldots sets a default exception catcher/handler for
          exceptions raised by \$placeholder calls.
     \end{enumerate}

\item Instructions to the parser/compiler
     \begin{enumerate}
     \item \code{\#breakpoint}
     \item \code{\#compiler-settings} \ldots \code{\#end compiler-settings}
     \end{enumerate}
\end{enumerate}

Cheetah also supports two ASP (Active Server Pages) style tags as escapes to
pure Python code. These are not part of Cheetah's core language, but are
included to facilitate migration from ASP-style markup languages to Cheetah.
These tags may not be used inside other Cheetah tags.
\begin{enumerate}
\item evalute expression and print the output: \code{<\%=} \ldots \code{\%>} 
\item execute code and discard output: \code{<\%} \ldots \code{\%>}
\end{enumerate}


%%%%%%%%%%%%%%%%%%%%%%%%%%%%%%%%%%%%%%%%%%%%%%%%%%%%%%%%%%%%%%%%%%%%%%%%%%%%%%%%
\subsection{Placeholder Syntax Rules}
\label{language.placeholdersRules}

\begin{itemize} 
     
\item Placeholder names consist of one or more identifiers separated by periods.
     Each identifier follows the same rules as a Python variable name or
     attribute name: it must start with a letter or an underscore, and the
     subsequent characters must be letters, numbers or underscores.  An
     identifier may be followed by argument sets enclosed in ``()'' and/or
     key/subscript arguments in ``[]''.

\item Identifiers are case-sensitive. \code{\$var} does not equal \code{\$Var}
     or \code{\$vAr} or \code{\$VAR}.     
     
\item String literals inside argument sets must be quoted, just like in Python.
     All Python quoting styles are allowed.
     
\item All variable names inside argument sets should be prefixed with a \$.
     Use \$func(\$var) instead of \$func(var).  Python's builtin functions also
     follow this rule.

\item All argument names inside argument sets should be prefixed with a \$.
     Use \$func(\$arg=1234) instead of \$func(arg=1234).  This rule also applies
     for *arg and **kw forms: \$func(\$*args) instead of \$func(*args).
    
\item Trailing periods are ignored.  Cheetah will recognize that placeholder
     name in \code{\$varName.} is \code{varName} and the period will be left
     alone in the template output.
     
\item Placeholders can also be written in the form \code{\$\{placeholderName\}}
     and \code{\$(placeholderName)}.  This is useful for cases where there is no
     whitespace between the placeholder and surrounding text
     (\code{surrounding\$\{embeddedVar\}text}).
   
\item Cheetah ignores all dollar signs (\code{\$}) that are not followed by a
     letter or an underscore.  Cheetah also ignores any placeholder escaped by a
     backslash (\code{$\backslash$\$placeholderName}).

\end{itemize} 

The following are valid \$placeholders:
\begin{verbatim}
$a $_ $var $_var $var1 $_1var $var2_ $dict.key $list[3]
$object.method $object.method() $object.method
$nest($nest($var))
\end{verbatim}

These are not \$placeholders:
\begin{verbatim}
$@var $^var $15.50
\end{verbatim}

%%%%%%%%%%%%%%%%%%%%%%%%%%%%%%%%%%%%%%%%%%%%%%%%%%%%%%%%%%%%%%%%%%%%%%%%%%%%%%%%
\subsection{Directive Syntax Rules}
\label{language.directives}

Directives tags are used for all functionality that cannot be handled with
simple placeholders. Some directives consist of a single tag while others
consist of a pair of {\bf start} and {\bf end} tags that surround a chunk of
text.  End tags are written in the form \code{\#end [directive-name]}.
Directives are case-sensitive.

%%%%%%%%%%%%%%%%%%%%%%%%%%%%%%%%%%%%%%%%%%%%%%%%%%%%%%%%%%%%%%%%%%%%%%%%%%%%%%%%
\subsubsection{Escaping directives}
\label{language.directives.escaping}

Directives can be escaped by placing a backslash before them.  Escaped
directives will be printed verbatim.

%%%%%%%%%%%%%%%%%%%%%%%%%%%%%%%%%%%%%%%%%%%%%%%%%%%%%%%%%%%%%%%%%%%%%%%%%%%%%%%%
\subsubsection{Directive closures and whitespace handling}
\label{language.directives.closures}
Directive tags can closed explicitly with \code{\#} or implicitly with the end
of the line if you're feeling lazy.

\begin{verbatim}
#block testBlock #
Text in the contents area of the
block directive
#end block testBlock #
\end{verbatim}
is identical to:
\begin{verbatim}
#block testBlock
Text in the contents area of the
block directive
#end block testBlock
\end{verbatim}

When a directive tag is closed explicitly it can be followed with other text on
the same line:

\begin{verbatim}
bah, bah, #if $sheep.color == 'black'# black#end if # sheep.
\end{verbatim}

When a directive tag is closed implicitly with the end of the line all trailing
whitespace is gobbled, including the newline character:
\begin{verbatim}
"""
foo #set $x = 2 
bar
"""
outputs 
"""
foo bar
"""

while 
"""
foo #set $x = 2 #
bar
"""
outputs 
"""
foo 
bar
"""
\end{verbatim}

When a directive tag is closed implicitly AND there is no other text on the
line, the ENTIRE line will be gobbled up including any preceeding whitespace:
\begin{verbatim}
"""
foo 
   #set $x = 2 
bar
"""
outputs 
"""
foo
bar
"""

while 
"""
foo 
 - #set $x = 2
bar
"""
outputs 
"""
foo 
 - bar
"""
\end{verbatim}

The \code{\#slurp} directive, which is covered in more depth in section
\ref{directives.slurp} is a dummy directive that exists only to facilitate
gobbling of whitespace.

%%%%%%%%%%%%%%%%%%%%%%%%%%%%%%%%%%%%%%%%%%%%%%%%%%%%%%%%%%%%%%%%%%%%%%%%%%%%%%%%
\subsubsection{Variables in directives}
\label{language.directives.variables}

Variable names used inside a directive tag should be prefaced by \$, like in this
example:

\begin{verbatim}
#for $clientName, $address in $clients.addresses
$clientName: $address
#end for
\end{verbatim}


%%%%%%%%%%%%%%%%%%%%%%%%%%%%%%%%%%%%%%%%%%%%%%%%%%%%%%%%%%%%%%%%%%%%%%%%%%%%%%%%
\subsection{NameMapper Syntax}
\label{language.namemapper}

One of our core aims with Cheetah was to make it easy for non-programmers to
use. To achieve this aim, Cheetah uses a simplified syntax for mapping variable
names in Cheetah to values in Python. It's known as the{\bf NameMapper syntax}
and makes it possible for non-programmers to use Cheetah without knowing (a)
what the difference is between an object and a dictionary, (b) what functions
and methods are, and (c) what 'self' is. A side benefit is that NameMapper
syntax insulates the code in Cheetah templates from changes in the implementation
of the Python data structures behind them.

NameMapper syntax is used for all variables in Cheetah placeholders and
directives. If desired, it can be turned off via the \code{Template} class'
\code{'useNameMapper'} compiler setting.

%%%%%%%%%%%%%%%%%%%%%%%%%%%%%%%%%%%%%%%%%%%%%%%%%%%%%%%%%%%%%%%%%%%%%%%%%%%%%%%%
\subsubsection{Example}
\label{language.namemapper.example}

Consider this scenario:

You've been hired as a consultant to design and implement a customer information
system for your client. The class you create has a 'customers' method that
returns a dictionary of all the customer objects.  Each customer object has an
'address' method that returns the a dictionary with information about the
customer's address.

The designers working for your client want to use information from your system
on the client's website --AND-- they want to maintain the display code
themselves.

Using PSP, the display code for the website would look something like the
following, assuming your servlet subclasses the class you created for managing
customer information:


\begin{verbatim}
  <%= self.customer()[ID].address()['city'] %>   (42 chars)
\end{verbatim}

Using Cheetah's NameMapper syntax it could be any of the following:

\begin{verbatim}
   $self.customers()[$ID].address()['city']       (39 chars)
   --OR--                                         
   $customers()[$ID].address()['city']           
   --OR--                                         
   $customers()[$ID].address().city              
   --OR--                                         
   $customers()[$ID].address.city                
   --OR--                                         
   $customers()[$ID].address.city
   --OR--
   $customers[$ID].address.city                   (27 chars)                     
\end{verbatim}   

Which of these would you prefer to explain to the designers, who have no
programming experience?  The last form is 15 characters shorter than the PSP
and -- conceptually -- is far more accessible. With PHP or ASP, the code would be
even messier than the PSP

This is a rather extreme example and, of course, you could also just implement
\code{\$customer(\$ID).city} and obey the Law of Demeter (search Google for more
on that).  But good object orientated design isn't the point here.

%%%%%%%%%%%%%%%%%%%%%%%%%%%%%%%%%%%%%%%%%%%%%%%%%%%%%%%%%%%%%%%%%%%%%%%%%%%%%%%%
\subsubsection{Dictionary Access}
\label{language.namemapper.dict}

NameMapper syntax allows access to items in a dictionary using the same dotted
notation used to access object attributes in Python.  This aspect of NameMapper
syntax is known as 'Unified Dotted Notation'.

For example, with Cheetah it is possible to write:
\begin{verbatim}
   $customers()['kerr'].address()  --OR--  $customers().kerr.address()
\end{verbatim}
where the second form is in NameMapper syntax.

This only works with dictionary keys that are also valid python identifiers.

%%%%%%%%%%%%%%%%%%%%%%%%%%%%%%%%%%%%%%%%%%%%%%%%%%%%%%%%%%%%%%%%%%%%%%%%%%%%%%%%
\subsubsection{Autocalling}
\label{language.namemapper.autocalling}

Cheetah automatically detects functions and methods in Cheetah \$variables and calls
them if the parentheses have been left off.  

For example if 'a' is an object, 'b' is a method
\begin{verbatim}
  $a.b
\end{verbatim}

is equivalent to

\begin{verbatim}
  $a.b()
\end{verbatim}

If b returns a dictionary, then following variations are possible
\begin{verbatim}
  $a.b.c  --OR--  $a.b().c  --OR--  $a.b()['c']
\end{verbatim}
where 'c' is a key in the dictionary that a.b() returns.

Further notes:
\begin{itemize}
\item Cheetah autocalls the function or method without any arguments.  Thus
autocalling can only be used with functions or methods that either have no
arguments or have default values for all arguments.

\item Cheetah only autocalls functions and methods.  Classes and callable objects
will not be autocalled.  

\item Autocalling can be disabled using Cheetah's 'useAutocalling' setting.
\end{itemize}

%%%%%%%%%%%%%%%%%%%%%%%%%%%%%%%%%%%%%%%%%%%%%%%%%%%%%%%%%%%%%%%%%%%%%%%%%%%%%%%%
\subsubsection{Underscored attributes}
\label{language.namemapper.underscore}

If a 'name' in Cheetah doesn't correspond to a valid object attribute name in
Python, but there is an attribute in the form '_<name>' NameMapper will return
the underscored attribute.

Thus, it removes the need to change all placeholders like \code{\$clients.list} to
\code{\$clients._list} when the 'list' attribute of 'clients' is changed to underscored
attribute, and vice-versa.


%%%%%%%%%%%%%%%%%%%%%%%%%%%%%%%%%%%%%%%%%%%%%%%%%%%%%%%%%%%%%%%%%%%%%%%%%%%%%%%%
\subsubsection{Namespace cascading and the searchList}
\label{language.namemapper.searchList}

When Cheetah maps variable name in a template to a Python value it searches
through a list of namespaces known as the Template object's \code{searchList}.
By default, the only namespace in the \code{searchList} is the template object
itself. This means that its attributes or methods can be accessed in templates
via \code{\$placeholders} without needing to include 'self' in the reference as
you do in Python. 


The \code{searchList} can be used to override or supplement the variables in the
template object's namespace without needing to create a subclass.  When Cheetah
fills in \code{\$myVar} it searches sequentially through the searchList until it
finds a value for \code{myVar}.  Thus, if three namespaces are loaded and two of
them contain a value for \code{\$myVar}, the value for \code{myVar} from the
namespace that is closest to the start of the searchList will be returned.

If you add a Python object to the searchList, its attributes and methods will be
accessible as \$placeholder names.  For example, \code{myObject} contains
\code{myAttrib} and \code{myMethod}.  If \code{myObject} is added to the
searchList, \code{\$myAttrib} and \code{\$myMethod} can be used as placeholder
names.

The\code{__init__} method of the \code{Template} class, and the subclasses
generated by the Cheetah compiler, accept a list of namespaces that will be
added to the start of the \code{searchList}.  Extra namespaces can be added to
the end of the searchList at at any time using the
\code{Template.addToSearchList()} method.

If you want to make absolutely sure that you are accessing an attribute or
method of the template object, include 'self' in the reference:
\code{\$self.attrib1} instead of \code{\$attrib1}.

%%%%%%%%%%%%%%%%%%%%%%%%%%%%%%%%%%%%%%%%%%%%%%%%%%%%%%%%%%%%%%%%%%%%%%%%%%%%%%%%
\subsubsection{Missing values}
\label{language.namemapper.missing}

If NameMapper can not find a Python value for a Cheetah variable name it will
raise the NameMapper.NotFound exception.

% Local Variables:
% TeX-master: "users_guide"
% End:      
