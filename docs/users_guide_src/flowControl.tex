%%%%%%%%%%%%%%%%%%%%%%%%%%%%%%%%%%%%%%%%%%%%%%%%%%%%%%%%%%%%%%%%%%%%%%%%%%%%%%%%
\section{Flow Control}
\label{flowControl}


%%%%%%%%%%%%%%%%%%%%%%%%%%%%%%%%%%%%%%%%%%%%%%%%%%%%%%%%%%%%%%%%%%%%%%%%%%%%%%%%
\subsection{\#for ... \#end for}
\label{flowControl.for}

The \code{\#for} directive iterates through a set of data.

Here's how to create list of numbers separated by hyphens. This ``\#end for''
tag shares the last line to avoid introducing a newline character after each
hyphen.  
\begin{verbatim}
#for $i in range(15)
$i - #end for
\end{verbatim}

The previous example will put an extra hyphen after last number.  Here's how to
get around that problem, using the \code{\#set} directive, which will be dealt
with in more detail below.
\begin{verbatim}
#set $sep = '' 
#for $name in $names 
$sep$name 
#set $sep = ', ' 
#end for 
\end{verbatim}

Here's how to loop through a dictionaries keys and values:
\begin{verbatim}
#for $key, $value in $dict
$key - $value
#end for
\end{verbatim}

Here's how to create a simple client listing:
\begin{verbatim}
<TABLE>
#for $client in $service.clients
<TR>
<TD>$client.surname, $client.firstname</TD>
<TD><A HREF="mailto:$client.email" >$client.email</A></TD>
</TR>
#end for
</TABLE>
\end{verbatim}


%%%%%%%%%%%%%%%%%%%%%%%%%%%%%%%%%%%%%%%%%%%%%%%%%%%%%%%%%%%%%%%%%%%%%%%%%%%%%%%%
\subsection{\#repeat ... \#end repeat}
\label{flowControl.repeat}


%%%%%%%%%%%%%%%%%%%%%%%%%%%%%%%%%%%%%%%%%%%%%%%%%%%%%%%%%%%%%%%%%%%%%%%%%%%%%%%%
\subsection{\#if ... \#else if ... \#else ... \#end if}
\label{flowControl.if}

The \code{\#if} directives and its kin are used to display a portion of text
conditionally. \code{\#if} and \code{\#else if} should be followed by a
True/False expression, while \code{\#else} should not.  Any Python valid
expression is allowed.  As in Python, the expression is true unless it evaluates
to 0, '', None, an empty list, or an empty dictionary. \code{\#elif} is accepted
as a synonym for {\#else if}.

Here are some examples:
\begin{verbatim}
#if $size >= 1500
It's big
#else if $size < 1500 and $size > 0 
It's small
#else
It's not there
#end if
\end{verbatim}

\begin{verbatim}
#if $testItem($item)
The item $item.name is OK.
#end if
\end{verbatim}

Here's an example that combines an \code{\#if} tag with a \code{\#for} tag.
\begin{verbatim}
#if $people
<TABLE>
<TR>
<TH>Name</TH>
<TH>Address</TH>
<TH>Phone</TH>
</TR>
#for $p in $people
<TR>
<TD>$p.name</TD>
<TD>$p.address</TD>
<TD>$p.phone</TD>
</TR>
#end for
</TABLE>
#else
<P> Sorry, the search did not find any people. </P>
#end if
\end{verbatim}


%%%%%%%%%%%%%%%%%%%%%%%%%%%%%%%%%%%%%%%%%%%%%%%%%%%%%%%%%%%%%%%%%%%%%%%%%%%%%%%%
\subsection{\#unless ... \#end unless}
\label{flowControl.unless}


%%%%%%%%%%%%%%%%%%%%%%%%%%%%%%%%%%%%%%%%%%%%%%%%%%%%%%%%%%%%%%%%%%%%%%%%%%%%%%%%
\subsection{\#break and \#continue}
\label{flowControl.break}

These directives are use as they are used in Python. \code{\#break} will
prematurely exit a \code{\#for} loop, while \code{\#continue} will immediately
jump to the next step in the \code{\#for} loop.

In this example the output list will not contain ``10 - ''. 
\begin{verbatim}
#for $i in range(15)
#if $i == 10
  #continue
#end if
$i - #slurp
#end for
\end{verbatim}

In this example the loop will exit if it finds a name that equals 'Joe':
\begin{verbatim}
#for $name in $names
#if $name == 10
  #break
#end if
$name - #slurp
#end for
\end{verbatim}


%%%%%%%%%%%%%%%%%%%%%%%%%%%%%%%%%%%%%%%%%%%%%%%%%%%%%%%%%%%%%%%%%%%%%%%%%%%%%%%%
\subsection{\#pass}
\label{flowControl.break}

The \code{\#pass} directive is identical to Python \code{pass} statement: it
does nothing. It can be used when a statement is required syntactically but the
program requires no action.

In this example the output list will not contain ``10 - ''. 
\begin{verbatim}
#if $A and $B:
   do something
#elif $A:
  #pass
#elif $B: 
  do something
#else:
  do something
#end if
\end{verbatim}

%%%%%%%%%%%%%%%%%%%%%%%%%%%%%%%%%%%%%%%%%%%%%%%%%%%%%%%%%%%%%%%%%%%%%%%%%%%%%%%%
\subsection{\#stop}
\label{flowControl.stop}

The \code{\#stop} directive is used to stop processing of a template at a
certain point and return ONLY what has been processed up to that point.  When a
\code{\#stop} is called inside an \code{\#include} it will only stop the
processing of the included code.  Likewise, when \code{\#stop} is called inside
a \code{\#def} or \code{\#block}, it stops only the \code{\#def} or
\code{\#block}.

\begin{verbatim}
the beginning
#if 1
  inside the if block
  #stop
#end if
the end
\end{verbatim}

will print 
\begin{verbatim}
the beginning
  inside the if block
\end{verbatim}

instead of 
\begin{verbatim}
the beginning
  inside the if block
the end
\end{verbatim}



% Local Variables:
% TeX-master: "users_guide"
% End:      
