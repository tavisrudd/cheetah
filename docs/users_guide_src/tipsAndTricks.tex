%%%%%%%%%%%%%%%%%%%%%%%%%%%%%%%%%%%%%%%%%%%%%%%%%%%%%%%%%%%%%%%%%%%%%%%%%%%%%%%%
\section{Miscellaneous Placeholder and Directive Tips}
\label{tips}

%%%%%%%%%%%%%%%%%%%%%%%%%%%%%%%%%%%%%%%%%%%%%%%%%%%%%%%%%%%%%%%%%%%%%%%%%%%%%%%%
\subsection{ASP-style tags}
\label{tips.ASP}

%%%%%%%%%%%%%%%%%%%%%%%%%%%%%%%%%%%%%%%%%%%%%%%%%%%%%%%%%%%%%%%%%%%%%%%%%%%%%%%%
\subsection{Placeholder Tips}
\label{tips.placeholder}

Here's how to do certain important lookups that may not be obvious.
For each, we show first the Cheetah expression and then the Python equivalent,
because you can use these either in templates or in pure Python mixin classes.
The Cheetah examples use NameMapper shortcuts (uniform dotted notation, 
autocalling) as much as possible.

To verify whether a variable exists in the searchList:
\begin{verbatim}
$varExists('theVariable')
self.varExists('theVariable')
\end{verbatim}
This is useful in \code{\#if} or \code{\#unless} constructs to avoid a
\code{\#NameMapper.NotFound} error if the variable doesn't exist.  For instance,
a CGI GET parameter that is normally supplied but in this case the user typed
the URL by hand and forgot the parameter (or didn't know about it).

To look up a variable in the searchList from a Python method:
\begin{verbatim}
self.getVar('theVariable')
self.getVar('theVariable', None)
self.getVar('theVariable', myDefault)
\end{verbatim}
This is the equivalent to \code{\$theVariable} in the template.  \code{getVar}
returns the second argument (\code{None} or \code{myDefault} if the variable is
missing; or, if there is no second argument, it raises raises
\code{NameMapper.NotFound}.  However, it usually easier to write your method
so that all needed searchList values come in as method arguments.  That way
the caller can just use a \$placeholder to specify the argument, which is
less verbose than you writing a getVar call.

To do a ``safe'' placeholder lookup that returns a default value if the
variable is missing:
\begin{verbatim}
$getVar('theVariable', None)
$getVar('theVariable', $myDefault)
\end{verbatim}

To get an environmental variable, put \code{os.environ} as one of the
elements in the searchList.  Or read the envvar in Python code and set a
placeholder variable for it.

Remember that variables found earlier in the searchList override same-name
variables located in a later searchList object.  Be careful when adding objects
containing other variables besides the ones you want (e.g., \code{os.environ},
CGI parameters).  The "other" variables may override variables your application
depends on, leading to hard-to-find bugs.  Also, users can inadvertantly or
maliciously set an environmental variable or CGI parameter you didn't expect,
screwing up your program.  To avoid all this, know what your namespaces
contain, and place the namespaces you have the most control over first.  For
namespaces that could contain user-supplied "other" variables, don't put the
namespace itself in the searchList; instead, copy the needed variables into
your own "safe" namespace.

% @@ MO: If getVar() is called from Python, does errorCatcher apply?


% Local Variables:
% TeX-master: "users_guide"
% End:      
