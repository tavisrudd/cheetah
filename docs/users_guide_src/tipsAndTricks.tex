%%%%%%%%%%%%%%%%%%%%%%%%%%%%%%%%%%%%%%%%%%%%%%%%%%%%%%%%%%%%%%%%%%%%%%%%%%%%%%%%
\section{Tips, Tricks and Troubleshooting}
\label{tips}

This chapter contains short stuff that doesn't fit anywhere else.

See the Cheetah FAQ for more specialized issues and for troubleshooting tips.
Check the wiki periodically for recent tips contributed by users.  If you
get stuck and none of these resources help, ask on the mailing list.

%%%%%%%%%%%%%%%%%%%%%%%%%%%%%%%%%%%%%%%%%%%%%%%%%%%%%%%%%%%%%%%%%%%%%%%%%%%%%%%%
\subsection{Placeholder Tips}
\label{tips.placeholder}

Here's how to do certain important lookups that may not be obvious.
For each, we show first the Cheetah expression and then the Python equivalent,
because you can use these either in templates or in pure Python subclasses.
The Cheetah examples use NameMapper shortcuts (uniform dotted notation, 
autocalling) as much as possible.

To verify whether a variable exists in the searchList:
\begin{verbatim}
$varExists('theVariable')
self.varExists('theVariable')
\end{verbatim}
This is useful in \code{\#if} or \code{\#unless} constructs to avoid a
\code{\#NameMapper.NotFound} error if the variable doesn't exist.  For instance,
a CGI GET parameter that is normally supplied but in this case the user typed
the URL by hand and forgot the parameter (or didn't know about it).
(\code{.hasVar} is a synonym for \code{.varExists}.)

To look up a variable in the searchList from a Python method:
\begin{verbatim}
self.getVar('theVariable')
self.getVar('theVariable', myDefault)
\end{verbatim}
This is the equivalent to \code{\$theVariable} in the template.  If the 
variable is missing, it returns the second argument, \code{myDefault}, if 
present, or raises \code{NameMapper.NotFound} if there is no second argument.
However, it usually easier to write your method so that all needed searchList
values come in as method arguments.  That way the caller can just use a
\code{\$placeholder} to specify the argument, which is less verbose than you
writing a getVar call.

To do a ``safe'' placeholder lookup that returns a default value if the
variable is missing:
\begin{verbatim}
$getVar('theVariable', None)
$getVar('theVariable', $myDefault)
\end{verbatim}

To get an environmental variable, put \code{os.environ} on the searchList as a
container.  Or read the envvar in Python code and set a placeholder variable
for it.

Remember that variables found earlier in the searchList override same-name
variables located in a later searchList object.  Be careful when adding objects
containing other variables besides the ones you want (e.g., \code{os.environ},
CGI parameters).  The "other" variables may override variables your application
depends on, leading to hard-to-find bugs.  Also, users can inadvertently or
maliciously set an environmental variable or CGI parameter you didn't expect,
screwing up your program.  To avoid all this, know what your namespaces
contain, and place the namespaces you have the most control over first.  For
namespaces that could contain user-supplied "other" variables, don't put the
namespace itself in the searchList; instead, copy the needed variables into
your own "safe" namespace.

%%%%%%%%%%%%%%%%%%%%%%%%%%%%%%%%%%%%%%%%%%%%%%%%%%%%%%%%%%%%%%%%%%%%%%%%%%%%%%%%
\subsection{Diagnostic Output}
\label{tips.diagnostic}

If you need send yourself some debugging output, you can use \code{\#silent} to
output it to standard error:
\begin{verbatim}
#silent $sys.stderr.write("Incorrigible var is '$incorrigible'.\n")
#silent $sys.stderr.write("Is 'unknown' in the searchList? " +
    $getVar("unknown", "No.") + "\n" )
\end{verbatim}
(Tip contributed by Greg Czajkowski.)

%%%%%%%%%%%%%%%%%%%%%%%%%%%%%%%%%%%%%%%%%%%%%%%%%%%%%%%%%%%%%%%%%%%%%%%%%%
\subsection{When to use Python methods}
\label{tips.pythonMethods}

You always have a choice whether to code your methods as Cheetah \code{\#def}
methods or Python methods (the Python methods being located in a class your
template inherits).  So how do you choose?

Generally, if the method consists mostly of text and placeholders, use a 
Cheetah method (a \code{\#def} method).  That's why \code{\#def} exists, to
take the tedium out of writing those kinds of methods.  And if you have a
couple \code{\#if} stanzas to \code{\#set} some variables, followed by a
\code{\#for} loop, no big deal.  But if your method consists mostly of
directives and only a little text, you're better off writing it in Python.
Especially be on the watch for extensive use of \code{\#set}, \code{\#echo} and
\code{\#silent} in a Cheetah method--it's a sure sign you're probably using the
wrong language.  Of course, though, you are free to do so if you wish.  

Another thing that's harder to do in Cheetah is adjacent or nested
multiline stanzas (all those directives with an accompanying \code{\#end}
directive).  Python uses indentation to show the beginning and end of nested
stanzas, but Cheetah can't do that because any indentation shows up in the
output, which may not be desired.  So unless all those extra spaces and tabs
in the output are acceptable, you have to keep directives flush with the left
margin or the preceding text.  

The most difficult decisions come when you have conflicting goals.  What if
a method generates its output in parts (i.e., output concatenation), contains
many searchList placeholders and lots of text, {\em and} requires lots of
\code{\#if \ldots \#set \ldots \#else \#set \ldots \#end if} stanzas.  A Cheetah
method would be more advantageous in some ways, but a Python method in others.
You'll just have to choose, perhaps coding groups of methods all the same
way.  Or maybe you can split your method into two, one Cheetah and one Python,
and have one method call the other.  Usually this means the Cheetah method
calling the Python method to calculate the needed values, then the Cheetah
method produces the output.  One snag you might run into though is that
\code{\#set} currently can set only one variable per statement, so if your
Python method needs to return multiple values to your Cheetah method, you'll
have to do it another way.



%%%%%%%%%%%%%%%%%%%%%%%%%%%%%%%%%%%%%%%%%%%%%%%%%%%%%%%%%%%%%%%%%%%%%%%%%%%%%%%%
\subsection{Calling superclass methods, and why you have to}
\label{tips.callingSuperclassMethods}

If your template or pure Python class overrides a standard method or attribute
of \code{Template} or one of its base classes, you should call the superclass
method in your method to prevent various things from breaking.  The most common
methods to override are \code{.awake} and \code{.\_\_init\_\_}.  \code{.awake}
is called automatically by Webware early during the web transaction, so it makes
a convenient place to put Python initialization code your template needs.
You'll definitely want to call the superclass \code{.awake} because it sets up
many wonderful attributes and methods, such as those to access the CGI input
fields.

There's nothing Cheetah-specific to calling superclass methods, but
because it's vital, we'll recap the standard Python techniques
here. We mention only the solution for old-style classes because
Cheetah classes are old-style (in other Python documentation, you will
find the technique for new-style classes, but they are not listed here
because they cannot be used with Cheetah if you use
dynamically-compiled templates).

\begin{verbatim}
from Cheetah.Template import Template
class MyClass(Template):
	def awake(self, trans):
		Template.awake(self, trans)
		... great and exciting features written by me ...
\end{verbatim}

[ @@MO: Need to test this.  .awake is in Servlet, which is a superclass
of Template.  Do we really need both imports?  Can we call
Template.awake? ]

To avoid hardcoding the superclass name, you can use this
function \code{callbase()}, which emulates \code{super()} for older versions of
Python.  It also works even \code{super()} does exist, so you don't have to
change your servlets immediately when upgrading.  Note that the argument
sequence is different than \code{super} uses.

\begin{verbatim}
===========================================================================
# Place this in a module SOMEWHERE.py .  Contributed by Edmund Lian.
class CallbaseError(AttributeError):
    pass

def callbase(obj, base, methodname='__init__', args=(), kw={},
    raiseIfMissing=None):
    try: method = getattr(base, methodname)
    except AttributeError:
        if raiseIfMissing:
            raise CallbaseError, methodname
        return None
    if args is None: args = ()
    return method(obj, *args, **kw)
===========================================================================
# Place this in your class that's overriding .awake (or any method).
from SOMEWHERE import callbase
class MyMixin:
        def awake(self, trans):
                args = (trans,)
                callbase(self, MyMixin, 'awake', args)
                ... everything else you want to do ...
===========================================================================
\end{verbatim}

% @@MO: Edmund wants us to mention delegation too, as an alternative to
% inheritance.  Contact elian@inbrief.net for details.

%%%%%%%%%%%%%%%%%%%%%%%%%%%%%%%%%%%%%%%%%%%%%%%%%%%%%%%%%%%%%%%%%%%%%%%%%%%%%%%%
\subsection{All methods}
\label{tips.allMethods}

Here is a list of all the standard methods and attributes that can be accessed
from a placeholder.  Some of them exist for you to call, others are mainly used
by Cheetah internally but you can call them if you wish, and others are only
for internal use by Cheetah or Webware.  Do not use these method names in mixin
classes (\code{\#extends}, section \ref{inheritanceEtc.extends}) unless you
intend to override the standard method.  

Variables with a star prefix ({\bf *}) are frequently used in templates or in
pure Python classes.

\paragraph*{Inherited from Cheetah.Template}

\begin{description}
\item{{\bf compile(source=None, file=None, moduleName=None, 
    mainMethodName='respond')}}  Compile the template.  Automatically called
    by \code{.\_\_init\_\_}.
\item{{\bf generatedModuleCode()}}  Return the module code the compiler
    generated, or \code{None} if no compilation took place.
\item{{\bf generatedClassCode()}}   Return the class code the compiler
    generated, or \code{None} if no compilation took place.
\item{{\bf * searchList()}}  Return a reference to the underlying search list.
    (a list of objects).  Use this to print out your searchList for debugging.
    Modifying the returned list will affect your placeholder searches!
\item{{\bf * errorCatcher()}}  Return a reference to the current error
    catcher.
\item{{\bf * refreshCache(cacheKey=None)}}  If 'cacheKey' is not \code{None},
    refresh that item in the cache.  If \code{None}, delete all items in the
    cache so they will be recalculated the next time they are encountered.
\item{{\bf * shutdown()}}  Break reference cycles before discarding a servlet.
\item{{\bf * getVar(varName, default=NoDefault, autoCall=True)}}  Look up a
    variable in the searchList.  Same as \code{\$varName} but allows you to
    specify a default value and control whether autocalling occurs.
\item{{\bf * varExists(varName, autoCall=True)}}
\item{{\bf * getFileContents(path)}}  Read the named file.  If used as a
    placeholder, inserts the file's contents in the output without
    interpretation, like \code{\#include\ raw}.  If used in an expression,
    returns the file's content (e.g., to assign it to a variable).
\item{{\bf runAsMainProgram()}}  This is what happens if you run a
    .py template module as a standalone program.
\end{description}

%Private methods: {\bf \_bindCompiledMethod}, {\bf \_bindFunctionAsMethod},
%{\bf \_includeCheetahSource}, {\bf \_genTmpFilename}, 
%{\bf \_importAsDummyModule}, {\bf \_makeDummyPackageForDir},
%{\bf \_importFromDummyPackage}, {\bf \_importModuleFromDirectory}.
%
%Other private attributes:
%\begin{description}
%\item{{\bf * \_fileMtime}}  Time the template definition was modified, in
%    Unix ticks.  \code{None} if the template definition came from a string or
%    file handle rather than a named file, same for the next three variables.
%\item{{\bf * \_fileDirName}}  The directory containing the template definition.
%\item{{\bf * \_fileBaseName}}  The basename of the template definition  file.
%\item{{\bf * \_filePath}}  The directory+filename of the template definition.
%\end{description}

\paragraph*{Inherited from Cheetah.Utils.WebInputMixin}

\begin{description}
\item{{\bf nonNumericInputError}}  Exception raised by \code{.webInput}.
\item{{\bf * webInput(...)}}  Convenience method to access GET/POST variables
    from a Webware servlet or CGI script, or Webware cookie or session
    variables.  See section \ref{webware.webInput} for usage information.
\end{description}

\paragraph*{Inherited from Cheetah.SettingsManager}

\begin{description}
\item{{\bf setting(name, default=NoDefault)}}  Get a compiler setting.
\item{{\bf hasSetting(name)}}  Does this compiler setting exist?
\item{{\bf setSetting(name, value)}}  Set setting 'name' to 'value'.
    See \code{\#compiler-settings}, section 
    \ref{parserInstructions.compiler-settings}.
\item{{\bf settings()}}  Return the underlying settings dictionary.  (Warning:
    modifying this dictionary will change Cheetah's behavior.)
\item{{\bf copySettings()}}  Return a copy of the underlying settings
    dictionary.
\item{{\bf deepcopySettings()}}  Return a deep copy of the underlying settings
    dictionary.  See Python's \code{copy} module.
\item{{\bf updateSettings(newSettings, merge=True)}}  Update Cheetah's 
    compiler settings from the 'newSettings' dictionary.  If 'merge' is true,
    update only the names in newSettings and leave the other names alone.
    (The SettingsManager is smart enough to update nested dictionaries one
    key at a time rather than overwriting the entire old dictionary.)
    If 'merge' is false, delete all existing settings so that the new ones are
    the only settings.
\item{{\bf updateSettingsFromPySrcStr(theString, merge=True)}}  Same, 
    but pass a string of \code{name=value} pairs rather
    than a dictionary, the same as you would provide in a
    \code{\#compiler-settings} directive, section
    \ref{parserInstructions.compiler-settings}.
\item{{\bf updateSettingsFromPySrcFile(path, merge=True)}}  Same, but
    exec a Python source file and use the variables it contains as the new
    settings.  (e.g., \code{cheetahVarStartToken\ =\ "@"}).
\item{{\bf updateSettingsFromConfigFile(path, **kw)}}  Same, but get the new
    settings from a text file in ConfigParser format (similar to Windows'
    *.ini file format).  See Python's \code{ConfigParser} module.
\item{{\bf updateSettingsFromConfigFileObj}}  Same, but read the open file
    object 'inFile' for the new settings.  
\item{{\bf updateSettingsFromConfigStr(configStr, convert=True, merge=True}}
    Same, but read the new settings from a string in ConfigParser format.
\item{{\bf writeConfigFile(path)}}  Write the current compiler settings to
    a file named 'path' in *.ini format.
\item{{\bf getConfigString()}}  Return a string containing the current
    compiler settings in *.ini format.
\end{description}

\paragraph*{Inherited from Cheetah.Servlet}

{\em Do not override these in a subclass or assign to them as attributes
if your template will be used as a servlet,} otherwise Webware will behave
unpredictably.  However, it {\em is} OK to put same-name variables in the
searchList, because Webware does not use the searchList.

EXCEPTION: It's OK to override {\bf awake} and {\bf sleep} as long as you
call the superclass methods.  (See section
\ref{tips.callingSuperclassMethods}.)

\begin{description}
\item{{\bf * isControlledByWebKit}}  True if this template instance is
    part of a live transaction in a running WebKit servlet.
\item{{\bf * isWebwareInstalled}}  True if Webware is installed and the
    template instance inherits from WebKit.Servlet.  If not, it inherits
    from Cheetah.Servlet.DummyServlet.
\item{{\bf * awake(transaction)}}  Called by WebKit at the beginning of
    the web transaction.
\item{{\bf * sleep(transaction)}}  Called by WebKit at the end of the
    web transaction.
\item{{\bf * respond(transaction)}}  Called by WebKit to produce the 
    web transaction content.  For a template-servlet, this means
    filling the template.
\item{{\bf shutdown()}}  Break reference cycles before deleting instance.
\item{{\bf * serverSidePath()}}  The filesystem pathname of the 
    template-servlet (as opposed to the URL path).
\item{{\bf transaction}}  The current Webware transaction.
\item{{\bf application}}  The current Webware application.
\item{{\bf response}}  The current Webware response.
\item{{\bf request}}  The current Webware request.
\item{{\bf session}}  The current Webware session.
\item{{\bf write}}  Call this method to insert text in the filled template
    output.
\end{description}

Several other goodies are available to template-servlets under the 
\code{request} attribute, see section \ref{webware.input}.

\code{transaction}, \code{response}, \code{request} and \code{session} are
created from the current transaction when WebKit calls \code{awake}, and don't
exist otherwise.  Calling \code{awake} yourself (rather than letting WebKit
call it) will raise an exception because the \code{transaction} argument won't
have the right attributes.  

\paragraph*{Inherited from WebKit.Servlet}
These are accessible only if Cheetah knows Webware is installed.  This 
listing is based on a CVS snapshot of Webware dated 22 September 2002, and
may not include more recent changes.

The same caveats about overriding these methods apply.

\begin{description}
\item{name()}  The simple name of the class.  Used by Webware's logging and
    debugging routines.
\item{log()}  Used by Webware's logging and debugging routines.
\item{canBeThreaded()}  True if the servlet can be multithreaded.
\item{canBeReused()}  True if the servlet can be used for another transaction
    after the current transaction is finished.
\item{serverSideDir()}  Depreciated by \code{.serverSidePath()}.
\end{description}

%%%%%%%%%%%%%%%%%%%%%%%%%%%%%%%%%%%%%%%%%%%%%%%%%%%%%%%%%%%%%%%%%%%%%%%%%%%%%%%%
\subsection{Optimizing templates}
\label{tips.optimizing}

Here are some things you can do to make your templates fill faster and user
fewer CPU cycles.  Before you put a lot of energy into this, however, make
sure you really need to.  In many situations, templates appear to initialize
and fill instantaneously, so no optimization is necessary.  If you do find a
situation where your templates are filling slowly or taking too much memory or
too many CPU cycles, we'd like to hear about it on the mailing list.

Cache \$placeholders whose values don't change frequently.  (Section
\ref{output.caching}).

Use \code{\#set} for values that are very frequently used, especially if they
come out of an expensive operation like a deeply.nested.structure or a database
lookup.  \code{\#set} variables are set to Python local variables, which have a
faster lookup time than Python globals or values from Cheetah's searchList.

Moving variable lookups into Python code may provide a speedup in certain
circumstances.  If you're just reading \code{self} attributes, there's no
reason to use NameMapper lookup (\$placeholders) for them.  NameMapper does
a lot more work than simply looking up a \code{self} attribute.  

On the other hand, if you don't know exactly where the value will come from
(maybe from \code{self}, maybe from the searchList, maybe from a CGI input
variable, etc), it's easier to just make that an argument to your method, and
then the template can handle all the NameMapper lookups for you:
\begin{verbatim}
#silent $myMethod($arg1, $arg2, $arg3)
\end{verbatim}
Otherwise you'd have to call \code{self.getVar('arg1')} etc in your
method, which is more wordy, and tedious.



%%%%%%%%%%%%%%%%%%%%%%%%%%%%%%%%%%%%%%%%%%%%%%%%%%%%%%%%%%%%%%%%%%%%%%%%%%%%%%%%
\subsection{PSP-style tags}
\label{tips.PSP}

\code{<\%= \ldots \%>} and \code{<\% \ldots \%>} allow an escape
to Python syntax inside the template.  You do not need it to use Cheetah
effectively, and we're hard pressed to think of a case to recommend it.
Nevertheless, it's there in case you encounter a situation you can't
express adequately in Cheetah syntax.  For instance, to set a local
variable to an elaborate initializer.  

\code{<\%= \ldots \%>} encloses a Python expression whose result will
be printed in the output.  

\code{<\% \ldots \%>} encloses a Python statement or expression (or set of
statements or expressions) that will be included as-is into the generated
method.  The statements themselves won't produce any output, but you can use
the local function \code{write(EXPRESSION)} to produce your own output.
(Actually, it's a method of a file-like object, but it looks like a local
function.) This syntax also may be used to set a local variable with a
complicated initializer.

To access Cheetah services, you must use Python code like you would in an
inherited Python class.  For instance, use \code{self.getVar()} to look up
something in the searchList.

{\em Warning:} {\bf No error checking is done!}  If you write:
\begin{verbatim}
<% break %>      ## Wrong!
\end{verbatim}
you'll get a \code{SyntaxError} when you fill the template, but that's what you
deserve.

Note that these are PSP-{\em style} tags, not PSP tags.  A Cheetah template
is not a PSP document, and you can't use PSP commands in it.



%%%%%%%%%%%%%%%%%%%%%%%%%%%%%%%%%%%%%%%%%%%%%%%%%%%%%%%%%%%%%%%%%%%%%%%%%%%%%%%%
\subsection{Makefiles}
\label{tips.Makefile}

If your project has several templates and you get sick of typing
``cheetah compile FILENAME.tmpl'' all the time--much less remembering which
commands to type when--and your system has the \code{make}
command available, consider building a Makefile to make your life easier.

Here's a simple Makefile that controls two templates, ErrorsTemplate and
InquiryTemplate.  Two external commands, \code{inquiry} and \code{receive},
depend on ErrorsTemplate.py.  Aditionally, InquiryTemplate
itself depends on ErrorsTemplate.

\begin{verbatim}
all:  inquiry  receive

.PHONY:  all  receive  inquiry  printsource

printsource:
        a2ps InquiryTemplate.tmpl ErrorsTemplate.tmpl

ErrorsTemplate.py:  ErrorsTemplate.tmpl
        cheetah compile ErrorsTemplate.tmpl

InquiryTemplate.py:  InquiryTemplate.tmpl ErrorsTemplate.py
        cheetah compile InquiryTemplate.tmpl

inquiry: InquiryTemplate.py  ErrorsTemplate.py

receive: ErrorsTemplate.py
\end{verbatim}

Now you can type \code{make} anytime and it will recompile all the templates
that have changed, while ignoring the ones that haven't.  Or you can
recompile all the templates \code{receive} needs by typing \code{make receive}.
Or you can recompile only ErrorsTemplate by typing
\code{make ErrorsTemplate}.  There's also another target, ``printsource'':
this sends a Postscript version of the project's source files to the printer.
The .PHONY target is explained in the \code{make} documentation; essentially,
you have it depend on every target that doesn't produce an output file with
the same name as the target.


%%%%%%%%%%%%%%%%%%%%%%%%%%%%%%%%%%%%%%%%%%%%%%%%%%%%%%%%%%%%%%%%%%%%%%%%%%%%%%%%
\subsection{Using Cheetah in a Multi-Threaded Application}
\label{tips.threads}

Template classes may be shared freely between threads.  However, template
instances should not be shared unless you either:
\begin{itemize}
\item Use a lock (mutex) to serialize template fills, to prevent two threads
from filling the template at the same time.
\item Avoid thread-unsafe features:
    \begin{itemize}
    \item Modifying searchList values or instance variables.
    \item Caching (\code{\$*var}, \code{\#cache}, etc).
    \item \code{\#set global}, \code{\#filter}, \code{\#errorCatcher}.
    \end{itemize}
    Any changes to these in one thread will be visible in other threads,
    causing them to give inconsistent output.
\end{itemize}

About the only advantage in sharing a template instance is building up the
placeholder cache.  But template instances are so low overhead that it
probably wouldn't take perceptibly longer to let each thread instantiate its
own template instance.  Only if you're filling templates several times a
second would the time difference be significant, or if some of the placeholders
trigger extremely slow calculations (e.g., parsing a long text file each time).
The biggest overhead in Cheetah is importing the \code{Template} module in
the first place, but that has to be done only once in a long-running
application.

You can use Python's \code{mutex} module for the lock, or any similar
mutex.  If you have to change searchList values or instance variables
before each fill (which is usually the case), lock the mutex before 
doing this, and unlock it only after the fill is complete.

For Webware servlets, you're probably better off using Webware's servlet
caching rather than Cheetah's caching.  Don't override the servlet's
\code{.canBeThreaded()} method unless you avoid the unsafe operations
listed above.


%%%%%%%%%%%%%%%%%%%%%%%%%%%%%%%%%%%%%%%%%%%%%%%%%%%%%%%%%%%%%%%%%%%%%%%%%%%%%%%%
\subsection{Using Cheetah with gettext}
\label{tips.gettext}

{\bf gettext} is a project for creating internationalized applications.  For 
more details, visit \url{http://docs.python.org/lib/module-gettext.html}.
gettext can be used with Cheetah to create internationalized applications, even
for CJK character sets, but you must keep a couple things in mind:
\begin{itemize}
\item xgettext is used on compiled templates, not on the templates themselves.
\item The way the NameMapper syntax gets compiled to Python gets in the way of 
the syntax that xgettext recognizes.  Hence, a special case exists for the
functions \code{_}, \code{N_}, and \code{ngettext}.  If you need to use a
different set of functions for marking strings for translation, you must set
the Cheetah setting \code{gettextTokens} to a list of strings representing the
names of the functions you are using to mark strings for translation.
\end{itemize}


% Local Variables:
% TeX-master: "users_guide"
% End:      
