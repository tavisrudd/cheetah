%%%%%%%%%%%%%%%%%%%%%%%%%%%%%%%%%%%%%%%%%%%%%%%%%%%%%%%%%%%%%%%%%%%%%%%%%%%%%%%%
\section{Template and other libraries}
\label{libraries}

Cheetah comes ``batteries included'' with libraries of templates, functions,
classes and other objects you can use in your own programs.  The different
types are listed alphabetically below, followed by a longer description of
the SkeletonPage framework.  Some of the objects are classes for specific
purposes (e.g., filters or error catchers), while others are standalone and
often can be used without Cheetah.

If you develop any objects which are generally useful for Cheetah sites,
please consider posting them on the mailing list so that we can incorporate
them into the standard library.  That way, all Cheetah users will benefit, and
it will encourage others to contribute their objects, which might include
something you want.

%% @@MSO: If sed-filters get added to Cheetah, mention them here.

%%%%%%%%%%%%%%%%%%%%%%%%%%%%%%%%%%%%%%%%%%%%%%%%%%%%%%%%%%%%%%%%%%%%%%%%%%%%%%%%
\subsection{ErrorCatchers}
\label{libraries.ErrorCatchers}

Module \code{Cheetah.ErrorCatchers} contains error-handling classes
suitable for the \code{\#errorCatcher} directive.
See section \ref{errorHandling.errorCatcher} for a description of the 
error catchers bundled with Cheetah.


%%%%%%%%%%%%%%%%%%%%%%%%%%%%%%%%%%%%%%%%%%%%%%%%%%%%%%%%%%%%%%%%%%%%%%%%%%%%%%%%
\subsection{FileUtils}
\label{libraries.FileUtils}

Module \code{Cheetah.FileUtils} contains generic functions and classes for 
doing bulk search-and-replace on several files, and for finding all the files
in a directory hierarchy whose names match a glob pattern.

%%%%%%%%%%%%%%%%%%%%%%%%%%%%%%%%%%%%%%%%%%%%%%%%%%%%%%%%%%%%%%%%%%%%%%%%%%%%%%%%
\subsection{Filters}
\label{libraries.Filters}

Module \code{Filters} contains filters suitable for the \code{\#Filter}
directive.  See section \ref{output.filter} for a description of the
filters bundled with Cheetah.

As of Cheetah 0.9.9a6, the set of filters is rapidly changing.  While this is
mostly a case of more filters and more features, it is possible that some 
filers may be removed in the future or be changed in backward-incompatible ways.
If you are depending on a filter besides the default filter, and you can not
risk it changing in the next version of Cheetah, make yourself a separate copy
of it outside the Cheetah directory and use that.

%%%%%%%%%%%%%%%%%%%%%%%%%%%%%%%%%%%%%%%%%%%%%%%%%%%%%%%%%%%%%%%%%%%%%%%%%%%%%%%%
\subsection{SettingsManager}
\label{libraries.SettingsManager}

The \code{SettingsManager} class in the \code{Cheetah.SettingsManager} module is
a mixin class that provides facilities for managing application settings.  
SettingsManager is designed to:
\begin{itemize}
\item work well with nested settings dictionaries of any depth
\item read/write \code{.ini style config files} (or strings)
\item read settings from Python source files (or strings) so that
     complex Python objects can be stored in the application's settings
     dictionary.  For example, you might want to store references to various
     classes that are used by the application, and plugins to the application
     might want to substitute one class for another.
\item allow sections in \code{.ini config files} to be extended by settings in
     Python src files.  If a section contains a setting like
     ``\code{importSettings=mySettings.py}'', \code{SettingsManager} will merge
     all the settings defined in ``\code{mySettings.py}'' with the settings for
     that section that are defined in the \code{.ini config file}.
\item maintain the case of setting names, unlike the ConfigParser module
\end{itemize}

Cheetah uses \code{SettingsManager} to manage its configuration settings.
\code{SettingsManager} might also be useful in your own applications. See the
source code and docstrings in the file \code{src/SettingsManager.py} for more
information. If there is sufficient interest in \code{SettingsManager}, we will
release it as a standalone module.

Note that there are two Cheetah directives that deal with settings:
\code{\#compiler-settings} affects the behaviour of Cheetah's parser and 
compiler.  \code{\#settings} affects the behaviour of certain filters and
error catchers (see \code{\#filter} and \code{\#errorCatcher}, and may also be
used in your own template applications and and templates.

By default, both kinds of settings are also visible in the searchList, so they
may be referred to by \code{\$placeholders}.  They are at the very end of the
searchList, so any same-name anywhere else in the searchList will override
what the \code{\$placeholder} sees.

%% @@MSO: Alternate wording:
%% Class \code{Cheetah.SettingsManager.SettingsManager} facilitates the use of
%% user-supplied configuration files to fine tune an application.  A setting is
%% a key/value pair that an application or component (e.g., a filter, or your 
%% own servlets) looks up and treats as a configuration value to modify its (the
%% component's) behaviour.  For more information, see section
%% \ref{inheritanceEtc.settings} and the class source.


%%%%%%%%%%%%%%%%%%%%%%%%%%%%%%%%%%%%%%%%%%%%%%%%%%%%%%%%%%%%%%%%%%%%%%%%%%%%%%%%
\subsection{Templates}
\label{libraries.templates}

Package \code{Cheetah.Templates} contains stock templates that you can
either use as is, or extend by using the \code{\#def} directive to redefine
specific {\bf blocks}.  Currently, the only template in here is SkeletonPage,
which is described in detail below.

%%%%%%%%%%%%%%%%%%%%%%%%%%%%%%%%%%%%%%%%%%%%%%%%%%%%%%%%%%%%%%%%%%%%%%%%%%%%%%%%
\subsection{Tools}
\label{libraries.Tools}

Package \code{Cheetah.Tools} contains functions and classes contributed by third
parties.  Some are Cheetah-specific but others are generic and can be used
standalone.  None of them are imported by any other Cheetah component; you can 
delete the Tools/ directory and Cheetah will function normally.  

Some of the items in Tools/ are experimental and have been placed there just to
see how useful they will be, and whether they attract enough users to make 
refining them worthwhile (the tools, not the users :).

Nothing in Tools/ is guaranteed to be: (A) tested, (B) reliable, (C) immune from
being deleted in a future Cheetah version, or (D) immune from having 
backwards-incompatable changes made to it.  If you depend on something in Tools/
on a production system, consider making a copy of it outside the Cheetah/
directory so that this version won't be lost when you upgrade Cheetah.  Also,
learn enough about Python and about the Tool so that you can maintain it and
bugfix it if necessary.

If anything in Tools/ is found to be necessary to Cheetah's operation (i.e., if
another Cheetah component starts importing it), it will be moved to the
\code{Cheetah.Utils} package.

Current Tools include:
\begin{itemize}
\item {\bf Cheetah.Tools.MondoReport} -- an ambitious class useful when iterating
     over records of data (\code{\#for} loops), displaying one pageful of records
     at a time (with previous/next links), and printing summary statistics about
     the records or the current page.  See \code{MondoReportDoc.txt} in the same
     directory as the module.  Some features are not implemented yet.  Contributed
     by Mike Orr.
\item {\bf Cheetah.Tools.RecursiveNull} -- Nothing, but in a friendly way.  Good
     for filling in for objects you want to hide.  If \code{\$form.f1} is a
     RecursiveNull object, then \code{\$form.f1.anything["you"].might("use")}
     will resolve to the empty string.  You can also put a \code{RecursiveNull}
     instance at the end of the searchList to convert missing values to ''
     rather than raising a \code{NotFound} error or having a (less efficient)
     errorCatcher handle it.  Of course, maybe you prefer to get a \code{NotFound}
     error...  Contributed by Ian Bicking.
\item {\bf Cheetah.Tools.SiteHierarchy} -- Provides navigational links to this
     page's parents and children.  The constructor takes a recursive list of
     (url,description) pairs representing a tree of hyperlinks to every page in the
     site (or section, or application...), and also a string containing the current
     URL.  Two methods 'menuList' and 'crumbs' return output-ready HTML showing an
     indented menu (hierarchy tree) or crumbs list (Yahoo-style bar:
     home > grandparent > parent > currentURL).  Contributed by Ian Bicking.
\item{\bf Cheetah.Tools.WebwareMixin} -- Additional methods for your Template object.
     Currently there is one method, \code{.cgiImport}.  It imports the specified CGI
     fields, cookies or session variables into a new object on the searchList, so that
     they can be accessed by name in \$placeholders (e.g., \code{\$myField}).  This
     method may be renamed in the next version of Cheetah, and will likely be split 
     into two methods to handle single vs multiple values for a field.  Contributed
     by Mike Orr.
\end{itemize}


%%%%%%%%%%%%%%%%%%%%%%%%%%%%%%%%%%%%%%%%%%%%%%%%%%%%%%%%%%%%%%%%%%%%%%%%%%%%%%%%
\subsection{Utils}
\label{libraries.Utils}

Package \code{Cheetah.Utils} contains non-Cheetah-specific functions and
classes that are imported by other Cheetah components.  Many of these utils can
be used standalone in other applications too.  

Current Utils include:
\begin{itemize}
\item {\bf Cheetah.Utils.VerifyType} -- Functions to verify the type of a
     user-supplied function argument.
\end{itemize}

%%%%%%%%%%%%%%%%%%%%%%%%%%%%%%%%%%%%%%%%%%%%%%%%%%%%%%%%%%%%%%%%%%%%%%%%%%%%%%%%
\subsubsection{Cheetah.Templates.SkeletonPage}
\label{libraries.templates.skeletonPage}

A stock template class that will be very useful for web developers is defined in
the \code{Cheetah.Templates.SkeletonPage} module.  The \code{SkeletonPage}
template class is generated from the following Cheetha source code:

\begin{verbatim}
$*docType
<HTML>
####################
#block headerComment
<!-- This document was autogenerated by Cheetah. Don't edit it directly!

Copyright $currentYr() - $*siteCopyrightName - All Rights Reserved.
Feel free to copy any javascript or html you like on this site,
provided you remove all links and/or references to $*siteDomainName
However, please do not copy any content or images without permission.

$*siteCredits

-->

#end block 
#####################

#################
#block headTag
<HEAD>
<TITLE>$*title</TITLE>
$*metaTags()
$*stylesheetTags()
$*javascriptTags()
</HEAD>
#end block 
#################


#################
#block bodyTag
$bodyTag()
#end block 
#################

#block writeBody
This skeleton page has no flesh. Its body needs to be implemented.
#end block 

</BODY>
</HTML>
\end{verbatim}

You can redefine any of the blocks defined in this template by writing a new
template that \code{\#extends} SkeletonPage.  (As you remember, using 
\code{\#extends} makes your template implement the \code{.writeBody()}
method instead of \code{.respond()} -- which happens to be the same method
SkeletonPage expects the page content to be (note the writeBody block in
SkeletonPage).)

\begin{verbatim}
#def bodyContents
Here's my new body. I've got some flesh on my bones now.
#end def bodyContents
\end{verbatim}

%% @@MO: Is this still accurate?  Does the child template really need to put a
%% #def around its whole content?  Or by implementing .writeBody() does it
%% automatically insert itself as the writeBody portion of
%% SkeletonPage?

All of the \$placeholders used in the \code{SkeletonPage} template definition
are attributes or methods of the \code{SkeletonPage} class.  You can reimplement
them as you wish in your subclass.  Please read the source code of the file
\code{src/Templates/\_SkeletonPage.py} before doing so.  

You'll need to understand how to use the following methods of the
\code{SkeletonPage} class: \code{\$metaTags()}, \code{\$stylesheetTags()},
\code{\$javascriptTags()}, and \code{\$bodyTag()}.  They take the data you
define in various attributes and renders them into HTML tags.

\begin{itemize}
\item {\bf metaTags()} -- Returns a formatted vesion of the self.\_metaTags
     dictionary, using the formatMetaTags function from
     \code{Cheetah.Macros.HTML}.
\item {\bf stylesheetTags()} -- Returns a formatted version of the
     \code{self.\_stylesheetLibs} and \code{self.\_stylesheets} dictionaries.
     The keys in \code{self.\_stylesheets} must be listed in the order that
     they should appear in the list \code{self.\_stylesheetsOrder}, to ensure
     that the style rules are defined in the correct order.
\item {\bf javascriptTags()} -- Returns a formatted version of the
     \code{self.\_javascriptTags} and \code{self.\_javascriptLibs} dictionaries.
     Each value in \code{self.\_javascriptTags} should be a either a code string
     to include, or a list containing the JavaScript version number and the code
     string. The keys can be anything.  The same applies for
     \code{self.\_javascriptLibs}, but the string should be the SRC filename
     rather than a code string.
\item {\bf bodyTag()} -- Returns an HTML body tag from the entries in the dict
     \code{self.\_bodyTagAttribs}.
\end{itemize}

%% @@MO: (item 1 above) Macros don't exist any more.  What should this say?

The class also provides some convenience methods that can be used as
\$placeholders in your template definitions:

\begin{itemize}
\item {\bf imgTag(self, src, alt='', width=None, height=None, border=0)} --
     Dynamically generate an image tag.  Cheetah will try to convert the
     ``\code{src}'' argument to a WebKit serverSidePath relative to the
     servlet's location. If width and height aren't specified they are
     calculated using PIL or ImageMagick if either of these tools are available.
     If all your images are stored in a certain directory you can reimplement
     this method to append that directory's path to the ``\code{src}'' argument.
     Doing so would also insulate your template definitions from changes in your
     directory structure.
\end{itemize}

See the file \code{examples/webware\_examples/cheetahSite/SiteTemplate.tmpl}
for an extended example of how \code{SkeletonPage} can be used.



% Local Variables:
% TeX-master: "users_guide"
% End:      
