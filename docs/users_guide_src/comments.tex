%%%%%%%%%%%%%%%%%%%%%%%%%%%%%%%%%%%%%%%%%%%%%%%%%%%%%%%%%%%%%%%%%%%%%%%%%%%%%%%%
\section{Comments}
\label{comments}

Comments are used to mark notes, explanations, and decorative text that should
not appear in the output.  Cheetah maintains the comments in the Python module
it generates from the Cheetah source code. There are two forms of the comment
directive: single-line and multi-line.

All text in a Template Definition that lies between 2 hash characters
(\code{\#\#}) and the end of the line is treated as a single-line comment and
will not show up in the output, unless the 2 hash characters are escaped with a
backslash.
\begin{verbatim}
##=============================  this is a decorative comment-bar
$var    ## this is an end-of-line comment
##=============================
\end{verbatim}

Any text between \code{\#*} and \code{*\#} will be treated as a multi-line
comment.
\begin{verbatim}
#*
   Here is some multiline
   comment text
*#
\end{verbatim}


\subsection{Docstring Comments}
\label{directives.comments.docstring}
Python modules, classes, and methods can be documented with inline
'documentation strings' (aka 'docstrings').  Docstrings, unlike comments, are
accesible at run-time. Thus, they provide a useful hook for interactive help
utilities.

Cheetah comments can be transformed into doctrings by adding one of the
following prefixes:

\begin{verbatim}
##doc: This text will be added to the method docstring
#*doc: This text will be added to the method docstring *#

##doc-method: so will this
#*doc-method: so will this *#

##doc-class: This text will be added to the class docstring
#*doc-class: This text will be added to the class docstring *#

##doc-module: This text will be added to the module docstring
#*doc-module: This text will be added to the module docstring *#
\end{verbatim}

\subsection{Header Comments}
\label{directives.comments.headers}
Cheetah comments can also be transformed into module header comments using the
following syntax:

\begin{verbatim}
##header: This text will be added to the module header comment
#*header: This text will be added to the module header comment *#
\end{verbatim}

% Local Variables:
% TeX-master: "users_guide"
% End:      
