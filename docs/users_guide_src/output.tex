\section{Generating, Caching and Filtering Output}
\label{output}

%%%%%%%%%%%%%%%%%%%%%%%%%%%%%%%%%%%%%%%%%%%%%%%%%%%%%%%%%%%%%%%%%%%%%%%%%%%%%%%%
\subsection{Output from complex expressions: \#echo}
\label{output.echo}

Syntax:
\begin{verbatim}
#echo EXPR
\end{verbatim}

The \code{\#echo} directive is used to echo the output from expressions that
can't be written as simple \$placeholders.  

\begin{verbatim}
Here is my #echo ', '.join(['silly']*5) # example 
\end{verbatim}

This produces:

\begin{verbatim}
Here is my silly, silly, silly, silly, silly example.
\end{verbatim}


%%%%%%%%%%%%%%%%%%%%%%%%%%%%%%%%%%%%%%%%%%%%%%%%%%%%%%%%%%%%%%%%%%%%%%%%%%%%%%%%
\subsection{Executing expressions without output: \#silent}
\label{output.silent}

Syntax:
\begin{verbatim}
#silent EXPR
\end{verbatim}

\code{\#silent} is the opposite of \code{\#echo}.  It executes an expression
but discards the output.

\begin{verbatim}
#silent $myList.reverse()
#silent $myList.sort()
Here is #silent $covertOperation() # nothing
\end{verbatim}

If your template requires some Python code to be executed at the beginning;
(e.g., to calculate placeholder values, access a database, etc), you can put
it in a "doEverything" method you inherit, and call this method using
\code{\#silent} at the top of the template.

%%%%%%%%%%%%%%%%%%%%%%%%%%%%%%%%%%%%%%%%%%%%%%%%%%%%%%%%%%%%%%%%%%%%%%%%%%%%%%%%
\subsection{One-line \#if}
\label{output.oneLineIf}

Syntax:
\begin{verbatim}
#if EXPR1 then EXPR2 else EXPR3#
\end{verbatim}

The \code{\#if} flow-control directive (section \ref{flowControl.if}) has a
one-line counterpart akin to Perl's and C's \code{?:} operator.
If \code{EXPR1} is true, it evaluates \code{EXPR2} and outputs the result (just
like \code{\#echo\ EXPR2\#}).  Otherwise it evaluates \code{EXPR3} and outputs
that result.  This directive is short-circuiting, meaning the expression that
isn't needed isn't evaluated.

You MUST include both 'then' and 'else'.  If this doesn't work for you or you
don't like the style use multi-line \code{\#if} directives (section
\ref{flowControl.if}).

The trailing \code{\#} is the normal end-of-directive character.  As usual
it may be omitted if there's nothing after the directive on the same line.


%%%%%%%%%%%%%%%%%%%%%%%%%%%%%%%%%%%%%%%%%%%%%%%%%%%%%%%%%%%%%%%%%%%%%%%%%%%%%%%%
\subsection{Caching Output}
\label{output.caching}

%%%%%%%%%%%%%%%%%%%%%%%%%%%%%%%%%%%%%%%%%%%%%%%%%%%%%%%%%%%%%%%%%%%%%%%%%%%%%%%%
\subsubsection{Caching individual placeholders}
\label{output.caching.placeholders}

By default, the values of each \$placeholder is retrieved and
interpolated for every request. However, it's possible to cache the values
of individual placeholders if they don't change very often, in order to 
speed up the template filling.
                         
To cache the value of a single \code{\$placeholder}, add an asterisk after the
\$; e.g.,  \code{\$*var}.  The first time the template is
filled, \code{\$var} is looked up.  Then whenever the template is filled again,
the cached value is used instead of doing another lookup.  

The \code{\$*} format caches ``forever''; that is, as long as the template
instance remains in memory.  It's also possible to cache for a certain time
period using the form \code{\$*<interval>*variable}, where \code{<interval>} is
the interval.  The time interval can be specified in seconds (5s), minutes
(15m), hours (3h), days (2d) or weeks (1.5w). The default is minutes.

\begin{verbatim}
<HTML>
<HEAD><TITLE>$title</TITLE></HEAD>
<BODY>

$var ${var}           ## dynamic - will be reinterpolated for each request
$*var2 $*{var2}       ## static - will be interpolated only once at start-up
$*5*var3 $*5*{var3}   ## timed refresh - will be updated every five minutes.

</BODY>
</HTML>
\end{verbatim}

Note that ``every five minutes'' in the example really means every five
minutes: the variable is looked up again when the time limit is reached,
whether the template is being filled that frequently or not.  Keep this in
mind when setting refresh times for CPU-intensive or I/O intensive 
operations.

If you're using the long placeholder syntax, \verb+${}+, the braces go only
around the placeholder name: \verb+$*.5h*{var.func('arg')}+.

Sometimes it's preferable to explicitly invalidate a cached item whenever
you say so rather than at certain time intervals.  You can't do this with
individual placeholders, but you can do it with cached regions, which will
be described next.

%%%%%%%%%%%%%%%%%%%%%%%%%%%%%%%%%%%%%%%%%%%%%%%%%%%%%%%%%%%%%%%%%%%%%%%%%%%%%%%%
\subsubsection{Caching entire regions}
\label{output.caching.regions}

Syntax:
\begin{verbatim}
#cache [id=EXPR] [timer=EXPR] [test=EXPR]
#end cache
\end{verbatim}

The \code{\#cache} directive is used to cache a region of
content in a template.  The region is cached as a single unit, after 
placeholders and directives inside the region have been evaluated.  If there
are any \code{\$*<interval>*var} placholders inside the cache
region, they are refreshed only when {\em both} the cache region {\em and} the 
placeholder are simultaneously due for a refresh.

Caching regions offers more flexibility than caching individual placeholders.
You can specify the refresh interval using a placeholder or
expression, or refresh according to other criteria rather than a certain
time interval.

\code{\#cache} without arguments caches the region statically, the same
way as \code{\$*var}.  The region will not be automatically refreshed.

To refresh the region at an interval, use the \code{timer=EXPRESSION} argument,
equivalent to \code{\$*<interval>*}.  The expression should evaluate to a
number or string that is a valid interval (e.g., 0.5, '3m', etc).

To refresh whenever an expression is true, use \code{test=EXPRESSION}.
The expression can be a method/function returning true or false, a boolean
placeholder, several of these joined by \code{and} and/or \code{or}, or any
other expression.  If the expression contains spaces, it's easier to
read if you enclose it in \code{()}, but this is not required.

To refresh whenever you say so, use \code{id=EXPRESSION}.  Your program can
then call \code{.refreshCache(ID)} whenever it wishes.  This is useful if the
cache depends on some external condition that changes infrequently but has just
changed now.

You can combine arguments by separating them with commas.  For instance, you can
specify both \code{id=} and \code{interval=}, or \code{id=} and \code{test=}.
(You can also combine interval and test although it's not very useful.)
However, repeating an argument is undefined.

\begin{verbatim}
#cache
This is a static cache.  It will not be refreshed.
$a $b $c
#end cache

#cache timer='30m', id='cache1'
#for $cust in $customers
$cust.name:
$cust.street - $cust.city
#end for
#end cache

#cache id='sidebar', test=$isDBUpdated
... left sidebar HTML ...
#end cache

#cache id='sidebar2', test=($isDBUpdated or $someOtherCondition)
... right sidebar HTML ...
#end cache
\end{verbatim}


The \code{\#cache} directive cannot be nested.

We are planning to add a \code{'varyBy'} keyword argument in the future that
will allow a separate cache instances to be created for a variety of conditions,
such as different query string parameters or browser types. This is inspired by
ASP.net's varyByParam and varyByBrowser output caching keywords.

% @@MO: Can we cache by Webware sessions?  What about sessions where the
% session ID is encoded as a path prefix in the URI?  Need examples.


%%%%%%%%%%%%%%%%%%%%%%%%%%%%%%%%%%%%%%%%%%%%%%%%%%%%%%%%%%%%%%%%%%%%%%%%%%%%%%%%
\subsection{\#raw}
\label{output.raw}

Syntax:
\begin{verbatim}
#raw
#end raw
\end{verbatim}

Any section of a template definition that is inside a \code{\#raw \ldots
\#end raw} tag pair will be printed verbatim without any parsing of
\$placeholders or other directives. This can be very useful for debugging, or
for Cheetah examples and tutorials.

\code{\#raw} is conceptually similar to HTML's \code{<PRE>} tag and LaTeX's
\code{\\verbatim\{\}} tag, but unlike those tags, \code{\#raw} does not cause
the body to appear in a special font or typeface.  It can't, because Cheetah
doesn't know what a font is.  


%%%%%%%%%%%%%%%%%%%%%%%%%%%%%%%%%%%%%%%%%%%%%%%%%%%%%%%%%%%%%%%%%%%%%%%%%%%%%%%%
\subsection{\#include}
\label{output.include}

Syntax:
\begin{verbatim}
#include [raw] FILENAME_EXPR
#include [raw] source=STRING_EXPR
\end{verbatim}

The \code{\#include} directive is  used to include text from outside the
template definition.  The text can come from an external file or from a
\code{\$placeholder} variable.  When working with external files, Cheetah will
monitor for changes to the included file and update as necessary.  

This example demonstrates its use with external files:
\begin{verbatim}
#include "includeFileName.txt"
\end{verbatim}
The content of "includeFileName.txt" will be parsed for Cheetah syntax.

And this example demonstrates use with \code{\$placeholder} variables:
\begin{verbatim}
#include source=$myParseText
\end{verbatim}
The value of \code{\$myParseText} will be parsed for Cheetah syntax. This is not
the same as simply placing the \$placeholder tag ``\code{\$myParseText}'' in
the template definition.  In the latter case, the value of \$myParseText would
not be parsed.

By default, included text will be parsed for Cheetah tags.  The argument
``\code{raw}'' can be used to suppress the parsing.

\begin{verbatim}
#include raw "includeFileName.txt"
#include raw source=$myParseText
\end{verbatim}

Cheetah wraps each chunk of \code{\#include} text inside a nested
\code{Template} object.  Each nested template has a copy of the main
template's searchList.  However, \code{\#set} variables are visible
across includes only if the defined using the \code{\#set global} keyword.

All directives must be balanced in the include file.  That is, if you start
a \code{\#for} or \code{\#if} block inside the include, you must end it in
the same include.  (This is unlike PHP, which allows unbalanced constructs
in include files.)

% @@MO: What did we decide about #include and the searchList?  Does it really
% use a copy of the searchList, or does it share the searchList with the
% parent?

% @@MO: deleted
%These nested templates share the same \code{searchList}
%as the top-level template. 

%%%%%%%%%%%%%%%%%%%%%%%%%%%%%%%%%%%%%%%%%%%%%%%%%%%%%%%%%%%%%%%%%%%%%%%%%%%%%%%%
\subsection{\#slurp}
\label{output.slurp}

Syntax:
\begin{verbatim}
#slurp
\end{verbatim}

The \code{\#slurp} directive eats up the trailing newline on the line it
appears in, joining the following line onto the current line.


It is particularly useful in \code{\#for} loops:
\begin{verbatim}
#for $i in range(5)
$i #slurp
#end for
\end{verbatim}
outputs:
\begin{verbatim}
0 1 2 3 4
\end{verbatim}


%%%%%%%%%%%%%%%%%%%%%%%%%%%%%%%%%%%%%%%%%%%%%%%%%%%%%%%%%%%%%%%%%%%%%%%%%%%%%%%%
\subsection{\#indent}
\label{output.indent}

This directive is not implemented yet.  When/if it's completed, it will allow
you to 
\begin{enumerate}
\item indent your template definition in a natural way (e.g., the bodies
    of \code{\#if} blocks) without affecting the output
\item add indentation to output lines without encoding it literally in the
    template definition.  This will make it easier to use Cheetah to produce
    indented source code programmatically (e.g., Java or Python source code).  
\end{enumerate}

There is some experimental code that recognizes the \code{\#indent}
directive with options, but the options are purposely undocumented at this
time.  So pretend it doesn't exist.  If you have a use for this feature
and would like to see it implemented sooner rather than later, let us know
on the mailing list.

The latest specification for the future \code{\#indent} directive is in the
TODO file in the Cheetah source distribution.

% @@MO: disabled because it's not implemented and the spec is changing
% \code{\#indent} decouples the indentation in the template definition from the
% indentation in the output.  Normally, Cheetah outputs indentation exactly as
% it sees it, no matter whether the indentation is on the first line of a 
% paragraph, in front of a directive, or wherever.  \code{\#indent} has two main
% uses:
% \begin{enumerate}
% \item To strip all indentation from source lines.  This lets you indent
%     multiline directives (e.g., \code{\#if}, \code{\#for}) in a natural way
%     without having that indentation appear in the output.
% \item To indent every text line in the output according to a user-specified
%     ``indentation level'', independent of whatever indentation the source lines
%     may have.  This is useful for producing Python output, or any language that
%     requires strict indentation levels at certain places.  To accomplish this,
%     Cheetah adds a call to an indentation method at the beginning of every
%     affected source line.
% \end{enumerate}
% 
% To accomplish the first part, Cheetah removes leading whitespace from the
% affected source lines before the compiler see them.  To accomplish the second
% part, Cheetah keeps track of the current indentation level, a value you have
% full control over.  At the beginning of every affected text line, Cheetah calls
% a method that outputs the appropriate indentation string.  This affects only
% lines in the template definition itself, not multiline placeholder values.  
% See the \code{Indent} filter below to indent multiline placeholder values.
% 
% All \code{\#indent} commands operate on the lines physically below them in
% the template definition until the next \code{\#indent}, regardless of scope.
% This means they work thorugh all other directives (\code{\#def}, \code{\#for},
% \code{\#if}, etc) -- so that if you turn on indentation inside a \code{\#def},
% it remains in effect past the \code{\#end def}.
% 
% The following commands turn indentation on and off:
% \begin{description}
% \item{\code{\#indent on}}  Strip leading whitespace and add indentation to the
%     following lines.  This fulfills use \#2 above.
% \item{\code{\#indent off}} Do not strip leading whitespace or add indentation.
%     This is Cheetah's default behavior.
% \item{\code{\#indent strip}}  Strip leading whitespace but do {\em not} add
%     indentation.  This fulfills use \#1 above.
% \end{description}
% 
% Indentation by default uses real tabs.  But you can change the indentation
% string thus:
% \begin{verbatim}
% ## Output four spaces for each indentation level.
% #indent chars '    '
% ## Output the mail reply prefix for each indentation level.
% #indent chars '> '
% ## Use a placeholder.
% #indent chars $indentChars
% ## Return to the default behavior.
% #indent chars '\t'
% \end{verbatim}
% 
% 
% The following commands change the indentation level, which is a non-negative
% integer initially at zero.  All of these commands implicitly do an 
% \code{\#indent on}:
% \begin{description}
% \item{\code{\#indent ++}}  Increment the current indentation level.
% \item{\code{\#indent --}}  Decrement the current indentation level.
% \item{\code{\#indent +3}}  Add three indentation levels (or any number).
% \item{\code{\#indent -3}}  Subtract three indentation levels (or any number).
% \item{\code{\#indent =3}}   Set the indentation level to 3.
% \item{\code{\#indent push +2}}  Save the current indentation level on a stack
%     and add two.  
% \item{\code{\#indent pop}}  Return to the most recently pushed level.  Raise
%     \code{IndentationStackEmptyError} if there is no previous level.
% \end{description}
% 
% The expressions after \code{+}/\code{-}/\code{=} may be numeric literals or
% Cheetah expressions.  The effect is undefined if the value is negative.  There
% may be whitespace after the \code{+}/\code{-}/\code{=} symbol.
% The initial implementation uses a simple preprocessor that doesn't understand
% newline characters in expressions.  \code{\\n} is fine, but not a real newline.
% 
% To indent multiline placeholder values using the current indentation level,
% use the \code{Indent} filter:
% \begin{verbatim}
% #filter Indent
% \end{verbatim}
% It works like the default filter but adds indentation after every newline.  It
% does not strip any leading whitespace.  It hooks into \code{\$self.\_indenter},
% defined in \code{Cheetah.Utils.Indenter}.  This object keeps track of the
% current indentation level.  Specifically, the filter calls
% \code{\$self.\_indent()}, which is a shortcut to the indenter's
% \code{.indent()} method.  This is the same thing \code{\#indent} does.
% However, the filter is usable even when indentation is in
% \code{off} or \code{strip} mode.


%%%%%%%%%%%%%%%%%%%%%%%%%%%%%%%%%%%%%%%%%%%%%%%%%%%%%%%%%%%%%%%%%%%%%%%%%%%%%%%%
\subsection{Ouput Filtering and \#filter}
\label{output.filter}

Syntax:
\begin{verbatim}
#filter FILTER_CLASS_NAME
#filter $PLACEHOLDER_TO_A_FILTER_INSTANCE
#filter None
\end{verbatim}


Output from \$placeholders is passed through an ouput filter.  The default
filter merely returns a string representation of the placeholder value,
unless the value is \code{None}, in which case the filter returns an empty
string.  Only top-level placeholders invoke the filter; placeholders inside
expressions do not.

Certain filters take optional arguments to modify their behaviour.  To pass
arguments, use the long placeholder syntax and precede each filter argument by
a comma.  By convention, filter arguments don't take a \code{\$} prefix, to
avoid clutter in the placeholder tag which already has plenty of dollar signs.
For instance, the MaxLen filter takes an argument 'maxlen':

\begin{verbatim}
${placeholderName, maxlen=20}
${functionCall($functionArg), maxlen=$myMaxLen}
\end{verbatim}

To change the output filter, use the \code{'filter'} keyword to the
\code{Template} class constructor, or the \code{\#filter}
directive at runtime (details below).  You may use \code{\#filter} as often as
you wish to switch between several filters, if certain \code{\$placeholders}
need one filter and other \code{\$placeholders} need another.

The standard filters are in the module \code{Cheetah.Filters}.  Cheetah
currently provides:

\begin{description}
\item{\code{Filter}}
     \\ The default filter, which converts None to '' and everything else to
     \code{str(whateverItIs)}.  This is the base class for all other filters,
     and the minimum behaviour for all filters distributed with Cheetah.
\item{\code{ReplaceNone}}
     \\ Same.
\item{\code{MaxLen}}
     \\ Same, but truncate the value if it's longer than a certain length.
     Use the 'maxlen' filter argument to specify the length, as in the
     examples above.  If you don't specify 'maxlen', the value will not be
     truncated.
\item{\code{Pager}}
     \\ Output a "pageful" of a long string.  After the page, output HTML
     hyperlinks to the previous and next pages.  This filter uses several
     filter arguments and environmental variables, which have not been 
     documented yet.
\item{\code{WebSafe}}
     \\ Same as default, but convert HTML-sensitive characters ('$<$', '\&',
     '$>$')
     to HTML entities so that the browser will display them literally rather
     than interpreting them as HTML tags.  This is useful with database values
     or user input that may contain sensitive characters.  But if your values
     contain embedded HTML tags you want to preserve, you do not want this 
     filter.
     
     The filter argument 'also' may be used to specify additional characters to
     escape.  For instance, say you want to ensure a value displays all on one
     line.  Escape all spaces in the value with '\&nbsp', the non-breaking
     space:
\begin{verbatim}
${$country, also=' '}}
\end{verbatim}
\end{description}

To switch filters using a class object, pass the class using the
{\bf filter} argument to the Template constructor, or via a placeholder to the
\code{\#filter} directive: \code{\#filter \$myFilterClass}.  The class must be
a subclass of \code{Cheetah.Filters.Filter}.  When passing a class object, the
value of {\bf filtersLib} does not matter, and it does not matter where the
class was defined.

To switch filters by name, pass the name of the class as a string using the
{\bf filter} argument to the Template constructor, or as a bare word (without
quotes) to the \code{\#filter} directive: \code{\#filter TheFilter}.  The
class will be looked up in the {\bf filtersLib}.

The filtersLib is a module containing filter classes, by default
\code{Cheetah.Filters}.  All classes in the module that are subclasses of
\code{Cheetah.Filters.Filter} are considered filters.  If your filters are in
another module, pass the module object as the {\bf filtersLib} argument to the
Template constructor.  

Writing a custom filter is easy: just override the \code{.filter} method.
\begin{verbatim}
    def filter(self, val, **kw):     # Returns a string.
\end{verbatim}
Return the {\em string} that should be output for `val'.  `val' may be any
type.  Most filters return `' for \code{None}.  Cheetah passes one keyword
argument: \verb+kw['rawExpr']+ is the placeholder name as it appears in
the template definition, including all subscripts and arguments.  If you use
the long placeholder syntax, any options you pass appear as keyword
arguments.  Again, the return value must be a string.

You can always switch back to the default filter this way:
\code{\#filter None}.  This is easy to remember because "no filter" means the
default filter, and because None happens to be the only object the default
filter treats specially.

We are considering additional filters; see
\url{http://webware.colorstudy.net/twiki/bin/view/Cheetah/MoreFilters}
for the latest ideas.

%% @@MO: Is '#end filter' implemented?  Will it be?  Can filters nest?
%% Will '#end filter' and '#filter None' be equivalent?

%% @@MO: Tavis TODO: fix the description of the Pager filter.  It needs a howto.

%% @@MO: How about using settings to provide default arguments for filters?
%% Each filter could look up FilterName (or FilterNameDefaults) setting,
%% whose value would be a dictionary containing keyword/value pairs.  These
%% would be overridden by same-name keys passed by the placeholder.

%% @@MO: If sed-filters (#sed) get added to Cheetah, give them a section here.

% Local Variables:
% TeX-master: "users_guide"
% End:      

% vim: shiftwidth=4 tabstop=4 expandtab
