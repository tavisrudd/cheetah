\section{Visual Editors}
\label{visualEditors}

This chapter is about maintaining Cheetah templates with visual editors,
and the tradeoffs between making it friendly to both text editors and visual
editors.

Cheetah's main developers do not use visual editors.  Tavis uses \code{emacs};
Mike uses \code{vim}.  So our first priority is to make templates easy to
maintain in text editors.  In particular, we don't want to add features
like Zope Page Template's 
placeholder-value-with-mock-text-for-visual-editors-all-in-an-XML-tag.
The syntax is so verbose it makes for a whole lotta typing just to insert a
simple placeholder, for the benefit of editors we never use.  However, as users
identify features which would help their visual editing without making it
harder to maintain templates in a text editor, we're all for it.  

As it said in the introduction, Cheetah purposely does not use HTML/XML
tags for \$placeholders or \#directives.  That way, when you preview the
template in an editor that interprets HTML tags, you'll still see the
placeholder and directive source definitions, which provides some ``mock text''
even if it's not the size the final values will be, and allows you to use
your imagination to translate how the directive output will look visually in
the final.

If your editor has syntax highlighting, turn it on.  That makes a big 
difference in terms of making the template easier to edit.  Since no
``Cheetah mode'' has been invented yet, set your highlighting to Perl
mode, and at least the directives/placeholders will show up in different
colors, although the editor won't reliably guess where the 
directive/placeholder ends and normal text begins.

% Local Variables:
% TeX-master: "users_guide"
% End:      




