\section{Cheetah vs. Other Template Engines}
\label{comparisons}

This appendix compares Cheetah with various other template/emdedded scripting
languages and Internet development frameworks.  As Cheetah is similar to
Velocity at a superficial level, you may also wish to read comparisons between
Velocity and other languages at
\url{http://jakarta.apache.org/velocity/ymtd/ymtd.html}.

%%%%%%%%%%%%%%%%%%%%%%%%%%%%%%%%%%%%%%%%%%%%%%%%%%%%%%%%%%%%%%%%%%%%%%%%%%%%%%%%
\subsection{Which features are unique to Cheetah}
\label{comparisons.unique}

\begin{itemize}
\item The {\bf block framework} (section \ref{inheritanceEtc.block})
\item Cheetah's powerful yet simple {\bf caching framework} (section
     \ref{output.caching})
\item Cheetah's {\bf Unified Dotted Notation} and {\bf autocalling}
     (sections \ref{language.namemapper.dict} and
     \ref{language.namemapper.autocalling})
\item Cheetah's searchList (section \ref{language.searchList})
     information.
\item Cheetah's \code{\#raw} directive (section \ref{output.raw})
\item Cheetah's \code{\#slurp} directive (section \ref{output.slurp})
\item Cheetah's tight integration with Webware for Python (section
     \ref{webware})
\item Cheetah's {\bf SkeletonPage framework} (section
     \ref{libraries.templates.skeletonPage})
\item Cheetah's ability to mix PSP-style code with Cheetah
     Language syntax (section \ref{tips.ASP}) 
     Because of Cheetah's design and Python's flexibility it is
     relatively easy to extend Cheetah's syntax with syntax elements from almost
     any other template or embedded scripting language.
\end{itemize}

%% @@MO: Is #extend unique too?

%% @@MO: What about the new features we've been adding?

%%%%%%%%%%%%%%%%%%%%%%%%%%%%%%%%%%%%%%%%%%%%%%%%%%%%%%%%%%%%%%%%%%%%%%%%%%%%%%%%
\subsection{Cheetah vs. Velocity}
\label{comparisons.velocity}

For a basic introduction to Velocity, visit
\url{http://jakarta.apache.org/velocity}.

Velocity is a Java template engine.  It's older than Cheetah, has a larger user
base, and has better examples and docs at the moment. Cheetah, however, has a
number of advantages over Velocity:
\begin{itemize}
\item Cheetah is written in Python. Thus, it's easier to use and extend.
\item Cheetah's syntax is closer to Python's syntax than Velocity's is to
Java's.
\item Cheetah has a powerful caching mechanism.  Velocity has no equivalent.
\item It's far easier to add data/objects into the namespace where \$placeholder
     values are extracted from in Cheetah.  Velocity calls this namespace a 'context'.
     Contexts are dictionaries/hashtables. You can put anything you want into a
     context, BUT you have to use the .put() method to populate the context; 
     e.g.,

\begin{verbatim}
VelocityContext context1 = new VelocityContext();
context1.put("name","Velocity");
context1.put("project", "Jakarta");
context1.put("duplicate", "I am in context1");
\end{verbatim}
     
     Cheetah takes a different approach. Rather than require you to manually
     populate the 'namespace' like Velocity, Cheetah will accept any existing
     Python object or dictionary AS the 'namespace'.  Furthermore, Cheetah
     allows you to specify a list namespaces that will be searched in sequence
     to find a varname-to-value mapping.  This searchList can be extended at
     run-time.
     
    If you add a `foo' object to the searchList and the `foo' has an attribute
    called 'bar', you can simply type \code{\$bar} in the template.  If the
    second item in the searchList is dictionary 'foofoo' containing
    \code{\{'spam':1234, 'parrot':666\}}, Cheetah will first look in the `foo'
    object for a `spam' attribute.  Not finding it, Cheetah will then go to
    `foofoo' (the second element in the searchList) and look among its
    dictionary keys for `spam'.  Finding it, Cheetah will select
    \code{foofoo['spam']} as \code{\$spam}'s value.

\item In Cheetah, the tokens that are used to signal the start of
     \$placeholders and \#directives are configurable. You can set them to any
     character sequences, not just \$ and \#.
\end{itemize}


%%%%%%%%%%%%%%%%%%%%%%%%%%%%%%%%%%%%%%%%%%%%%%%%%%%%%%%%%%%%%%%%%%%%%%%%%%%%%%%%
\subsection{Cheetah vs. WebMacro}
\label{comparisons.webmacro}

For a basic introduction to WebMacro, visit
\url{http://webmacro.org}.

The points discussed in section \ref{comparisons.velocity} also apply to the
comparison between Cheetah and WebMacro.  For further differences please refer
to \url{http://jakarta.apache.org/velocity/differences.html}.

%%%%%%%%%%%%%%%%%%%%%%%%%%%%%%%%%%%%%%%%%%%%%%%%%%%%%%%%%%%%%%%%%%%%%%%%%%%%%%%%
\subsection{Cheetah vs. Zope's DTML}
\label{comparisons.dtml}

For a basic introduction to DTML, visit
\url{http://www.zope.org/Members/michel/ZB/DTML.dtml}.

\begin{itemize}
\item Cheetah is faster than DTML.
\item Cheetah does not use HTML-style tags; DTML does.  Thus, Cheetah tags are
     visible in rendered HTML output if something goes wrong.
\item DTML can only be used with ZOPE for web development; Cheetah can be
     used as a standalone tool for any purpose.
\item Cheetah's documentation is more complete than DTML's.
\item Cheetah's learning curve is shorter than DTML's.
\item DTML has no equivalent of Cheetah's blocks, caching framework, 
     unified dotted notation, and \code{\#raw} directive.
\end{itemize}

Here are some examples of syntax differences between DTML and Cheetah:
\begin{verbatim}
<ul>
<dtml-in frogQuery>
 <li><dtml-var animal_name></li>
</dtml-in>
</ul>
\end{verbatim}

\begin{verbatim}
<ul>
#for $animal_name in $frogQuery
 <li>$animal_name</li>
#end for
</ul>
\end{verbatim}

\begin{verbatim}
<dtml-if expr="monkeys > monkey_limit">
  <p>There are too many monkeys!</p>
<dtml-elif expr="monkeys < minimum_monkeys">
  <p>There aren't enough monkeys!</p>
<dtml-else>
  <p>There are just enough monkeys.</p>
</dtml-if>
\end{verbatim}

\begin{verbatim}
#if $monkeys > $monkey_limit
  <p>There are too many monkeys!</p>
#else if $monkeys < $minimum_monkeys
  <p>There aren't enough monkeys!</p>
#else
  <p>There are just enough monkeys.</p>
#end if
\end{verbatim}

\begin{verbatim}
<table>
<dtml-in expr="objectValues('File')">
  <dtml-if sequence-even>
    <tr bgcolor="grey">
  <dtml-else>
    <tr>
  </dtml-if>    
  <td>
  <a href="&dtml-absolute_url;"><dtml-var title_or_id></a>
  </td></tr>
</dtml-in>
</table>
\end{verbatim}

\begin{verbatim}
<table>
#set $evenRow = 0
#for $file in $files('File')
  #if $evenRow
    <tr bgcolor="grey">
    #set $evenRow = 0
  #else
    <tr>
    #set $evenRow = 1
  #end if
  <td>
  <a href="$file.absolute_url">$file.title_or_id</a>
  </td></tr>
#end for
</table>
\end{verbatim}

The last example changed the name of \code{\$objectValues} to 
\code{\$files} because that's what a Cheetah developer would write.
The developer would be responsible for ensuring \code{\$files} returned a 
list (or tuple) of objects (or dictionaries) containing the attributes (or
methods or dictionary keys) `absolute\_url' and `title\_or\_id'.  All these
names (`objectValues', `absolute\_url' and `title\_or\_id') are standard parts
of Zope, but in Cheetah the developer is in charge of writing them and giving
them a reasonable behaviour.

Some of DTML's features are being ported to Cheetah, such as
\code{Cheetah.Tools.MondoReport}, which is based on the
\code{<dtml-in>} tag.  We are also planning an output filter as flexible as
the \code{<dtml-var>} formatting options.  However, neither of these are
complete yet.

%%%%%%%%%%%%%%%%%%%%%%%%%%%%%%%%%%%%%%%%%%%%%%%%%%%%%%%%%%%%%%%%%%%%%%%%%%%%%%%%
\subsection{Cheetah vs. Zope Page Templates}
\label{comparisons.zpt}

For a basic introduction to Zope Page Templates, please visit
\url{http://www.zope.org/Documentation/Articles/ZPT2}.

%%%%%%%%%%%%%%%%%%%%%%%%%%%%%%%%%%%%%%%%%%%%%%%%%%%%%%%%%%%%%%%%%%%%%%%%%%%%%%%%
\subsection{Cheetah vs. PHPLib's Template class}
\label{comparisons.php}

PHP (\url{http://www.php.net/}) is one of the few scripting languages
expressly designed for web servlets.  However, it's also a full-fledged
programming language with libraries similar to Python's and Perl's.  The
syntax and functions are like a cross between Perl and C plus some of their
own ideas (e.g.; a single array type serves as both a list and a dictionary,
'\$arr\[\] = "value";' appends to an array).

PHPLib (\url(http://phplib.netuse.de/) is a collection of classes for various
web objects (authentication, shopping cart, sessions, etc), but what we're
interested in is the \code{Template} object.  Differences from Cheetah:

\begin{itemize}
\item Templates consist of text with \code{\{placeholders\}} in braces.
\item Instead of a searchList, there is one flat namespace.  Every variable
     must be assigned via the \code{set\_var} method.  However, you can pass
     this method an array (dictionary) of several variables at once.
\item You cannot embed lookups or calculations into the template.  Every
     placeholder must be an exact variable name.
\item There are no directives.  You must do all display logic (if, for, etc)
     in the calling routine.  
\item There is, however, a ``block'' construct.  A block is a portion of text
     between the comment markers \code{<!-- BEGIN blockName --> \ldots
     <!-- END blockName>}.  The \code{set\_block} method extracts this text
     into a namespace variable and puts a placeholder referring to it in the
     template.  This has a few parallels with Cheetah's \code{\#block}
     directive but is overall quite different.  
\item To do the equivalent of \code{\#if}, extract the block. Then if true, do
     nothing.  If false, assign the empty string to the namespace variable.
\item To do the equivalent of \code{\#for}, extract the block.  Set any
     namespace variables needed inside the loop.  To parse one iteration, use
     the \code{parse} method to fill the block variable (a mini-template) into
     another namespace variable, appending to it.  Refresh the namespace
     variables needed inside the loop and parse again; repeat for each
     iteration.  You'll end up with a mini-result that will be plugged into the
     main template's placeholder.
\item To read a template definition from a file, use the \code{set\_file}
     method.  This places the file's content in a namespace variable.
     To read a template definition from a string, assign it to a namespace
     variable.
\item Thus, for complicated templates, you are doing a lot of recursive block
     filling and file reading and parsing mini-templates all into one flat
     namespace as you finally build up values for the main template.  In
     Cheetah, all this display logic can be embedded into the template using
     directives, calling out to Python methods for the more complicated tasks.
\item Although you can nest blocks in the template, it becomes tedious and
     arguably hard to read, because all blocks have identical syntax.  Unless
     you choose your block names carefully and put comments around them, it's
     hard to tell which blocks are if-blocks and which are for-blocks, or what
     their nesting order is.
\item PHPLib templates do not have caching, output filters, etc.
\end{itemize}

%%%%%%%%%%%%%%%%%%%%%%%%%%%%%%%%%%%%%%%%%%%%%%%%%%%%%%%%%%%%%%%%%%%%%%%%%%%%%%%%
\subsection{Cheetah vs. PSP, PHP, ASP, JSP, Embperl, etc.}
\label{comparisons.pspEtc}

\begin{description}
\item[Webware's PSP Component] -- \url{http://webware.sourceforge.net/Webware/PSP/Docs/}
\item[Tomcat JSP Information] -- \url{http://jakarta.apache.org/tomcat/index.html}
\item[ASP Information at ASP101] -- \url{http://www.asp101.com/}
\item[Embperl] -- \url{http://perl.apache.org/embperl/}
\end{description}


Here's a basic Cheetah example:
\begin{verbatim}
<TABLE>
#for $client in $service.clients
<TR>
<TD>$client.surname, $client.firstname</TD>
<TD><A HREF="mailto:$client.email" >$client.email</A></TD>
</TR>
#end for
</TABLE>
\end{verbatim}

Compare this with PSP:

\begin{verbatim}
<TABLE>
<% for client in service.clients(): %>
<TR>
<TD><%=client.surname()%>, <%=client.firstname()%></TD>
<TD><A HREF="mailto:<%=client.email()%>"><%=client.email()%></A></TD>
</TR>
<%end%>
</TABLE>
\end{verbatim}

% Local Variables:
% TeX-master: "users_guide"
% End:      
