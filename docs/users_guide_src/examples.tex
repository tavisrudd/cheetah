\section{Examples}
\label{examples}

The Cheetah distribution comes with an 'examples' directory.  Browse the
files in this directory and its subdirectories for examples of how
Cheetah can be used.

%%%%%%%%%%%%%%%%%%%%%%%%%%%%%%%%%%%%%%%%%%%%%%%%%%%%%%%%%%%%%%%%%%%%%%%%%%%%%%%%
\subsection{Syntax examples}
The \code{Cheetah.Tests} module contains a large number of test cases that can
double as examples of how the Cheetah Language works.  To view these cases go
to the base directory of your Cheetah distribution and open the file
\code{Cheetah/Tests/SyntaxAndOutput.py} in a text editor.


%%%%%%%%%%%%%%%%%%%%%%%%%%%%%%%%%%%%%%%%%%%%%%%%%%%%%%%%%%%%%%%%%%%%%%%%%%%%%%%%
\subsection{Webware Examples}
The 'examples' directory has a subdirectory called 'webware\_examples'.  It
contains example servlets that use Webware.  

A subdirectory titled 'cheetahSite' contains a complete website example. It
is in fact the Cheetah web site
(\url{http://www.cheetahtemplate.org/}), although the content has not been kept
up to date.  The site demonstrates many advanced Cheetah features.  It also
demonstrates how the \code{cheetah-compile} program can be used to generate
Webware .py servlet files from .tmpl Template Definition files.

%% MO: Removed because Tavis deleted the directory:
% A subdirectory titled 'webwareSite' contains a complete website example. This
% site is my proposal for the new Webware website.  The site demonstrates the
% advanced Cheetah features.  It also demonstrates how the 
% \code{cheetah-compile} program can be used to generate Webware .py servlet
% files from .tmpl Template Definition files.

% Local Variables:
% TeX-master: "users_guide"
% End:      
