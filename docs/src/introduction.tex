\section{Introduction}
\label{intro}

%%%%%%%%%%%%%%%%%%%%%%%%%%%%%%%%%%%%%%%%%%%%%%%%%%%%%%%%%%%%%%%%%%%%%%%%%%%%%%%%
\subsection{What is Cheetah?}
\label{intro.whatIs}

Cheetah is a Python-powered template engine and code generator.  It can be used
as a standalone utility or it can be combined with other tools.  Cheetah has
many potential uses -- generating SQL, SGML or LaTeX source, PostScript, form
emails, etc. -- but web developers looking for a viable alternative to ASP, JSP,
PHP and PSP are expected to be its principle user group.

Cheetah:
\begin{itemize}        
\item works with HTML, SGML, XML, SQL, Postscript, form email, LaTeX, or any
     other text-based format.
     
\item makes it easy to cleanly separate content, graphic design, and program
     code.  This leads to highly modular, flexible, and reusable site
     architectures; faster development time; and HTML and program code that is
     easier to understand and maintain. It is particularly well suited for team
     efforts.
     
\item blends the power and flexibility of Python with the simplicity of a lean
     Template Definition Language that non-programmers can understand.  
     
\item gives template writers full access to any Python data structure,
     function, object, or method in their templates.
     
\item compiles `Template Definitions' into native Python code at startup and
     provides a simple, yet powerful, caching mechanism that can dramatically
     improve the performance of a dynamic website.

\end{itemize}   

Cheetah integrates tightly with {\bf Webware for Python}
(\url{http://webware.sourceforge.net/}): a Python-powered application server and
persistent servlet framework. Webware provides automatic session, cookie, and
user management and can be used with almost any operating-system, web server, or
database. Through Python, it works with XML, SOAP, XML-RPC, CORBA, COM, DCOM,
LDAP, IMAP, POP3, FTP, SSL, etc.. Python supports structured exception handling,
threading, object serialization, unicode, string internationalization, advanced
cryptography, and more. It can also be extended with code and libraries written
in C, C++, Java and other languages.

Like Python, Cheetah and Webware are Open Source Software and are supported by
active user communities.  Together, they are a powerful and elegant framework
for building dynamic web sites. 

Like its namesake, Cheetah is fast, flexible and powerful.

%%%%%%%%%%%%%%%%%%%%%%%%%%%%%%%%%%%%%%%%%%%%%%%%%%%%%%%%%%%%%%%%%%%%%%%%%%%%%%%%
\subsection{Give me an example!}
\label{intro.whatIs}

Here's a very simple example that illustrates some of Cheetah's basic syntax:

\begin{verbatim}

<HTML>
<HEAD><TITLE>$title</TITLE></HEAD>
<BODY>

<TABLE>
#for $client in $clients
<TR>
<TD>$client.surname, $client.firstname</TD>
<TD><A HREF="mailto:$client.email">$client.email</A></TD>
</TR>
#end for
</TABLE>

</BODY>
</HTML>
\end{verbatim}

Compare this with PSP:

\begin{verbatim}
<HTML>
<HEAD><TITLE><%=title%></TITLE></HEAD>
<BODY>

<TABLE>
<% for client in clients: %>
<TR>
<TD><%=client['surname']%>, <%=client'[firstname']%></TD>
<TD><A HREF="mailto:<%=client['email']%>"><%=client['email']%></A></TD>
</TR>
<%end%>
</TABLE>

</BODY>
</HTML>
\end{verbatim}


%% @@TR: I'm going to extend this and briefly introduce: 
%% - Template objects vs. .tmpl files.
%% - how to get data into it 

%%%%%%%%%%%%%%%%%%%%%%%%%%%%%%%%%%%%%%%%%%%%%%%%%%%%%%%%%%%%%%%%%%%%%%%%%%%%%%%%
\subsection{What is the philosophy behind Cheetah?}
\label{intro.philosophy}

Cheetah is guided by these principles:
\begin{itemize}
\item {\bf It should be easy to separate content, graphic design, and program code.}
     
     A clean separation makes it easier for a team of content writers,
     HTML/graphic designers, and programmers to work together without stepping
     on each other's toes and polluting each other's work.  The HTML framework
     and the content it contains are two separate things, and analytical
     calculations (program code) is a third thing.  Each team member should be
     able to concentrate on their specialty and to implement their changes
     without having to go through one of the others (i.e., the dreaded
     ``webmaster bottleneck'').
     
\item {\bf But, it should also be easy to integrate content, graphic design, and
       program code.}
     
     While it should be easy to develop content, graphics and program
     code separately, it should be easy to integrate them together into a 
     website.  In particular, it should be easy:

     \begin{itemize}
     \item for {\bf programmers} to create reusable components and functions
          that are accessible and understandable to designers.
     \item for {\bf designers} to mark out placeholders for content and 
          dynamic components in their templates.
     \item for {\bf designers} to soft-code aspects of their design that are
          either repeated in several places or are subject to change.
     \item for {\bf designers} to extend and customize existing templates and
          thus minimize duplication of effort and code.
     \item and, of course, for {\bf content writers} to use the templates that
          designers have created.
     \end{itemize}

\item {\bf Python for the complex tasks, Cheetah for the simple tasks.}
     
\item {\bf Non-programmers should be able to understand Cheetah's syntax.}
     
\end{itemize}

%%%%%%%%%%%%%%%%%%%%%%%%%%%%%%%%%%%%%%%%%%%%%%%%%%%%%%%%%%%%%%%%%%%%%%%%%%%%%%%%
\subsection{Who developed Cheetah?}
\label{intro.developers}

Cheetah is one of several templating frameworks that grew out of a `templates'
thread on the Webware For Python email list.  Tavis Rudd, Mike Orr, Chuck
Esterbrook, and Ian Bicking are the core developers.

%%%%%%%%%%%%%%%%%%%%%%%%%%%%%%%%%%%%%%%%%%%%%%%%%%%%%%%%%%%%%%%%%%%%%%%%%%%%%%%%
\subsection{How mature is Cheetah?}
\label{intro.mature}

Cheetah is alpha software as some aspects of its design are
still subject to change and the Users' Guide is incomplete.
We plan to release a stable version later in 2001.

Here's a summary of known issues and aspects of the design that are in flux:
\begin{itemize}

\item The \#include directive is being reworked to monitor for changes in the
     included file at run-time. It currently does the include once-off at
     compile-time.

\item The parsing of macros is currently being worked on to make it more
     robust.  However, the macro interface is stable.
     
\item Quoted string literals in \$placeholder argument sets is supposed to
     follow the same rules as Python, but there are some bugs.  'string\'s'
     stops at the second '.  (Believed to be an easy fix.)  """ ... """ and '''
     ... ''' fail when spanning multiple lines.

\end{itemize}

The syntax and usage directions described throughout the rest of this guide are
stable unless noted otherwise.

%%%%%%%%%%%%%%%%%%%%%%%%%%%%%%%%%%%%%%%%%%%%%%%%%%%%%%%%%%%%%%%%%%%%%%%%%%%%%%%%
\subsection{Where can I get releases?}
\label{intro.releases}

Download Cheetah releases from
\url{http://CheetahTemplate.sourceforge.net}

%%%%%%%%%%%%%%%%%%%%%%%%%%%%%%%%%%%%%%%%%%%%%%%%%%%%%%%%%%%%%%%%%%%%%%%%%%%%%%%%
\subsection{Where can I get news?}
\label{intro.news}

News and updates can be obtained from the the Cheetah website:
\url{http://CheetahTemplate.sourceforge.net}

Cheetah discussions take place on the list
\email{cheetahtemplate-discuss@lists.sourceforge.net}.

If you encounter difficulties, or are unsure about how to do something,
please post a detailed message to the list.

%%%%%%%%%%%%%%%%%%%%%%%%%%%%%%%%%%%%%%%%%%%%%%%%%%%%%%%%%%%%%%%%%%%%%%%%%%%%%%%%
\subsection{How can I contribute?}
\label{intro.contribute}

Cheetah is the work of many volunteers.  If you use Cheetah please share your
experiences, tricks, customizations, and frustrations.

\subsubsection{Bug reports and patches}

If you think there is a bug in Cheetah, send a message to the email list
with the following information:

\begin{enumerate}
\item a description of what you were trying to do and what happened
\item all tracebacks and error output
\item your version of Cheetah
\item your version of Python
\item your operating system
\item whether you have changed anything in the Cheetah installation
\end{enumerate}

\subsubsection{Example sites and tutorials}
If you're developing a website with Cheetah, please send a link to the
email list so we can keep track of Cheetah sites.  Also, if you discover
new and interesting ways to use Cheetah please share your experience and
write a quick tutorial about your technique.

\subsubsection{Macro libraries and function libraries}
We hope to build up a framework of macros libraries (see section
\ref{macros.libraries}) to distribute with Cheetah and would appreciate
any contributions.

\subsubsection{Test cases}
Cheetah is packaged with a regression testing suite that is run with each
new release to ensure that everything is working as expected and that recent
changes haven't broken anything.  The test cases are in the Cheetah.Tests
module.  If you find a reproduceable bug please consider writing a test case
that will pass only when the bug is fixed.  Send any new test cases to the email
list with the subject-line ``new test case for Cheetah.''

\subsubsection{Publicity}
Help spread the word ... recommend it to others, write articles about it, etc.

%%%%%%%%%%%%%%%%%%%%%%%%%%%%%%%%%%%%%%%%%%%%%%%%%%%%%%%%%%%%%%%%%%%%%%%%%%%%%%%%
\subsection{Acknowledgements}
\label{intro.acknowledgments}
    
We'd like to thank the following people for contributing valuable advice, code
and encouragement: Geoff Talvola, Jeff Johnson, Graham Dumpleton, Clark C.
Evans, Craig Kattner, Franz Geiger, Tom Schwaller, Rober Kuzelj, Jay Love,
Terrel Shumway, Sasa Zivkov, Arkaitz Bitorika, Jeremiah Bellomy, Baruch Even,
Paul Boddie, Stephan Diehl, Chui Tey, and Geir Magnusson.
    
The Velocity, WebMacro and Smarty projects provided inspiration and design
ideas.  Cheetah has benefitted from the creativity and energy of their
developers. Thank you.

%%%%%%%%%%%%%%%%%%%%%%%%%%%%%%%%%%%%%%%%%%%%%%%%%%%%%%%%%%%%%%%%%%%%%%%%%%%%%%%%
\subsection{License}
\label{intro.license}

\subsubsection{The gist}
Cheetah is open source, but there is no requirement that products developed with
or derivative to Cheetah become open source.

\subsubsection{Legal terms}
Copyright \copyright 2001, The Cheetah Development Team: Tavis Rudd, Mike Orr,
Chuck Esterbrook, Ian Bicking.

Permission to use, copy, modify, and distribute this software for any purpose
and without fee is hereby granted, provided that the above copyright notice
appear in all copies and that both that copyright notice and this permission
notice appear in supporting documentation, and that the names of the authors not
be used in advertising or publicity pertaining to distribution of the software
without specific, written prior permission.

THE AUTHORS DISCLAIM ALL WARRANTIES WITH REGARD TO THIS SOFTWARE, INCLUDING ALL
IMPLIED WARRANTIES OF MERCHANTABILITY AND FITNESS. IN NO EVENT SHALL THE AUTHORS
BE LIABLE FOR ANY SPECIAL, INDIRECT OR CONSEQUENTIAL DAMAGES OR ANY DAMAGES
WHATSOEVER RESULTING FROM LOSS OF USE, DATA OR PROFITS, WHETHER IN AN ACTION OF
CONTRACT, NEGLIGENCE OR OTHER TORTIOUS ACTION, ARISING OUT OF OR IN CONNECTION
WITH THE USE OR PERFORMANCE OF THIS SOFTWARE.

