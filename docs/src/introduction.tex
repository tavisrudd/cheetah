\section{Introduction}
\label{intro}

%%%%%%%%%%%%%%%%%%%%%%%%%%%%%%%%%%%%%%%%%%%%%%%%%%%%%%%%%%%%%%%%%%%%%%%%%%%%%%%%
\subsection{What is Cheetah?}
\label{intro.whatIs}

Cheetah is a Python-based template engine and code-generator.  It aims:

\begin{itemize}
\item {\bf to make it easy to separate content, graphic design, and program code.}
     
     Program code should not pollute HTML and HTML should not pollute program
     code. Nor should content pollute the structure of complex HTML designs and
     vice versa.

     There should be no need for a designer to work through a programmer to
     change a website's design or to use dynamic components that have already
     been coded.  Likewise, content-providers should not have to work through a
     'webmaster' to add new content to a site.  All members of the team should
     be able to work independently and in parallel.
     
     A clean separation makes it easier for a team of content-providers,
     designers, and programmers to work together without stepping on each
     other's toes and polluting each other's work. Other advantages include
     faster development time; HTML and program code that are easier to
     understand and maintain; content that can be displayed in a variety of
     non-HTML formats such as PDF; and highly modular, flexible, and reusable
     site architectures.       
     
\item {\bf to make it easy to integrate content, graphic design, and program code.}
     
     While it should be easy to develop content, graphic design, and program
     code separately, it should NOT be difficult to integrate them as part of a
     website.  There should be no difficult hoops to jump through.
     
     It should be easy:
     \begin{itemize}
     \item for programmers to create reusable components and functions that are
          accessible and understandable to designers.
     \item for designers to mark out placeholders for content and dynamic components
          in their templates.
     \item for designers to soft-code aspects of their design that are either
          repeated in several places or are subject to change.
     \item for designers to extend and customize existing templates and thus minimize
          duplication of effort and code.
     \item and, of course, for content-providers to use the templates that
          designers have created.
     \end{itemize}

     
\item {\bf to provide template designers with a small set of 'Display Logic'
       programming structures such as conditional blocks and 
       for loops}
     
     Graphic designers often do tasks that would be easier, faster, and less
     error prone if they had access to {\bf conditional blocks} and {\bf for
       loops}.  However, a full programming language would be overkill for these
     simple tasks and most designers don't have the time or desire to learn one.
     
\item {\bf to be equally well-suited for HTML, SGML, XML, SQL, Postscript, form
       email, LaTeX, or any other text-based format.}
     
     Although it was designed with dynamic websites and web applications in mind,
     Cheetah is not HTML-specific.
     
\item {\bf to achieve all these aims in a manner that is efficient, flexible, and
       extendable.}
     
\end{itemize}

Cheetah achieves these aims by:

\begin{itemize}     
     
\item blending the power and flexibility of Python with the simplicity of a
     small Template Definition language that non-programmers can understand.
     
\item giving template designers a simple way of accessing Python variables,
     objects, and functions in their templates.
     
\item providing a modular, object-orientated framework that makes it easy to
     create and maintain large websites.
     
\item compiling 'Template Definitions' into native Python code at startup.
     Thereafter this code is executed for each request.  This approach is
     dramatically faster than the string substitution approach used by many
     templating engines.

\item providing a very simple, yet powerful, caching mechanism that can
     significantly increase the responsiveness of a dynamic website.

\end{itemize}


%%%%%%%%%%%%%%%%%%%%%%%%%%%%%%%%%%%%%%%%%%%%%%%%%%%%%%%%%%%%%%%%%%%%%%%%%%%%%%%%
\subsection{Why is it called Cheetah?}
\label{intro.name}

Cheetah is fast, flexible, agile and graceful - like its namesake. 


%%%%%%%%%%%%%%%%%%%%%%%%%%%%%%%%%%%%%%%%%%%%%%%%%%%%%%%%%%%%%%%%%%%%%%%%%%%%%%%%
\subsection{Who developed Cheetah?}
\label{intro.developers}

Cheetah is one of several templating frameworks that grew out of a 'templates'
thread on the 'Webware For Python' email list.  Tavis Rudd, Mike Orr, Chuck
Esterbrook, Ian Bicking and Tom Schwaller are the core developers.

%%%%%%%%%%%%%%%%%%%%%%%%%%%%%%%%%%%%%%%%%%%%%%%%%%%%%%%%%%%%%%%%%%%%%%%%%%%%%%%%
\subsection{How mature is Cheetah?}
\label{intro.mature}

Cheetah is alpha/beta software as this User's Guide is incomplete and several aspects
of the design are still subject to change.  However, it has been tested
extensively and has few known issues.  We are hoping to release production
version 1.0 in the summer of 2001.

Here's a summary of known issues and aspects of the design that are in flux.
\begin{itemize}
\item The \#include directive is not working with relative path file includes
     when used with Webware. This should be resolved soon.
\item The \#include directive might be reworked to monitor for changes in the
     included file at run-time. It currently does the include once-off at
     compile-time.
\item The implementation of \$placeholders(WithArgstrings) needs to be fleshed
     out to handle nesting.
\end{itemize}

%%%%%%%%%%%%%%%%%%%%%%%%%%%%%%%%%%%%%%%%%%%%%%%%%%%%%%%%%%%%%%%%%%%%%%%%%%%%%%%%
\subsection{Where can I get releases?}
\label{intro.releases}

Cheetah releases can be downloaded from
\url{http://CheetahTemplate.sourceforge.net}

%%%%%%%%%%%%%%%%%%%%%%%%%%%%%%%%%%%%%%%%%%%%%%%%%%%%%%%%%%%%%%%%%%%%%%%%%%%%%%%%
\subsection{Where can I get news?}
\label{intro.news}

News and updates can be obtained from the the Cheetah website:
\url{http://CheetahTemplate.sourceforge.net}

Cheetah discussions take place on the list
\email{cheetahtemplate-discuss@lists.sourceforge.net}.

If you encounter difficulties, or are unsure about how to do something,
please post a detailed message to the list.

%%%%%%%%%%%%%%%%%%%%%%%%%%%%%%%%%%%%%%%%%%%%%%%%%%%%%%%%%%%%%%%%%%%%%%%%%%%%%%%%
\subsection{How can I contribute?}
\label{intro.contribute}

Cheetah is the work of many volunteers.  If you use Cheetah please share your
experiences, tricks, customizations, and frustrations.

\subsubsection{Bug reports and patches}

If you think there is a bug in Cheetah, send a message to the email list
with the following information:

\begin{enumerate}
\item a description of what you were trying to do and what happened
\item all tracebacks and error output
\item your version of Cheetah
\item your version of Python
\item your operating system
\item whether you have changed anything in the Cheetah installation
\end{enumerate}

\subsubsection{Example sites and tutorials}
If you're developing a website with Cheetah, please send a link to the
email list so we can keep track of Cheetah sites.  Also, if you discover
new and interesting ways to use Cheetah please share your experience and
write a quick tutorial about your technique.

\subsubsection{Macro libraries}
We hope to build up a framework of macros libraries (see section
\ref{macros.libraries}) to distribute with Cheetah and would appreciate
any contributions.

\subsubsection{Test cases}
Cheetah is packaged with a regression testing suite that is run with each
new release to ensure that everything is working as expected and that recent
changes haven't broken anything.  The test cases are in the Cheetah.Tests
module.  If you find a reproduceable bug please consider writing a test case
that will pass only when the bug is fixed.  Send any new test cases to the email
list with the subject-line ``new test case for Cheetah.''

\subsubsection{Publicity}
Help spread the word ... recommend it to others, write articles about it, etc.

%%%%%%%%%%%%%%%%%%%%%%%%%%%%%%%%%%%%%%%%%%%%%%%%%%%%%%%%%%%%%%%%%%%%%%%%%%%%%%%%
\subsection{Acknowledgements}
\label{intro.acknowledgments}
    
We'd like to thank the following people for contributing valuable advice, code
and encouragement: Geoff Talvola, Jay Love, Terrel Shumway, Sasa Zivkov, Arkaitz
Bitorika, Jeremiah Bellomy, Baruch Even, Paul Boddie, Stephan Diehl, and Geir
Magnusson.
    
The Velocity, WebMacro and Smarty projects provided inspiration and design
ideas.  Cheetah has benefited from the creativity and energy of their
developers. Thank you.

%%%%%%%%%%%%%%%%%%%%%%%%%%%%%%%%%%%%%%%%%%%%%%%%%%%%%%%%%%%%%%%%%%%%%%%%%%%%%%%%
\subsection{License}
\label{intro.license}

Cheetah is released for unlimited distribution under the terms of the
Python license.

