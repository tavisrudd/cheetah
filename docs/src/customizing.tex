\section{Customizing and extending Cheetah}
\label{customizing}

To extend and customize Cheetah you can create a {\bf sub-class} of Template that
reimplements some of its methods, you can use {\bf settings}, or you can use
{\bf plugins}.

%%%%%%%%%%%%%%%%%%%%%%%%%%%%%%%%%%%%%%%%%%%%%%%%%%%%%%%%%%%%%%%%%%%%%%%%%%%%%%%%
\subsection{Sub-classes of the Template class}

This is an advanced topic that is difficult to explain without full code
examples.  It's best to learn how to do this by looking at existing sub-classes
of the \code{Template} class.  The Cheetah.Servlet module and the
Cheetah.Templates package contain examples.


%%%%%%%%%%%%%%%%%%%%%%%%%%%%%%%%%%%%%%%%%%%%%%%%%%%%%%%%%%%%%%%%%%%%%%%%%%%%%%%%
\subsection{Settings}

\code{Template} objects have a \code{\_settings} attribute. This is a dictionary of
configuration settings that control most of Cheetah's core behaviour.  The
Template object's constructor method has a keyword argument 'settings' that
accepts a dictionary to override the default settings.  Many of these settings
are for internal use and are of little interest to end-users.  Some, however,
can be quite useful.  This section explains how to use them.

%%%%%%%%%%%%%%%%%%%%%%%%%%%%%%%%%%%%%%%%%%%%%%%%%%%%%%%%%%%%%%%%%%%%%%%%%%%%%%%%
\subsubsection{Using tokens other than the dollar sign for placeholders}

It is possible to use any character sequence to signal the start of a
\$placeholder.  The \code{placeholderStartToken} setting controls this.

\begin{verbatim}
myTemplateDef = "Here's my @@adj template."
myNamespace = {'adj': 'silly'}
from Cheetah.Template import Template
templateObj = Template(myTemplateDef, myNamespace, 
                 settings={'placeholderStartToken':'@@'})
\end{verbatim}

\code{placeholderStartToken} can be any character sequence of any length
provided that it doesn't end or start with \code{\#} as this is used by the
\#directives.  In the future we'd like to soft-code the
\code{directiveStartToken} as well.  You can't use a sequence of spaces as
the token, although you can have spaces mixed with other characters.


%%%%%%%%%%%%%%%%%%%%%%%%%%%%%%%%%%%%%%%%%%%%%%%%%%%%%%%%%%%%%%%%%%%%%%%%%%%%%%%%
\subsubsection{Disabling 'auto-calling'}

%% @@ TR: insert a \ref{} in the next line
Cheetah's auto-calling behaviour (see Section \#\#\#) can be 

\begin{verbatim}
from Cheetah.Template import Template
templateObj = Template(myTemplateDef, myNamespace, 
                 settings={'useAutocalling':0,})
\end{verbatim}


%%%%%%%%%%%%%%%%%%%%%%%%%%%%%%%%%%%%%%%%%%%%%%%%%%%%%%%%%%%%%%%%%%%%%%%%%%%%%%%%
\subsubsection{Unknown placeholder names}
%% handlers

The 'varNotFound\_handler' setting controls what happens when a \$placeholder
is not found in the searchList.  The default setting simply places the
Placeholder Tag in the output.  The ``Big Warning'' handler puts some
harder-to-miss decorations around it.  To switch to the Big Warning handler:

\begin{verbatim}
from Cheetah import CodeGenerator, Template
templateObj = Template(templateDef, namespace
                settings={'varNotFound_handler':CodeGenerator.varNotFound\_bigWarning})
\end{verbatim}

Defining your own handler is easy too, as you can see from the reprint of
Cheetah's handlers below.  Just define a function that takes two arguments, 
(1) the current TO and (2) the Placeholder Name as a string.  Return whatever
you want Cheetah to insert as the Placeholder Value.

\begin{verbatim}
## varNotFound handlers ##
def varNotFound\_echo(templateObj, tag):
    return "$" + tag

def varNotFound_bigWarning(templateObj, tag):
    return "="*15 + "&lt;$" + tag + " could not be found&gt;" + "="*15
    
\end{verbatim}

%%%%%%%%%%%%%%%%%%%%%%%%%%%%%%%%%%%%%%%%%%%%%%%%%%%%%%%%%%%%%%%%%%%%%%%%%%%%%%%%
\subsection{Plugins}

Plugins are objects that are scripted to automatically configure and extend the
Template object.  They are used to make it easy to complex modifications of
Cheetah's behaviour that would be awkward or too repetitive to do with {\bf
  settings} alone. This section explains how to use pre-built plugins. If you
want to build your own plugin please read the documented source code in the
Cheetah.Plugins package.

Cheetah.Template.Template() 'plugins' argument accepts a list of plugins. To use
a plugin import its class from the module that contains it and feed Template()
a list containing an instance of the class. Here's an example:

\begin{verbatim}
from Cheetah.Template import Template
from Cheetah.Plugins.MyPlugin import MyPlugin
...
TO = Template(templateDef, namespace, plugins=[MyPlugin(),])
\end{verbatim}


\subsubsection{The PSP plugin}
Cheetah ships with a plugin that enables pure PSP-style code to be used along
with Cheetah \$placeholders and \#directives.  PSP syntax can also be used
alone without \$placeholders and \#directives.

Here's a trivial example of how to use it:
\begin{verbatim}
from Cheetah.Template import Template
from Cheetah.Plugins.PSP import PSPplugin

templateDef = """
Testing Cheetah's PSP plugin:
 
$testVar
<% pspVar = 'X' %>
#set $list = [1,2,3]
 
#for $i in map(lambda x: x*x*x, $list)
$i
<%for j in range(15):%> <%=j*15%><%=pspVar%><%end%>
#end for
"""
print Template(templateDef, {'testVar':1234}, plugins=[PSPplugin()])

\end{verbatim}

