\section{Vocabulary}
\label{vocabulary}

To avoid repetition, all Cheetah-specific terms are defined here.  Terms are in
{\bf boldface} at their main definition, and abbreviations are in (parentheses)
immediately afterward.  

If you are a new user, you do not need to learn all these terms now.
Just come back to this page when you encounter a term you don't recognize.

%%%%%%%%%%%%%%%%%%%%%%%%%%%%%%%%%%%%%%%%%%%%%%%%%%%%%%%%%%%%%%%%%%%%%%%%%%%%%%%%
\subsection{Template object terms}

\begin{description}

\item{Template Object (TO, template)}  An instance of the class
     \code{Cheetah.Template.Template} or one of its subclasses.  A 
     TO contains a Template Definition and a SearchList.  Informally, 
     {\bf ``the template''} means a certain TO.

\item{Template Definition (TD)}  A string which is the source of the template.
     People sometimes incorrectly call this ``the template'', but it's better
     to call it ``the TD'' for clarity.

\item{Template Definition Language (TDL)}  The markup language used for
     Template Definitions.  This language defines Placeholder Tags and
     Directive Tags for the {\bf changeable data} in the template.

\item{Search List (SL)}  A list of Python objects which will be searched to find
     values for the placeholders in the TO.  Each object is called a
     {\bf Namespace (NS)} and must contain attributes, keys or subscripts to
     search in.  

\item{Filled Template (FT)}  A string which is the result of filling in the
     changeable data in the TO with their current values.
     Creating a Filled Template is called ``{\bf filling} the TO'' or 
     informally ``filling in the template''.

\item{Setting}  A configuration value in the configuration file or elsewhere.

\item{Parent Template Definition (Parent TD)}  The TD that is being extended
     when the \code{\#extend} directive is used.

\end{description}

%%%%%%%%%%%%%%%%%%%%%%%%%%%%%%%%%%%%%%%%%%%%%%%%%%%%%%%%%%%%%%%%%%%%%%%%%%%%%%%%
\subsection{Tag and value terms}

\begin{description}

\item{Directive Tag (\#D, directive)}  All tags that begin with ``\#''; e.g.,
     \code{\#if}.  These are commands or {\bf macros (\#M)}.  (See the
     ``Functions and Macros'' chapter for a definition of macros.)
     {\bf Display logic tags} is an informal term for the tags 
     \code{\#if} and \code{\#for}.

\item{Placeholder Tag (\$P, placeholder)}  All tags that begin with ``\$''.
     These contain a {\bf Placeholder Name (name)} which is looked up in the
     Search List.  A Placeholder Name may consist of one or more
     {\bf Identifiers} separated by periods, called {\bf Dotted Notation}.
     Identifiers may be followed by {\bf arguments} in ``()'' or ``[]''.  See
     the chapter ``Placeholder Tags'' for more details.  Note: this syntax is
     very similar to Python's although not identical.

\item{Namespace Key (NK)}  An attribute, key or subscript in one of the
     Namespaces.  This will be compared with the first Identifier in a
     Placeholder Name.

\item{Namespace Value}  The value of a Namespace Key.

\item{Placeholder Value (\$V, PV, value)}  The final value which will replace
     a Placeholder after the appropriate Namespace Value has been found and all
     conversions have been done.

\item{function/method (FM)}  A Python function or method.  These are the only
     types which may be {\bf autocalled}.  See the definition of ``compile''
     below for the reason.  Note: other callable objects
     (especially classes and instances) are {\em not} FMs and are never
     autocalled.

\item{compile}  When you instantiate a TO or call its \code{.startServer()}
     method, Cheetah compiles the Template Definition into Python code and then
     compiles that into Python object code.  This greatly increases the speed
     of producing a Filled Template, an important consideration when one TO
     is used to produce several Filled Templates, and a feature most other
     templating systems lack.  The compilation process looks up the Namespace
     Value for each Placeholder and determines whether it is autocallable or
     not; this is needed by the Python code, and is the reason you must not
     change a Namespace Value from autocallable to non-autocallable without
     recompiling the TO.  Further information about autocalling is in the
     ``Placeholder Tags'' chapter.


\end{description}


