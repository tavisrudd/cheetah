\section{Glossary}
\label{glossary}

To avoid confusion when discussing different aspects of the Cheetah system, all
Cheetah-specific terms are defined here.  If you are posting to the mailing
list, writing cheetah code or writing cheetah examples please use these terms.

\begin{description}

\item{Template Definition Language (TDL)}  The markup language used for
     Template Definitions.
     
\item{Template Object (TO, template)} An instance of the class
     \code{Cheetah.Template.Template} or one of its subclasses.  In the source
     code it is referred to as 'templateObj'.
     
\item{Template Definition (TD)} A string marked up in the TDL which is used to
     created a template object.  In the source code it is referred to as
     'templateDef'.
     
\item{Template Definition Extension} A string marked up in the TDL which is used
     to extend an existing template definition. A template definition extension
     usually contains a number of \code{\#redefine} directives that redefine blocks
     declared in the parent template definition. In the source code it is
     referred to as 'templateExt'.
     
\item{Search List} A list of Python data structures which will be searched to
     find values for the \$placeholders in a template definition.  Each object
     is referred to as a {\bf Namespace} and must be either a dictionary or an
     object. In the source code it is referred to as 'searchList'.

\item{Filled Template} A string which is the result of filling in the
     changeable data in the TO with their current values.

\item{Directive Tag (\#directive, \#D)}  All tags that begin with ``\#''; e.g.,
     \code{\#if}.  

\item{Placeholder Tag (\$placeholder, \$P)}  All tags that begin with ``\$''.
     
\item{compile} When you instantiate a template object or call its
     \code{.compileTemplate()} method, Cheetah compiles the Template Definition
     into Python code.

\end{description}


% Local Variables:
% TeX-master: "users_guide"
% End:      
