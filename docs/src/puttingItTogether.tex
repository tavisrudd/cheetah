\section{Putting it all together}
\label{puttingItTogether}

This section explains how to use Cheetah as a stand-alone tool and gives an
under-the-hood view of how Cheetah works.  If you are using Cheetah with Webware
and want to get straight into it, skip ahead to section
\ref{webware.inheritance.tmpl}.

%%%%%%%%%%%%%%%%%%%%%%%%%%%%%%%%%%%%%%%%%%%%%%%%%%%%%%%%%%%%%%%%%%%%%%%%%%%%%%%%
\subsection{The Template class}
\label{puttingItTogether.TO}
The \code{Template} class (\code{Cheetah.Template.Template}) is the heart of
Cheetah.  Objects of this class parse and compile template definitions into a
chunk of Python code.  This chunk of Python code is then bound to the template
object's \code{.__str__()} and \code{.respond()} methods, and is executed each
time you request the filled output from the template by calling either of these
methods.  

The \code{.__str__()} method is called automatically by Python's print keyword:
\code{print templateObj}.  


%%%%%%%%%%%%%%%%%%%%%%%%%%%%%%%%%%%%%%%%%%%%%%%%%%%%%%%%%%%%%%%%%%%%%%%%%%%%%%%%
\subsubsection{Constructing template objects}

Template objects may be created manually in Python code or automatically using
\code{.tmpl} files when Cheetah is used with Webware. See section
\ref{webware.inheritance.tmpl} for more information on \code{.tmpl} files.  The
section only deals with manual creation.

The constructor method of the \code{Template()} class 

You must pass either a Template Definition string or a 'file' keyword argument,
but not both.  

Here are some examples of the various ways to create a template object:
\begin{description}
\item{\code{templateObj = Template("The king is a \$placeholder1.")}}
     Pass the Template Definition as a string.
\item{\code{templateObj = Template(file="fink.ctd")}}
     Read the Template Definition from a file named "fink.ctd".  Note, do not
     confuse \code{.ctd} files with the \code{.tmpl} files that may be used
     directly with Webware (see section \ref{webware.inheritance.tmpl})
\item{\code{templateObj = Template(file=f)}}
     Read the Template Definition from file-like object 'f'.
\item{\code{templateObj = Template("The king is a \$placeholder1.", dict, obj)}}
     Pass the Template Definition as a string.  Also pass two Namespaces for the
     searchList: a dictionary 'dict' and an instance 'obj'.
\item{\code{templateObj = Template(None, dict, obj, file="fink.txt")}}
     Same, but pass a filename instead of a string.  The \code{None} is required
     here to represent the missing Template Definition string -- this due to
     Python's rules for positional parameters.
\item{\code{templateObj = Template(None, dict, obj, file="fink.txt")}}
     Same with a file object.
\end{description}

The following usage examples are not allowed:
\begin{verbatim}
templateObj = Template() 
templateObj = Template("The king is a $placeholder1", file="fink.txt")
templateObj = Template("The king is a $placeholder1", file=f)
\end{verbatim}

Certain other keyword arguments are also accepted, see 
``Keyword arguments accepted by the constructor'' below.

%%%%%%%%%%%%%%%%%%%%%%%%%%%%%%%%%%%%%%%%%%%%%%%%%%%%%%%%%%%%%%%%%%%%%%%%%%%%%%%%
\subsubsection{Keyword arguments accepted by the constructor}

None of these keyword arguments are necessary for normal usage of a TO.

\begin{description}

\item(file) Discussed above in ``Constructing Template Objects''.

\item{macros} Dictionary of macro name (string) -> macro definition (function).
     These macros are defined before the TO is compiled.

\item{cheetahBlocks} Dictionary of block name (string) -> block definition
     (string).

% Why would a programmer want to call this?

\item{settings}  Dictionary of configuration settings to merge in.  This is
     often useful to specify an alternate behavior for missing Namespace Keys:
\begin{verbatim}
from Cheetah import CodeGenerator
settings = {'varNotFound_handler': CodeGenerator.varNotFound_bigWarning}
templateObj = Template(file="fink.txt", settings=settings)       # Or:
templateObj = Template("The king is a $placeholder1.", settings=settings)
\end{verbatim}
    

\item{overwriteSettings}  Dictionary of configuration settings.  Any existing
     configuration settings are erased.

\item{plugins}  List of Cheetah plugins to register, in addition to those
     specified by the configuration setting `plugins'.  Cheetah ships with two
     plugins so far: a debugger and a PSP interface.

\end{description}

     
Any positional arguments besides the first are Namespaces for the Search List.
The constructor builds a Search List as a flat list of Namespaces,
concatenated in the following order:

\begin{enumerate}

\item  All the Namespaces passed to the constructor, from left to right.

\item  The TO itself (its \code{self} attribute), 

\item  The Namespaces in the keyword argument \code{searchList} if present.  
     The argument value be a list or a tuple.

\end{enumerate}

Every Namespace must have Namespace Keys and Namespace Values.  This means it
must be an object that supports attribute/value pairs or key/value pairs.  
Instances, dictionaries, modules and classes all make suitable Namespaces.
The Namespace Keys must be strings; the Namespace Values may be any Python
object.

If the Placeholder Name uses Dotted Notation, the Namespace Value will itself
be used as a Namespace recursively to resolve each Identifier from left to 
right.  Each recursive ``Namespace'' must be a valid Namespace type 
{\em or} a Python list or list-compatible instance.  However, the top-level
Namespace Value may {\em not} be a list, because the first Identifier in a
Placeholder Name may not be numeric.  (If this paragraph is unclear, read it
again after you're read the ``Placeholder Tags'' chapter.)

%%%%%%%%%%%%%%%%%%%%%%%%%%%%%%%%%%%%%%%%%%%%%%%%%%%%%%%%%%%%%%%%%%%%%%%%%%%%%%%%
\subsection{Using Template objects}

To use a TO, simply print it, apply Python's
\code{str()} function to it, or call its \code{.respond} method.  This is
called ``filling'' the TO or informally,  ``filling in the template''.  The
result is a string with all the Placeholders filled in with their current
Placeholder Values, and all the directives executed.  The resulting string is
called the Filled Template.

\begin{verbatim}
>>> print templateObj
<HTML>
<HEAD><TITLE>Hello World Example</TITLE></HEAD>
<BODY>
Hello World!
</BODY>
</HTML>
\end{verbatim}

To prove that all three templateObj's produce identical results, we'll run an
equality test on their Filled Templates.  The test will print ``1'' if the
Filled Templates are identical, or ``0'' if they are not.

\begin{verbatim}
>>> str(templateObj) == str(templateObj2) == str(templateObj3)
1
\end{verbatim}


%%%%%%%%%%%%%%%%%%%%%%%%%%%%%%%%%%%%%%%%%%%%%%%%%%%%%%%%%%%%%%%%%%%%%%%%%%%%%%%%
\subsection{Modifying Template objects}

From the same TO you can produce as many Filled Templates as you
want, each with its own set of changeable data.  To set the changeable data for
the next Filled Template, simply modify the Namespaces in the Search List.

\begin{verbatim}
>>> nameSpace['contents'] = 'Goodbye, cruel world!'
>>> print templateObj
<HTML>
<HEAD><TITLE>Hello World Example</TITLE></HEAD>
<BODY>
Goodbye, cruel world!
</BODY>
</HTML>
\end{verbatim}

Or for templateObj's 2 and 3:

\begin{verbatim}
>>> templateObj2.contents = 'Goodbye, cruel world!'
>>> str(templateObj) == str(templateObj2) == str(templateObj3)
1
\end{verbatim}

(templateObj3 didn't have to be modified because it's reading the same
\code{nameSpace} dictionary we already modified.)

All the Namespace Values above are strings, but they can actually be any Python
type.  You may even change values from one type to another.  However, if you
change a value from an autocalled type (a function or method) to a
non-autocalled type (any other type) or vice-versa, you must recompile the
TO by calling the \code{.compileTemplate()} method {\em before filling it}.
Otherwise, Cheetah will autocall when it shouldn't or
won't autocall when it should.  For the reason why, see the section
``Autocalling'' in the chapter ``Placeholder Tags''.

% Mention the exact consequences of wrongly autocalling or wrongly not
% autocalling.



%%%%%%%%%%%%%%%%%%%%%%%%%%%%%%%%%%%%%%%%%%%%%%%%%%%%%%%%%%%%%%%%%%%%%%%%%%%%%%%%
\subsection{Methods}

None of these methods are necessary for normal usage of a TO.

\begin{description}

\item{.addToSearchList(theNamespace, restart=True)}  Append the given Namespace
     to the Search List.  Leave the optional second argument at its default
     value so that the template gets recompiled.  (This method is of dubious
     value and may be removed.)

\item{.defineTemplateBlock(blockName, blockContents)}  Add a named block to the
template.  Both arguments must be strings.

% What good is it to define a template block?  Does this append the block to
% the Template Definition?  When will the block be visible in the Filled
% Template?

\item{.extendTemplate(extension)}  The docstring just says, ''@@needs
     documenting'').  Presumably, 'extension' is a string.  Used to 
     programmatically create nested documents?

\item(.killTemplateBlock(*blockNames))  Delete the named blocks from the 
Template Definition.  They will no longer show up in Filled Templates.

% I propose we delete .killTemplateBlock.  The #if directive covers all its
% functionality and more.  Also, .killTemplateBlock is destructive in regard to
% future Filled Templates: there's no way to get the block back if you need it
% again later, whereas #if will happily hide and show blocks as appropriate.

\item{.loadMacro, .loadMacros and .loadMacrosFromModule}  See the chapter
     ``Functions and Macros''.

\item{.getFileContents(fileName) -> string}  Returns the contents of the named
     file.  May be overridded to do, e.g., URL retrievals.

\item{.registerServerPlugin(plugin)}  Register a plugin that extends the
     functionality of the Template.  See ``plugins'' in the section on keyword
     arguments.

\item{.runAsMainProgram()}

\item{.searchList()}  Return the Search List.  It's a UserList subclass, so you
     can use Python's standard list operations on it.  If you add or delete
     namespaces, call \code{.compileTemplate()} afterwards to recompile the template
     {\em before trying to fill it}.

\item{.compileTemplate()}  Recompile the template.  

\item{.recompile()}  Synonym for \code{.compileTemplate}.

\item{.translatePath()}  ``A hook to enable proper handling of server-side
     paths with Webware.''
     
\item{.getUnknowns()}  Return a list of Placeholder Names which are missing in
     the Search List.

\end{description}

Since the Template class inherits from the SettingsManager mixin, those methods
are available to change configuration settings after instantiation.

