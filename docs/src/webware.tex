%%%%%%%%%%%%%%%%%%%%%%%%%%%%%%%%%%%%%%%%%%%%%%%%%%%%%%%%%%%%%%%%%%%%%%%%%%%%%%%%
\section{Using Cheetah with Webware}
\label{webware}

%%%%%%%%%%%%%%%%%%%%%%%%%%%%%%%%%%%%%%%%%%%%%%%%%%%%%%%%%%%%%%%%%%%%%%%%%%%%%%%%
\subsection{Background}
\label{webware.background}

Webware is a 'Python-Powered Internet Platform' that uses servlets in a manner
similar to Java servlets.  'WebKit' is the name of Webware's application server.
For more details please visit http://webware.sourceforge.net.

As Cheetah's core is flexible there are many ways to use it with Webware
servlets.  There are two broad categories: the Inheritance approach and
the Containment approach.  In the Inheritance approach a servlet is created that
subclasses both the Template class and Webware's HTTPServlet class.  The
Template instance IS the servlet and its .respond() method is
automatically called by WebKit for each request.  All pre-request processing is
handled via Cheetah.

In the Containment approach an instance of the Template class is wrapped
up inside of a Webware servlet class.  Instances of the servlet class must
explicitly call the Template instance's .respond(), or .__str__(), method
for each request.  In this case the servlet class can handle whatever
per-request processing needs to be done before it calls Cheetah.respond().

The Inheritance approach is the simplest and is best suited for building sites
from scratch.

The Containment approach is slightly more complex and is best suited for use
with existing Webware servlets.  It is also ideal for cases where you wish to
use Cheetah for only a portion of the servlet's output, such as a
discussion-forum table at the bottom of a webpage. 

%%%%%%%%%%%%%%%%%%%%%%%%%%%%%%%%%%%%%%%%%%%%%%%%%%%%%%%%%%%%%%%%%%%%%%%%%%%%%%%%
\subsection{Using the Inheritance approach}
\label{webware.inheritance}

Cheetah.Servlet provides a servlet class that can be subclassed to create 
Webware servlets, as in the trivial example below.

\begin{verbatim}
## FILE: hello_world.py ##
template = """
<HTML>
<HEAD><TITLE>'Hello World - Test Servlet</TITLE></HEAD>
<BODY>
Hello World!
</BODY>
</HTML>
"""
from Cheetah.Servlet import TemplateServlet
class hello_world(TemplateServlet):
    def __init__(self):
        TemplateServlet.__init__(self, template) 
\end{verbatim}

TemplateServlet's constructor method (\code{TemplateServlet.__init__()}) adds
the attribute dictionary of the servlet to the searchList that
\code{\$placeholder} variables can extracted from.  Thus, attributes and methods
of the servlet object can be interpolated into the template like this:

\begin{verbatim}
## FILE: hello_world.py ##
template = """
<HTML>
<HEAD><TITLE>$title</TITLE></HEAD>
<BODY>
$contents
</BODY>
</HTML>
"""
from Cheetah.Servlet import TemplateServlet
class hello_world(TemplateServlet):
    title = 'Hello World - Test Servlet'
    def __init__(self):
        TemplateServlet.__init__(self, template) 

    def contents(self):
        return 'Hello World!'
\end{verbatim}

%%%%%%%%%%%%%%%%%%%%%%%%%%%%%%%%%%%%%%%%%%%%%%%%%%%%%%%%%%%%%%%%%%%%%%%%%%%%%%%%
\subsubsection{Using .tmpl files}
\label{webware.inheritance.tmpl}

This process can be simplified for non-programmers. All the Python wrapper code
in these examples can be generated automatically by \code{cheetah-compile}, a small program
that is installed with Cheetah. \code{cheetah-compile} parses Template Definitions
written in files with the .tmpl extension and generates a Webware servlet file
with the .py extension for each .tmpl file.  Type \code{cheetah-compile} after
installing Cheetah to get information on how to use it.

Here's the first example as a \code{.tmpl} file:

\begin{verbatim}
## FILE: hello_world.tmpl ##
<HTML>
<HEAD><TITLE>Hello World - Test Servlet</TITLE></HEAD>
<BODY>
Hello World!
</BODY>
</HTML>
\end{verbatim}

Here's the second example as a \code{.tmpl} file.  Note that all lines that begin with
\code{\#\#} are comment lines.

\begin{verbatim}
## FILE: hello_world.tmpl ##
##===================================
#data
title = 'Hello World - Test Servlet'
def contents():
    return 'Hello World!'
#/data
##===================================
<HTML>
<HEAD><TITLE>$title</TITLE></HEAD>
<BODY>
$contents
</BODY>
</HTML>
\end{verbatim}


%%%%%%%%%%%%%%%%%%%%%%%%%%%%%%%%%%%%%%%%%%%%%%%%%%%%%%%%%%%%%%%%%%%%%%%%%%%%%%%%
\subsubsection{The SkeletonPage framework}
\label{webware.inheritance.skeletonPage}

%\subsubsection{PlateKit}
%Forget what you knew about PlateKit. It's undergoing some changes and is out of
%action for the time-being.

%%%%%%%%%%%%%%%%%%%%%%%%%%%%%%%%%%%%%%%%%%%%%%%%%%%%%%%%%%%%%%%%%%%%%%%%%%%%%%%%
\subsection{Using the Containment approach}
\label{webware.containment}


%%%%%%%%%%%%%%%%%%%%%%%%%%%%%%%%%%%%%%%%%%%%%%%%%%%%%%%%%%%%%%%%%%%%%%%%%%%%%%%%
\subsection{User interaction in either approach}
\label{webware.userInteraction}

%%%%%%%%%%%%%%%%%%%%%%%%%%%%%%%%%%%%%%%%%%%%%%%%%%%%%%%%%%%%%%%%%%%%%%%%%%%%%%%%
\subsection{Components}
\label{webware.components}


%%%%%%%%%%%%%%%%%%%%%%%%%%%%%%%%%%%%%%%%%%%%%%%%%%%%%%%%%%%%%%%%%%%%%%%%%%%%%%%%
\subsubsection{Using components}
\label{webware.components.using}

%%%%%%%%%%%%%%%%%%%%%%%%%%%%%%%%%%%%%%%%%%%%%%%%%%%%%%%%%%%%%%%%%%%%%%%%%%%%%%%%
\subsubsection{Building components}
\label{webware.components.building}

