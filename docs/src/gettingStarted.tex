\section{Getting Started}
\label{gettingStarted}

%%%%%%%%%%%%%%%%%%%%%%%%%%%%%%%%%%%%%%%%%%%%%%%%%%%%%%%%%%%%%%%%%%%%%%%%%%%%%%%%
\subsection{Requirements}
Cheetah requires Python release 2.0 or greater and should run on any
operating system that Python 2.0 runs on.

%%%%%%%%%%%%%%%%%%%%%%%%%%%%%%%%%%%%%%%%%%%%%%%%%%%%%%%%%%%%%%%%%%%%%%%%%%%%%%%%
\subsection{Installation}

To install Cheetah for a single user:
\begin{enumerate}
\item copy the 'Cheetah' sub-directory  to a location that is in the user's
     PYTHONPATH
\end{enumerate}

To install Cheetah for system-wide use:
\begin{enumerate}
\item on POSIX systems (AIX, Solaris, Linux, IRIX, etc.) become the 'root' user
     and run: python setup.py install
     
\item On non-POSIX systems, such as Windows NT, login as an administrator and
     type this at the command prompt:  python setup.py install
\end{enumerate}


On POSIX systems, the system-wide installation will also install the Cheetah's
command-line compiler program, cheetah-compile, to the "bin/" directory of your
Python distribution.  Depending on your system, you may want to copy or symlink
this file to a directory on your system path for easy access.


%%%%%%%%%%%%%%%%%%%%%%%%%%%%%%%%%%%%%%%%%%%%%%%%%%%%%%%%%%%%%%%%%%%%%%%%%%%%%%%%
\subsection{Testing your installation}
You can run the test suite to insure that your installation is correct by
following these steps:
\begin{enumerate}
\item CD into the directory ./src/Tests   
\item type: \code{python Test.py} 
\end{enumerate}

If the tests pass, start Python in interactive mode and try the example in the
Introduction section of this guide.

If any of the tests fail please send a message to the email list with a copy of
the test output and the following details about your installation:

\begin{enumerate}
\item your version of Cheetah
\item your version of Python
\item your operating system
\item whether you have changed anything in the Cheetah installation
\end{enumerate}

