\section{Getting Started}
\label{gettingStarted}


%%%%%%%%%%%%%%%%%%%%%%%%%%%%%%%%%%%%%%%%%%%%%%%%%%%%%%%%%%%%%%%%%%%%%%%%%%%%%%%%
\subsection{Who should read this Guide?}

This Users Guide is for those wishing an overview, tutorial and reference for
the Cheetah template system.  A basic knowledge of Python is assumed.
This Guide also contains examples of integrating Cheetah with Webware for
Python, a web application development framework.  Knowledge of Webware is not
assumed, but of course you will have to learn Webware from its own 
documentation in order to build a Webware + Cheetah site.

%%%%%%%%%%%%%%%%%%%%%%%%%%%%%%%%%%%%%%%%%%%%%%%%%%%%%%%%%%%%%%%%%%%%%%%%%%%%%%%%
\subsection{Requirements}
\label{gettingStarted.requirements}

Cheetah requires Python release 2.0 or greater and should run on any
operating system that Python 2.0 runs on.

%%%%%%%%%%%%%%%%%%%%%%%%%%%%%%%%%%%%%%%%%%%%%%%%%%%%%%%%%%%%%%%%%%%%%%%%%%%%%%%%
\subsection{Installation}
\label{gettingStarted.install}

To install Cheetah for a single user:
\begin{enumerate}
\item copy the 'src' sub-directory  to a location that is in the user's
     PYTHONPATH and rename it as 'Cheetah'.
\end{enumerate}

To install Cheetah for system-wide use:
\begin{enumerate}
\item on POSIX systems (AIX, Solaris, Linux, IRIX, etc.) become the 'root' user
     and run: python setup.py install
     
\item On non-POSIX systems, such as Windows NT, login as an administrator and
     type this at the command prompt:  python setup.py install
\end{enumerate}


On POSIX systems, the system-wide installation will also install the Cheetah's
command-line compiler program, cheetah-compile, to the "bin/" directory of your
Python distribution.  

%%%%%%%%%%%%%%%%%%%%%%%%%%%%%%%%%%%%%%%%%%%%%%%%%%%%%%%%%%%%%%%%%%%%%%%%%%%%%%%%
\subsection{Testing your installation}
\label{gettingStarted.test}

You can run the test suite to insure that your installation is correct by
following these steps:
\begin{enumerate}
\item CD into the directory ./src/Tests   
\item type: \code{python Test.py} 
\end{enumerate}

If the tests pass, start Python in interactive mode and try the example in the
Introduction section of this guide.

If any of the tests fail please send a message to the email list with a copy of
the test output and the following details about your installation:

\begin{enumerate}
\item your version of Cheetah
\item your version of Python
\item your operating system
\item whether you have changed anything in the Cheetah installation
\end{enumerate}

%%%%%%%%%%%%%%%%%%%%%%%%%%%%%%%%%%%%%%%%%%%%%%%%%%%%%%%%%%%%%%%%%%%%%%%%%%%%%%%%
\subsection{Quickstart tutorial}
\label{gettingStarted.tutorial}

This tutorial briefly introduces the basic usage of Cheetah.  See the
following chapters for more detailed explanations.  

{\bf This tutorial will be fleshed out further at later date.} 

The core of Cheetah is the \code{Template} class in the \code{Cheetah.Template}
module. The following example shows how to use the \code{Template} class from an
interactive Python session. Lines prefixed with \code{>>>} and \code{...} are
user input.  The remaining lines are Python output.

\begin{verbatim}
>>> from Cheetah.Template import Template
>>> templateDef = """
... <HTML>
... <HEAD><TITLE>$title</TITLE></HEAD>
... <BODY>
... $contents
... ## this is a single-line Cheetah comment and won't appear in the output
... #* This is a multi-line comment
...    blah, blah, blah 
... *#
... </BODY>
... </HTML>"""
>>> nameSpace = {'title': 'Hello World Example', 'contents': 'Hello World!'}
>>> templateObj = Template(templateDef, nameSpace)
>>> print templateObj
 
<HTML>
<HEAD><TITLE>Hello World Example</TITLE></HEAD>
<BODY>
Hello World!
</BODY>
</HTML>
>>> print templateObj  # templateObj can be printed as many times as you need
 
<HTML>
<HEAD><TITLE>Hello World Example</TITLE></HEAD>
<BODY>
Hello World!
</BODY>
</HTML>

\end{verbatim}

% Local Variables:
% TeX-master: "users_guide"
% End:      

