\section{Examples}
\label{examples}

The Cheetah distribution comes with an 'examples' directory.  Browse the
files in this directory and its subdirectories for examples of how
Cheetah can be used.

%%%%%%%%%%%%%%%%%%%%%%%%%%%%%%%%%%%%%%%%%%%%%%%%%%%%%%%%%%%%%%%%%%%%%%%%%%%%%%%%
\subsection{Syntax examples}
Cheetah's \code{Tests} module contains a large number of test cases
that can double as examples of how the Template Definition Language works.
To view these cases go to the base directory of your Cheetah distribution
and open the file Cheetah/Tests.py in a text editor.


%%%%%%%%%%%%%%%%%%%%%%%%%%%%%%%%%%%%%%%%%%%%%%%%%%%%%%%%%%%%%%%%%%%%%%%%%%%%%%%%
\subsection{Webware Examples}
The 'examples' directory has a subdirectory called 'webware_examples'.  It
contains example servlets that use Webware.  

A subdirectory titled 'webwareSite' contains a complete website example. This
site is my proposal for the new Webware website.  The site demonstrates the
advanced Cheetah features such as the \code{\#data} and \code{\#redefine}
directives.  It also demonstrates how the TScompile program can be used to
generate Webware .py servlet files from .tmpl Template Definition files.

% Local Variables:
% TeX-master: "users_guide"
% End:      
