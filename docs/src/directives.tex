\section{Directive Tags}

% Will promote subsubsections to subsections.


Directives tags are used for all functionality that cannot be handled with
simple placeholders and are enclosed in \code{\#} and \code{/\#}.
Cheetah does not use HTML/XML style tags because they would be hard to
distinguish from real HTML tags and would not be visible in rendered HTML when
something goes wrong.

Some directives consist of a single tag while others have {\bf start} and
{\bf end} tags that surround a chunk of text.  End tags are written in the form
\code{\#end nameOfTheDirective/\#}.

\subsubsection{Escaping directives}
Directives can be escaped by placing a backslash ($\backslash$) before them.
Escaped directives will be printed verbatim.

\subsubsection{Tag closures: explicit and implicit}
Directive tags can closed explicitly with \code{/\#} or implicitly with the end
of the line if you're feeling lazy.
\begin{verbatim}
#block /#
Text in the contents area of the
block directive
#end block /#
\end{verbatim}
or
\begin{verbatim}
#block
Text in the contents area of the
block directive
#end block
\end{verbatim}

\subsubsection{Whitespace handling}


%%%%%%%%%%%%%%%%%%%%%%%%%%%%%%%%%%%%%%%%%%%%%%%%%%%%%%%%%%%%%%%%%%%%%%%%%%%%%%%%
\subsection{Comment directives}

Comment directives are used to mark notes, explanations, and decorative text
that should not appear in the output.  There are two forms of the comment
directive: single-line and multi-line.

All text in a template definition that lies between 2 hash characters
(\code{\#\#}) and the end of the line is treated as a single-line comment and
will not show up in the output, unless the 2 hash characters are escaped with a
backslash.
\begin{verbatim}
<HTML>
<HEAD><TITLE>$title</TITLE></HEAD>
<BODY>
##====================================  a decorative comment
$contents                               ## an end-of-line comment
##====================================
</BODY>
</HTML>
\end{verbatim}

Any text between \code{\#*} and \code{*\#} will be treated as a multi-line
comment.
\begin{verbatim}
<HTML>
<HEAD><TITLE>$title</TITLE></HEAD>
<BODY>
#*
   Here is some multiline
   comment text
*#
##====================================  a decorative comment
$contents                               ## an end-of-line comment
##====================================
</BODY>
</HTML>
\end{verbatim}

%%%%%%%%%%%%%%%%%%%%%%%%%%%%%%%%%%%%%%%%%%%%%%%%%%%%%%%%%%%%%%%%%%%%%%%%%%%%%%%%
\subsection{\#raw directives}
Any section of a template definition that is delimeted by \code{\#raw} and
\code{\#end raw} will be printed verbatim without any parsing of
\$placeholders or other directives.  This can be very useful for debugging or
writing Cheetah examples and tutorials.

%%%%%%%%%%%%%%%%%%%%%%%%%%%%%%%%%%%%%%%%%%%%%%%%%%%%%%%%%%%%%%%%%%%%%%%%%%%%%%%%
\subsection{\#include directives}

\code{\#include} directives are used to include text from outside the template
definition.  The text can come from \code{\$placeholder} variables or from
external files.  The example below demonstrates use with \code{\$placeholder}
variables.

\begin{verbatim}
#include $myParseText
\end{verbatim}

This example demonstrates its use with external files.
\begin{verbatim}
#include "includeFileName.txt"
\end{verbatim}

By default, included text will be parsed for Cheetah tags.  The keyword
{\bf raw} can be used to mark the text for verbatim inclusion without any tag
parsing.

\begin{verbatim}
#include raw $myParseText
#include raw "includeFileName.txt"
\end{verbatim}

\code{Template} uses its .getFileContents(fileName) method to locate the file to
be included.  This method can be overriden in subclasses if you want to modify
or extend its behaviour.  It is possible to implement the logic for getting 
remote files such as \code{http://myserver.com/file.txt}.


%%%%%%%%%%%%%%%%%%%%%%%%%%%%%%%%%%%%%%%%%%%%%%%%%%%%%%%%%%%%%%%%%%%%%%%%%%%%%%%%
\subsection{\#cache directives}


%%%%%%%%%%%%%%%%%%%%%%%%%%%%%%%%%%%%%%%%%%%%%%%%%%%%%%%%%%%%%%%%%%%%%%%%%%%%%%%%
\subsection{Display logic directives}

\subsubsection{Conditional blocks}

\subsubsection{For loops}

%%%%%%%%%%%%%%%%%%%%%%%%%%%%%%%%%%%%%%%%%%%%%%%%%%%%%%%%%%%%%%%%%%%%%%%%%%%%%%%%
\subsection{\#block directives}

\subsubsection{\#redefine directives}


