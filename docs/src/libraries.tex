%%%%%%%%%%%%%%%%%%%%%%%%%%%%%%%%%%%%%%%%%%%%%%%%%%%%%%%%%%%%%%%%%%%%%%%%%%%%%%%%
\section{Macro, Template and other libraries}
\label{libraries}

Cheetah comes ``batteries included'' with libraries of macros, templates and
other objects you can use in your own programs. If you develop your own, please
consider posting them on the mailing list so others can benefit. 

Some useful functions, and other objects, used by Cheetah are in the
\code{Cheetah.Utilities} module.  All utility modules contributed by third
parties are in the \code{Cheetah.Tools} package.

%%%%%%%%%%%%%%%%%%%%%%%%%%%%%%%%%%%%%%%%%%%%%%%%%%%%%%%%%%%%%%%%%%%%%%%%%%%%%%%%
\subsection{Macro libraries}
\label{libraries.macros}

The package \code{Cheetah.Macros} includes libraries of macros that can be
loaded into Templates via the \code{Template.loadMacrosFromModule()} method.
See section \ref{directives.macros.existingFunctions} for more information on
this method.  All of the functions contained in the macro libraries can also be
added to the \code{searchList} and used as regular \$placeholders if you wish.

%%%%%%%%%%%%%%%%%%%%%%%%%%%%%%%%%%%%%%%%%%%%%%%%%%%%%%%%%%%%%%%%%%%%%%%%%%%%%%%%
\subsubsection{Cheetah.Macros.HTML}
\label{libraries.macros.HTML}
The module \code{Cheetah.Macros.HTML} contains a number of functions that are
usefull for generating common HTML elements.

\begin{itemize}
\item {\bf \code{currentYr()}} returns the current year. This is useful in
     copyright strings.
\item {\bf \code{currentDate(formatString=''\%b \%d, \%Y'')}} Returns a string
   representation of the current date. See the documentation for the Python
   \code{time} module for the formatString codes. If you change the
   formatString, remember that when this function used as a macro the date
   will only be calculated once at compile-time
\item {\bf \code{spacer(width=1,height=1)}} creates an image tag for a
     transparent spacer image. This very useful for generating complex image
     tables. You'll need to have a single-pixel transparent gif image called
     \code{spacer.gif} in the same directory as your template file.  If you want
     to put the image elsewhere you can easily reimplement the function behind
     this macro.
\item {\bf \code{formHTMLTag(tagName, attributes=\{\})}} will create an HTML tag
     with the attributes given.
     
\item {\bf \code{formatMetaTags(metaTags)}} will auto-format a dictionary of
     meta-tags.  It accepts a dictionary that contains one or two
     sub-dictionaries under the keys \code{HTTP_EQUIV} and \code{NAME}. Each of
     the sub-dictionaries should contain key-value pairs that represent the
     name-contents pairs for each of the desired meta-tags.
\end{itemize}
            

%%%%%%%%%%%%%%%%%%%%%%%%%%%%%%%%%%%%%%%%%%%%%%%%%%%%%%%%%%%%%%%%%%%%%%%%%%%%%%%%
\subsection{Template libraries}
\label{libraries.templates}

The \code{Cheetah.Templates} package contains stock templates that you can
either use as is, or extend by using the \code{\#redefine} directive to redefine
specific {\bf blocks}.

%%%%%%%%%%%%%%%%%%%%%%%%%%%%%%%%%%%%%%%%%%%%%%%%%%%%%%%%%%%%%%%%%%%%%%%%%%%%%%%%
\subsubsection{Cheetah.Templates.SkeletonPage}
\label{libraries.templates.skeletonPage}

A stock template that will be very useful for web developers is defined in the 
\code{Cheetah.Templates.SkeletonPage} module.

\begin{verbatim}
$*docType
<HTML>
####################
#block headerComment
<!-- This document was autogenerated by Cheetah. Don't edit it directly!

Copyright #currentYr() - $*siteCopyrightName - All Rights Reserved.
Feel free to copy any javascript or html you like on this site,
provided you remove all links and/or references to $*siteDomainName
However, please do not copy any content or images without permission.

$*siteCredits

-->

#end block headerComment
#####################

#################
#block headTag /#
<HEAD>
<TITLE>$*title</TITLE>
$*metaTags
$*stylesheetTags
$*javascriptTags
</HEAD>
#end block headTag /#
#################


#################
#block bodyTag /#
$bodyTag
#end block bodyTag /#
#################

#block bodyContents /#
This skeleton page has no flesh. Its body needs to be implemented.
#end block bodyContents /#

</BODY>
</HTML>
\end{verbatim}

You can reimplement any of the blocks defined in this template by using the
\code{\#redefine} directive (see section \ref{directives.redefine}).

%%%%%%%%%%%%%%%%%%%%%%%%%%%%%%%%%%%%%%%%%%%%%%%%%%%%%%%%%%%%%%%%%%%%%%%%%%%%%%%%
\subsection{Cheetah.Utilities}
\label{libraries.utilities}

\begin{itemize}
\item {\bf \code{mergeNestedDictionaries(dict1, dict2)}} Recursively merge the
     values of dict2 into dict1.  This little function is very handy for
     selectively overriding settings in a settings dictionary that has a nested
     structure.
\item {\bf \code{removeDuplicateValues(list)}} Remove all duplicate values in a
     list.
\item {\bf \code{lineNumFromPos(string, pos)}} Calculate what line a position in
     a string lies on. This doesn't work on Mac-OS.
\item {\bf \code{getLines(string, sliceObj)}} Slice a string up into a list of
     lines and return a slice.
\item {\bf \code{insertLineNums(string)}} Return a version of the string with
     each line prefaced with its line number.
\end{itemize}

%%%%%%%%%%%%%%%%%%%%%%%%%%%%%%%%%%%%%%%%%%%%%%%%%%%%%%%%%%%%%%%%%%%%%%%%%%%%%%%%
\subsection{Cheetah.Tools}
\label{libraries.tools}


%%%%%%%%%%%%%%%%%%%%%%%%%%%%%%%%%%%%%%%%%%%%%%%%%%%%%%%%%%%%%%%%%%%%%%%%%%%%%%%%
\subsection{Cheetah.SettingsManager}
\label{libraries.SettingsManager}


