\section{The Template Definition Language}
\label{TDL}

{\bf Template definitions} are text strings, or files, that have been marked up
with Cheetah's {\bf Template Definition Language} for special processing.  This
language is not a general purpose programming language like Python.  Rather,
it's a mini-language that was designed to make HTML-generation, and
code-generation in general, easy enough for non-programmers to understand and
programmers to love.  It is purposefully limited and leaves complex tasks to
Python, where they belong.

Cheetah does not use HTML/XML style tags as they are hard to distinguish from
real HTML tags and are not be visible in rendered HTML when something goes
wrong.

Cheetah's Template Definition Language has 2 primary types of tags: {\bf
  placeholders} and {\bf directives}. Placeholder tags begin with a dollar sign
(\code{\$varName}) and are replaced with the value of the variable they refer to
when the template is filled. Directives begin with a hash character (\#) and are
used for everything else: for loops, conditional blocks, comments, includes, and
other advanced features. The following sections deal with these tags in detail.

%%%%%%%%%%%%%%%%%%%%%%%%%%%%%%%%%%%%%%%%%%%%%%%%%%%%%%%%%%%%%%%%%%%%%%%%%%%%%%%%
\subsection{Placeholder Tags}
\label{TDL.placeholders}
Placeholder Tags are equivalent to data fields in a form letter, or to the
\code{\%(key)s} fields used with Python's \code{\%} operator. Here's an example:

\begin{verbatim}
<HTML>
<HEAD><TITLE>$title</TITLE></HEAD>
<BODY>
$contents
</BODY>
</HTML>
\end{verbatim}

When this template is filled in, \code{\$title} and \code{\$content} will be
replaced with the values for \code{title} and \code{content}.  

Although these values are strings, they could be any type.  Cheetah converts all
\$placeholder values to strings via Python's \code{str()} function each time a
request is processed.

%%%%%%%%%%%%%%%%%%%%%%%%%%%%%%%%%%%%%%%%%%%%%%%%%%%%%%%%%%%%%%%%%%%%%%%%%%%%%%%%
\subsubsection{Rules for placeholder names}
\label{TDL.placeholders.rules}

\begin{itemize} 
     
\item Placeholder names consist of one or more identifiers separated by periods.
     Each identifier follows the same rules as a Python variable name or
     attribute name: it must start with a letter or an underscore, and the
     subsequent characters must be letters, numbers or underscores.  An
     identifier may be followed by argument sets enclosed in ``()'' and/or
     key/subscript arguments in ``[]''.

\item Identifiers are case-sensitive. \code{\$var} does not equal \code{\$Var}
     or \code{\$vAr} or \code{\$VAR}.     
     
\item String literals inside argument sets must be quoted, just like in Python.
     All Python quoting styles are allowed.
     
\item All variable names inside argument sets should be prefixed with a \$.  Use
     \$func(\$var) instead of \$func(var).
    
\item Trailing periods are ignored.  Cheetah will recognize that
     placeholder name in \code{\$varName.} is \code{varName} and the period will
     be left alone in the filled template output.
     
\item Placeholders can also be written in the form \code{\$\{placeholderName\}}.
     This is useful for cases where there is no whitespace between the
     placeholder and surrounding text (\code{surrounding\$\{embeddedVar\}text}).
   
\item Cheetah ignores all dollar signs (\code{\$}) that are not followed by a
     letter or an underscore.  Dollar amounts (\code{\$2.50})are output
     literally as they should be.  Cheetah also ignores any placeholder escaped
     by a backslash (\code{$\backslash$\$placeholderName}).

\end{itemize} 

The following are valid \$placeholders:
\begin{verbatim}
$a $_ $var $_var $var1 $_1var $var2_ $dict.key $list[3]
$object.method $object.method() $*object.method
${*20*a.b1.c('A', 1, $federico + 1).d['-']}
$nest($nest($var))
\end{verbatim}

These are not valid:
\begin{verbatim}
$var@2 $^var
\end{verbatim}
          
%%%%%%%%%%%%%%%%%%%%%%%%%%%%%%%%%%%%%%%%%%%%%%%%%%%%%%%%%%%%%%%%%%%%%%%%%%%%%%%%
\subsubsection{Uniform Dotted Notation}
\label{TDL.placeholders.unifiedDottedNotation}

Cheetah lets you subscript objects and dictionaries in the same way, using
dotted notation.  This is far easier for non-programmers to understand and it
insulates you from changes in the API behind the template's \$placeholders.

\$a.b.c will work with "a = \{'b':\{'c':1234\}\}" or "a = myClass(); a.b =
anotherClass()". You can still write \$a\['b'\]\['c'\] if you want to.

To use Uniform Dotted Notation with dictionaries, the key must be a legal
identifier.  That is, it must start with a letter or an underscore, and contain
only letters, digits and underscores.  If your dictionary key isn't a valid
identifier you must use the \code{\$a['key 1234a*98']} syntax instead.

Among Template-engines, Uniform Dotted Notation is unique to Cheetah.

%%%%%%%%%%%%%%%%%%%%%%%%%%%%%%%%%%%%%%%%%%%%%%%%%%%%%%%%%%%%%%%%%%%%%%%%%%%%%%%%
\subsubsection{Auto-calling}
\label{TDL.placeholders.autocalling}

Cheetah allows 'auto-calling' of methods and functions embedded as
\$placeholders in the template definition.  If this feature is enabled
(it is by default) then \$func and \$obj.meth will translated to \$func() and
\$obj.meth().  If the function or method require args then you still
have to do it like in Python \$func('foo', 1234)

Only functions and methods are autocalled.  Other callable types -- in
particular classes and instances -- are never auto-called.  So if you want to
call a class or instance or another type, you must use ``()''.

Section \ref{customizing} has instructions on how turn this feature off.

Auto-calling is also unique to Cheetah.

%%%%%%%%%%%%%%%%%%%%%%%%%%%%%%%%%%%%%%%%%%%%%%%%%%%%%%%%%%%%%%%%%%%%%%%%%%%%%%%%
\subsubsection{The searchList}
\label{TDL.placeholders.searchList}

A {\bf namespace} is a Python dictionary that links names to values. Each
template definition that has been loaded into a \code{Template} object is
associated with an ordered list of namespaces in which values for \$placeholder
variable names can be stored.  This list is called the {\bf searchList}.

The searchList can contain one or more namespaces.  In most cases only one
namespace will be in the searchList unless you explicitly load extra ones.  When
Cheetah fills in \code{\$content} in previous example it searchs sequentially
through the searchList until it finds a value for \code{content}.  Thus, if
three namespaces are loaded and two of them contain a value for
\code{\$content}, the value for \code{content} from the namespace that is
closest to the start of the searchList will be returned.

If you add a Python object to the searchList, its attributes and methods will be
accessible as \$placeholder names.  For example, \code{myObject} contains
\code{myAttrib} and \code{myMethod}.  If \code{myObject} is added to the
searchList, \code{\$myAttrib} and \code{\$myMethod} can be used as placeholder
names.  

The default namespace in every searchList is the \code{Template} object itself.
This means that any attributes or methods that are added to classes that inherit
from \code{Template} can be accessed in templates via \code{\$placeholders}.  New
namespaces can be added to the searchList at any time using the
\code{Template.addToSearchList()} method.  See section \ref{webware} for more
information on how to use namespaces and the searchList.

%%%%%%%%%%%%%%%%%%%%%%%%%%%%%%%%%%%%%%%%%%%%%%%%%%%%%%%%%%%%%%%%%%%%%%%%%%%%%%%%
\subsubsection{Missing values}
\label{TDL.placeholders.missing}

If there is no value defined in the searchList for a placeholder name,
Cheetah will search for an leading underscore version of the name.  For
example, if it can't find \code{\$varName} it will attempt to find
\code{\$\_varName}.  

If all attempts to find a value fail, Cheetah inserts the placeholder tag
verbatim in the filled template output.  This behaviour can be customized and a
default value set for names that are not found.  See section
\ref{customizing} for more details.

%%%%%%%%%%%%%%%%%%%%%%%%%%%%%%%%%%%%%%%%%%%%%%%%%%%%%%%%%%%%%%%%%%%%%%%%%%%%%%%%
\subsubsection{Placeholder values}
\label{TDL.placeholders.values}

Placeholder names can map to Python text strings, numbers, dictionaries, lists
(arrays), functions, objects, or even nested Cheetah templates.  

\begin{verbatim}
<HTML>
<HEAD><TITLE>$title</TITLE></HEAD>
<BODY>

$aString                        ## these names are chosen to indicate what
$aNumber                        ## type of value they map to.  They could be
$aList                          ## any valid placeholder name.
$aDictionary
$aFunction
$aFunction()
$aFunction(myArg=1234)
$anObject

</BODY>
</HTML>
\end{verbatim}

Templates can be nested.  A \$placeholder may be another Template object.  When
your template object is filled, the inner template object will automatically be
filled, and any template object's {\em it} contains will also be filled, ad
infinitum.

%%%%%%%%%%%%%%%%%%%%%%%%%%%%%%%%%%%%%%%%%%%%%%%%%%%%%%%%%%%%%%%%%%%%%%%%%%%%%%%%
\subsubsection{Caching}
\label{TDL.placeholders.caching}

By default the value of each \code{\$placeholder} is updated for each request.
Cheetah can be told to cache individual \code{\$placeholders} or even entire
sections of a template definition.
                         
If you want to statically cache the value of a single \code{\$placeholder} upon
startup, add an asterisk after the \$ sign.  \code{\$var} becomes \code{\$*var}.

If you only need to update the value of the \code{\$placeholders} at specific
time intervals use this form: \code{\$variable} becomes \code{\$*2.5*variable} ,
where 2.5 is the time interval in minutes.  

\begin{verbatim}
<HTML>
<HEAD><TITLE>$title</TITLE></HEAD>
<BODY>

$var ${var}           ## dynamic - will be reinterpolated for each request
$*var2 ${*var2}       ## static - will be interpolated only once at start-up
$*5*var3 ${*5*var3}   ## timed refresh - will be updated every 5 minutes.

</BODY>
</HTML>
\end{verbatim}

To cache entire sections of a template definition use the \code{\#cache ...
  \#end cache} directive tags:
\begin{verbatim}
#cache
$var ${var}
#end cache 

#cache 2.5
$var ${var}            ## These are re-cached every 2.5 minutes
#end cache 
\end{verbatim}

There is more information on the \code{\#cache} directive in section
\ref{directives.cache}.

This simple, but powerful, caching mechanism is unique to Cheetah.

%%%%%%%%%%%%%%%%%%%%%%%%%%%%%%%%%%%%%%%%%%%%%%%%%%%%%%%%%%%%%%%%%%%%%%%%%%%%%%%%
\subsection{Directive tags}
\label{TDL.directives}
Directives tags are used for all functionality that cannot be handled with
simple placeholders. Section \ref{directives} provides a reference guide to the
various directives.  Some directives consist of a single tag while others
consist of a pair of {\bf start} and {\bf end} tags that surround a chunk of
text.  End tags are written in the form \code{\#end nameOfTheDirective}.

%%%%%%%%%%%%%%%%%%%%%%%%%%%%%%%%%%%%%%%%%%%%%%%%%%%%%%%%%%%%%%%%%%%%%%%%%%%%%%%%
\subsubsection{Escaping directives}
\label{TDL.directives.escaping}

Directives can be escaped by placing a backslash ($\backslash$) before them.
Escaped directives will be printed verbatim.

%%%%%%%%%%%%%%%%%%%%%%%%%%%%%%%%%%%%%%%%%%%%%%%%%%%%%%%%%%%%%%%%%%%%%%%%%%%%%%%%
\subsubsection{Tag closures and whitespace handling}
\label{TDL.directives.closures}
Directive tags can closed explicitly with \code{/\#} or implicitly with the end
of the line if you're feeling lazy.

\begin{verbatim}
#block testBlock /#
Text in the contents area of the
block directive
#end block testBlock/#
\end{verbatim}
is identical to:
\begin{verbatim}
#block testBlock
Text in the contents area of the
block directive
#end block testBlock
\end{verbatim}

When a directive tag is closed explicitly it can be followed with other text on
the same line:

\begin{verbatim}
bah, bah, #if $sheep.color == 'black'/# black#end if/# sheep.
\end{verbatim}

When a directive tag is closed implicitly with the end of the line all trailing
whitespace is gobbled, including the newline character:
\begin{verbatim}
"""
foo #set $x = 2 
bar
"""
outputs 
"""
foo bar
"""

while 
"""
foo #set $x = 2 /#
bar
"""
outputs 
"""
foo 
bar
"""
\end{verbatim}

When a directive tag is closed implicitly AND there is no other text on the
line, the ENTIRE line will be gobbled up including any preceeding whitespace:
\begin{verbatim}
"""
foo 
   #set $x = 2 
bar
"""
outputs 
"""
foo
bar
"""

while 
"""
foo 
 - #set $x = 2
bar
"""
outputs 
"""
foo 
 - bar
"""
\end{verbatim}

The \code{\#slurp} directive, which is covered in more depth in section
\ref{directives.slurp} is a dummy directive that exists only to facilitate
gobbling of whitespace.


%%%%%%%%%%%%%%%%%%%%%%%%%%%%%%%%%%%%%%%%%%%%%%%%%%%%%%%%%%%%%%%%%%%%%%%%%%%%%%%%
%\subsection{Function and class libraries}
%
%Cheetah comes ``batteries included'' with functions, classes, packages and
%other objects you can use in your program.  Functions and constants that are
%also used by Cheetah itself are in the \code{Cheetah.Utilities} module.
%Functions, classes and packages contributed by third parties are in the
%\code{Cheetah.Tools} package.  



