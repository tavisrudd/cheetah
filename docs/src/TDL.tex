\section{The Template Definition Language}

%% @@MO:
% This chapter is for any information related to the TDL as a whole.
% Since the vocabulary page already says the TDL consists of $P and #D,
% and the details are fully explained in the next three chapters
% ("Placeholder Tags", "Directive Tags", "Functions and Macros"), I don't
% know what if anything needs to go in this chapter.

%% @@TR: I disagree.

{\bf Template definitions} are text strings, or files, that have been marked up
with Cheetah's {\bf Template Definition Language} for special processing.  This
language is not a general purpose programming language like Python.  Rather,
it's a mini-language that was designed to make HTML-generation, and
code-generation in general, easy enough for non-programmers to understand and
programmers to love.  It is purposefully limited so complex tasks are left to
Python code, where they belong.

Cheetah's Template Definition Language has 2 primary types of tags:

\begin{enumerate}
\item {\bf placeholders}: for marking areas of the template that should be
     filled in with something. 
     
     Placeholders begin with a dollar sign (\code{\$varName}).

\item {\bf directives}: for everything else:
     \begin{enumerate}   % level 2
     \item {\bf raw text} for marking verbatim blocks should not be parsed for
          Cheetah tags.
     \item {\bf comments} that should not appear in the output
     \item {\bf includes} to include external text.  The text can be included
          verbatim or with parsing for Cheetah tags.
     \item {\bf display logic} such as {\bf conditional blocks} (if-blocks) and
          {\bf for loops}.
     \item {\bf blocks}, which are named sections of a template that can be
          redefined (overridden) in a subclass or by template users.
     \item etc.
     \end{enumerate}  % level 2

     Directives begin with a hash character (\#).

The following chapters deal with these tags in detail.

