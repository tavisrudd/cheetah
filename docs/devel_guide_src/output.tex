%%%%%%%%%%%%%%%%%%%%%%%%%%%%%%%%%%%%%%%%%%%%%%%%%%%%%%%%%%%%%%%%%%%%%%%%%%%%%
\section{Directives: Output}
\label{output}

%%%%%%%%%%%%%%%%%%%%%%%%%%%%%%%%%%%%%%%%%%%%%%%%%%%%%%%%%%%%%%%%%%%%%%%%%%
\subsection{\#echo}
\label{output.echo}

The template:
\begin{verbatim}
Here is my #echo ', '.join(['silly']*5) # example
\end{verbatim}

The output:
\begin{verbatim}
Here is my silly, silly, silly, silly, silly example
\end{verbatim}

The generated code:
\begin{verbatim}
write('Here is my ')
write(filter(', '.join(['silly']*5) ))
write(' example\n')
\end{verbatim}

%%%%%%%%%%%%%%%%%%%%%%%%%%%%%%%%%%%%%%%%%%%%%%%%%%%%%%%%%%%%%%%%%%%%%%%%%%
\subsection{\#silent}
\label{output.silent}

The template:
\begin{verbatim}
Here is my #silent ', '.join(['silly']*5) # example
\end{verbatim}

The output:
\begin{verbatim}
Here is my  example
\end{verbatim}

The generated code:
\begin{verbatim}
        write('Here is my ')
        ', '.join(['silly']*5) 
        write(' example\n')
\end{verbatim}

OK, it's not quite covert because that extra space gives it away, but it 
almost succeeds.


%%%%%%%%%%%%%%%%%%%%%%%%%%%%%%%%%%%%%%%%%%%%%%%%%%%%%%%%%%%%%%%%%%%%%%%%%%
\subsection{\#raw}
\label{output.raw}

The template:
\begin{verbatim}
Text before raw.
#raw
Text in raw.  $alligator.  $croc.o['dile'].  #set $a = $b + $c.
#end raw
Text after raw.
\end{verbatim}

The output:
\begin{verbatim}
Text before raw.
Text in raw.  $alligator.  $croc.o['dile'].  #set $a = $b + $c.
Text after raw.
\end{verbatim}

The generated code:
\begin{verbatim}
        write('''Text before raw.
Text in raw.  $alligator.  $croc.o['dile'].  #set $a = $b + $c.
Text after raw.
''')
\end{verbatim}

So we see that \code{\#raw} is really like a quoting mechanism.  It says that
anything inside it is ordinary text, and Cheetah joins a \code{\#raw} section 
with adjacent string literals rather than generating a separate \code{write}
call.

%%%%%%%%%%%%%%%%%%%%%%%%%%%%%%%%%%%%%%%%%%%%%%%%%%%%%%%%%%%%%%%%%%%%%%%%%%
\subsection{\#include}
\label{output.include}

The main template:
\begin{verbatim}
#include "y.tmpl"
\end{verbatim}

The included template y.tmpl:
\begin{verbatim}
Let's go $voom!
\end{verbatim}

The shell command and output:
\begin{verbatim}
% voom="VOOM" x.py --env
Let's go VOOM!
\end{verbatim}

The generated code:
\begin{verbatim}
write(self._includeCheetahSource("y.tmpl", trans=trans, includeFrom="file",
    raw=0))
\end{verbatim}

%%%%%%%%%%%%%%%%%%%%%%%%%%%%%%%%%%%%%%%%%%%%%%%%%%%%%%%%%%%%%%%%%%%%%%%%%%
\subsubsection{\#include raw}
\label{output.include.raw}

The main template:
\begin{verbatim}
#include raw "y.tmpl"
\end{verbatim}

The shell command and output:
\begin{verbatim}
% voom="VOOM" x.py --env
Let's go $voom!
\end{verbatim}

The generated code:
\begin{verbatim}
write(self._includeCheetahSource("y.tmpl", trans=trans, includeFrom="fil
e", raw=1))
\end{verbatim}

That last argument, \code{raw}, makes the difference.

%%%%%%%%%%%%%%%%%%%%%%%%%%%%%%%%%%%%%%%%%%%%%%%%%%%%%%%%%%%%%%%%%%%%%%%%%%
\subsubsection{\#include from a string or expression (eval)}
\label{output.include.expression}

The template:
\begin{verbatim}
#attr $y = "Let's go $voom!"
#include source=$y
#include raw source=$y
#include source="Bam!  Bam!"
\end{verbatim}

The output:
\begin{verbatim}
% voom="VOOM" x.py --env
Let's go VOOM!Let's go $voom!Bam!  Bam!
\end{verbatim}

The generated code:
\begin{verbatim}
write(self._includeCheetahSource(VFS(SL,"y",1), trans=trans, 
    includeFrom="str", raw=0, includeID="481020889808.74"))
write(self._includeCheetahSource(VFS(SL,"y",1), trans=trans, 
    includeFrom="str", raw=1, includeID="711020889808.75"))
write(self._includeCheetahSource("Bam!  Bam!", trans=trans, 
    includeFrom="str", raw=0, includeID="1001020889808.75"))
\end{verbatim}

Later in the generated class:
\begin{verbatim}
y = "Let's go $voom!"
\end{verbatim}


%%%%%%%%%%%%%%%%%%%%%%%%%%%%%%%%%%%%%%%%%%%%%%%%%%%%%%%%%%%%%%%%%%%%%%%%%%
\subsection{\#slurp}
\label{output.slurp}

The template:
\begin{verbatim}
#for $i in range(5)
$i
#end for
#for $i in range(5)
$i #slurp
#end for
Line after slurp.
\end{verbatim}

The output:
\begin{verbatim}
0
1
2
3
4
0 1 2 3 4 Line after slurp.
\end{verbatim}

The generated code:
\begin{verbatim}
for i in range(5):
    write(filter(i)) # generated from '$i' at line 2, col 1.
    write('\n')
for i in range(5):
    write(filter(i)) # generated from '$i' at line 5, col 1.
    write(' ')
write('Line after slurp.\n')
\end{verbatim}

The space after each number is because of the space before \code{\#slurp} in 
the template definition.

%%%%%%%%%%%%%%%%%%%%%%%%%%%%%%%%%%%%%%%%%%%%%%%%%%%%%%%%%%%%%%%%%%%%%%%%%%
\subsection{\#filter}
\label{output.filter}

The template:
\begin{verbatim}
#attr $ode = ">> Rubber Ducky, you're the one!  You make bathtime so much fun! <<"
$ode
#filter WebSafe
$ode
#filter MaxLen
${ode, maxlen=13}
#filter None
${ode, maxlen=13}
\end{verbatim}

The output:
\begin{verbatim}
>> Rubber Ducky, you're the one!  You make bathtime so much fun! <<
&gt;&gt; Rubber Ducky, you're the one!  You make bathtime so much fun! &lt;&lt;
>> Rubber Duc
>> Rubber Ducky, you're the one!  You make bathtime so much fun! <<
\end{verbatim}

The \code{WebSafe} filter escapes characters that have a special meaning in 
HTML.  The \code{MaxLen} filter chops off values at the specified length.
\code{\#filter None} returns to the default filter, which ignores the \code{maxlen}
argument.

The generated code:
\begin{verbatim}
 1  write(filter(VFS(SL,"ode",1))) # generated from '$ode' at line 2, col 1.
 2  write('\n')
 3  filterName = 'WebSafe'
 4  if self._filters.has_key("WebSafe"):
 5      filter = self._currentFilter = self._filters[filterName]
 6  else:
 7      filter = self._currentFilter = \
 8                  self._filters[filterName] = getattr(self._filtersLib, 
                       filterName)(self).filter
 9  write(filter(VFS(SL,"ode",1))) # generated from '$ode' at line 4, col 1.
10  write('\n')
11  filterName = 'MaxLen'
12  if self._filters.has_key("MaxLen"):
13      filter = self._currentFilter = self._filters[filterName]
14  else:
15      filter = self._currentFilter = \
16                  self._filters[filterName] = getattr(self._filtersLib, 
                       filterName)(self).filter
17  write(filter(VFS(SL,"ode",1), maxlen=13)) # generated from 
        #'${ode, maxlen=13}' at line 6, col 1.
18  write('\n')
19  filter = self._initialFilter
20  write(filter(VFS(SL,"ode",1), maxlen=13)) # generated from 
       #'${ode, maxlen=13}' at line 8, col 1.
21  write('\n')
\end{verbatim}

As we've seen many times, Cheetah wraps all placeholder lookups in a
\code{filter} call.  (This also applies to non-searchList lookups: local,
global and builtin variables.)  The \code{filter} ``function''
is actually an alias to the current filter object:
\begin{verbatim}
filter = self._currentFilter
\end{verbatim}
as set at the top of the main method.  Here in lines 3-8 and 11-16 we see
the filter being changed.  Whoops, I lied.  \code{filter} is not an alias to
the filter object itself but to that object's \code{.filter} method.  Line 19
switches back to the default filter.  

In like 17, we see the \code{maxlen} argument being passed as a keyword
argument to \code{filter} (not to \code{VFS}).  In line 20, the same thing
happens, although the default filter ignores the argument.

% Local Variables:
% TeX-master: "devel_guide"
% End:      
