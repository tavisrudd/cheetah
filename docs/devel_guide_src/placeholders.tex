\section{Placeholders}
\label{placeholders}

%%%%%%%%%%%%%%%%%%%%%%%%%%%%%%%%%%%%%%%%%%%%%%%%%%%%%%%%%%%%%%%%%%%%%%%%%%%%%%
\subsection{Simple placeholders}
\label{placeholders.simple}

Let's add a few \$placeholders to our template:

\begin{verbatim}
>>> from Cheetah.Template import Template
>>> values = {'what': 'surreal', 'punctuation': '?'}
>>> t = Template("""\
... Hello, $what world$punctuation
... One of Python's least-used functions is $xrange.
... """, [values])
>>> print t
Hello, surreal world?
One of Python's least-used functions is <built-in function xrange>.

>>> print t.generatedModuleCode()
  1	#!/usr/bin/env python
    
  2	"""
  3	Autogenerated by CHEETAH: The Python-Powered Template Engine
  4	 CHEETAH VERSION: 0.9.12
  5	 Generation time: Sun Apr 21 00:53:01 2002
  6	"""
    
  7	__CHEETAH_genTime__ = 'Sun Apr 21 00:53:01 2002'
  8	__CHEETAH_version__ = '0.9.12'
    
  9	##################################################
 10	## DEPENDENCIES
    
 11	import sys
 12	import os
 13	import os.path
 14	from os.path import getmtime, exists
 15	import time
 16	import types
 17	from Cheetah.Template import Template
 18	from Cheetah.DummyTransaction import DummyTransaction
 19	from Cheetah.NameMapper import NotFound, valueForName, 
           valueFromSearchList
 20	import Cheetah.Filters as Filters
 21	import Cheetah.ErrorCatchers as ErrorCatchers
    
 22	##################################################
 23	## MODULE CONSTANTS
    
 24	try:
 25	    True, False
 26	except NameError:
 27	    True, False = (1==1), (1==0)
    
 28	##################################################
 29	## CLASSES
    
 30	class GenTemplate(Template):
 31	    """
 32	    
 33	    Autogenerated by CHEETAH: The Python-Powered Template Engine
 34	    """
    
 35	    ##################################################
 36	    ## GENERATED METHODS
    
    
 37	    def __init__(self, *args, **KWs):
 38	        """
 39	        
 40	        """
    
 41	        Template.__init__(self, *args, **KWs)
    
 42	    def respond(self,
 43	            trans=None,
 44	            dummyTrans=False,
 45	            VFS=valueFromSearchList,
 46	            VFN=valueForName,
 47	            getmtime=getmtime,
 48	            currentTime=time.time):
    
    
 49	        """
 50	        This is the main method generated by Cheetah
 51	        """
    
 52	        if not trans:
 53	            trans = DummyTransaction()
 54	            dummyTrans = True
 55	        write = trans.response().write
 56	        SL = self._searchList
 57	        filter = self._currentFilter
 58	        globalSetVars = self._globalSetVars
 59	        
 60	        ########################################
 61	        ## START - generated method body
 62	        
 63	        write('Hello, ')
 64	        write(filter(VFS(SL,"what",1))) # generated from '$what' at 
                                                # line 1, col 8.
 65	        write(' world')
 66	        write(filter(VFS(SL,"punctuation",1))) # generated from 
                                     # '$punctuation' at line 1, col 19.
 67	        write("\nOne of Python's least-used methods is ")
 68	        write(filter(xrange)) # generated from '$xrange' at line 2, 
                                      # col 39.
 69	        write('.\n')
 70	        
 71	        ########################################
 72	        ## END - generated method body
 73	        
 74	        if dummyTrans:
 75	            return trans.response().getvalue()
 76	        else:
 77	            return ""
 78	        
 79	    ##################################################
 80	    ## GENERATED ATTRIBUTES
    
    
 81	    __str__ = respond
    
 82	    _mainCheetahMethod_for_GenTemplate= 'respond'
    
    
 83	# CHEETAH was developed by Tavis Rudd, Chuck Esterbrook, Ian Bicking 
        # and Mike Orr;
 84	# with code, advice and input from many other volunteers.
 85	# For more information visit http://www.CheetahTemplate.org
    
 86	##################################################
 87	## if run from command line:
 88	if __name__ == '__main__':
 89	    GenTemplate().runAsMainProgram()
    
\end{verbatim}

(Again, I have added line numbers and split the lines as in the previous
chapter.)

This generated template module is different from the previous one in several
trivial respects and one important respect.  Trivially, 
\code{.\_filePath} and \code{.\_fileMtime} are not updated in 
\code{.\_\_init\_\_}, so they inherit the value \code{None} from 
\code{Template}.  Also, that if-stanza in \code{.respond} that recompiles the
template if the source file changes is missing -- because there is no source
file.  So this module is several lines shorter than the other one.

But the important way this module is different is that instead of the one
\code{write} call outputting a string literal, this module has a series of
\code{write} calls (lines 63-69) outputting successive chunks of the
template.  Regular text has been translated into a string literal, and
placeholders into function calls.  Every placeholder is wrapped inside a
\code{filter} call to apply the current output filter.  (The default
output filter converts all objects to strings, and \code{None} to \code{""}.)

Placeholders referring to a Python builtin like \code{xrange} (line 68)
generate a bare variable name.  Placeholders to be looked up in the searchList
have a nested function call; e.g.,
\begin{verbatim}
write(filter(VFS(SL,"what",1))) # generated from '$what' at line 1, col 8. 
\end{verbatim}
\code{VFS}, remember, is a function imported from \code{Cheetah.NameMapper}
that looks up a value in a searchList.  So we pass it the searchList, the
name to look up, and a boolean (1) indicating we want autocalling.  (It's
\code{1} rather than \code{True} because it's generated from an
\code{and} expression, and that's what Python 2.2 outputs for true \code{and}
expressions.)

%%%%%%%%%%%%%%%%%%%%%%%%%%%%%%%%%%%%%%%%%%%%%%%%%%%%%%%%%%%%%%%%%%%%%%%%%%%%%%
\subsection{Complex placeholders}
\label{placeholders.complex}

Placeholders can get far more complicated than that.  This example shows what
kind of code the various NameMapper features produce.  The formulas are
taken from Cheetah's test suite, in the
\code{Cheetah.Tests.SyntaxAndOutput.Placeholders} class.  

\begin{verbatim}
1 placeholder: $aStr
2 placeholders: $aStr $anInt
2 placeholders, back-to-back: $aStr$anInt
1 placeholder enclosed in {}: ${aStr}
1 escaped placeholder: \$var
func placeholder - with (): $aFunc()
func placeholder - with (int): $aFunc(1234)
func placeholder - with (string): $aFunc('aoeu')
func placeholder - with ('''\nstring'\n'''): $aFunc('''\naoeu'\n''')
func placeholder - with (string*int): $aFunc('aoeu'*2)
func placeholder - with (int*float): $aFunc(2*2.0)
Python builtin values: $None $True $False
func placeholder - with ($arg=float): $aFunc($arg=4.0)
deeply nested argstring: $aFunc(  $arg = $aMeth( $arg = $aFunc( 1 ) ) ):
function with None: $aFunc(None)
autocalling: $aFunc! $aFunc().
nested autocalling: $aFunc($aFunc).
list subscription: $aList[0]
list slicing: $aList[:2]
list slicing and subcription combined: $aList[:2][0]
dict - NameMapper style: $aDict.one
dict - Python style: $aDict['one']
dict combined with autocalled string method: $aDict.one.upper
dict combined with string method: $aDict.one.upper()
nested dict - NameMapper style: $aDict.nestedDict.two
nested dict - Python style: $aDict['nestedDict']['two']
nested dict - alternating style: $aDict['nestedDict'].two
nested dict - NameMapper style + method: $aDict.nestedDict.two.upper
nested dict - alternating style + method: $aDict['nestedDict'].two.upper
nested dict - NameMapper style + method + slice: $aDict.nestedDict.two.upper[:4]
nested dict - Python style, variable key: $aDict[$anObj.meth('nestedDict')].two
object method: $anObj.meth1
object method + complex slice: $anObj.meth1[0: ((4/4*2)*2)/$anObj.meth1(2) ]
very complex slice: $( anObj.meth1[0: ((4/4*2)*2)/$anObj.meth1(2) ] )
\end{verbatim}

We'll need a big program to set up the placeholder values.  Here it is:

\begin{verbatim}
#!/usr/bin/env python
from ComplexExample import ComplexExample

try:   # Python >= 2.2.1
    True, False
except NameError:  # Older Python
    True, False = (1==1), (1==0)

class DummyClass:
    _called = False
    def __str__(self):
        return 'object'

    def meth(self, arg="arff"):
        return str(arg)

    def meth1(self, arg="doo"):
        return arg

    def meth2(self, arg1="a1", arg2="a2"):
        return str(arg1) + str(arg2)

    def callIt(self, arg=1234):
        self._called = True
        self._callArg = arg

def dummyFunc(arg="Scooby"):
    return arg

defaultTestNameSpace = {
    'aStr':'blarg',
    'anInt':1,
    'aFloat':1.5,
    'aList': ['item0','item1','item2'],
    'aDict': {'one':'item1',
              'two':'item2',
              'nestedDict':{1:'nestedItem1',
                          'two':'nestedItem2'
                          },
              'nestedFunc':dummyFunc,
              },
    'aFunc': dummyFunc,
    'anObj': DummyClass(),
    'aMeth': DummyClass().meth1,
}

print ComplexExample( searchList=[defaultTestNameSpace] )
\end{verbatim}

Here's the output:

\begin{verbatim}
1 placeholder: blarg
2 placeholders: blarg 1
2 placeholders, back-to-back: blarg1
1 placeholder enclosed in {}: blarg
1 escaped placeholder: $var
func placeholder - with (): Scooby
func placeholder - with (int): 1234
func placeholder - with (string): aoeu
func placeholder - with ('''\nstring'\n'''): 
aoeu'

func placeholder - with (string*int): aoeuaoeu
func placeholder - with (int*float): 4.0
Python builtin values:  1 0
func placeholder - with ($arg=float): 4.0
deeply nested argstring: 1:
function with None: 
autocalling: Scooby! Scooby.
nested autocalling: Scooby.
list subscription: item0
list slicing: ['item0', 'item1']
list slicing and subcription combined: item0
dict - NameMapper style: item1
dict - Python style: item1
dict combined with autocalled string method: ITEM1
dict combined with string method: ITEM1
nested dict - NameMapper style: nestedItem2
nested dict - Python style: nestedItem2
nested dict - alternating style: nestedItem2
nested dict - NameMapper style + method: NESTEDITEM2
nested dict - alternating style + method: NESTEDITEM2
nested dict - NameMapper style + method + slice: NEST
nested dict - Python style, variable key: nestedItem2
object method: doo
object method + complex slice: do
very complex slice: do

\end{verbatim}

And here -- tada! -- is the generated module.
To save space, I've included only the lines containing the \code{write} calls.
The rest of the module is the same as in the first example, chapter
\ref{pyModules.example}.  I've split some of the lines to make them fit on
the page.

\begin{verbatim}
 1  write('1 placeholder: ')
 2  write(filter(VFS(SL,"aStr",1))) # generated from '$aStr' at line 1, col 16.
 3  write('\n2 placeholders: ')
 4  write(filter(VFS(SL,"aStr",1))) # generated from '$aStr' at line 2, col 17.
 5  write(' ')
 6  write(filter(VFS(SL,"anInt",1))) 
        # generated from '$anInt' at line 2, col 23.
 7  write('\n2 placeholders, back-to-back: ')
 8  write(filter(VFS(SL,"aStr",1))) # generated from '$aStr' at line 3, col 31.
 9  write(filter(VFS(SL,"anInt",1))) 
        # generated from '$anInt' at line 3, col 36.
10  write('\n1 placeholder enclosed in {}: ')
11  write(filter(VFS(SL,"aStr",1))) # generated from '${aStr}' at line 4, 
        # col 31.
12  write('\n1 escaped placeholder: $var\nfunc placeholder - with (): ')
13  write(filter(VFS(SL,"aFunc",0)())) # generated from '$aFunc()' at line 6, 
        # col 29.
14  write('\nfunc placeholder - with (int): ')
15  write(filter(VFS(SL,"aFunc",0)(1234))) # generated from '$aFunc(1234)' at 
        # line 7, col 32.
16  write('\nfunc placeholder - with (string): ')
17  write(filter(VFS(SL,"aFunc",0)('aoeu'))) # generated from "$aFunc('aoeu')"
        # at line 8, col 35.
18  write("\nfunc placeholder - with ('''\\nstring'\\n'''): ")
19  write(filter(VFS(SL,"aFunc",0)('''\naoeu'\n'''))) # generated from 
        # "$aFunc('''\\naoeu'\\n''')" at line 9, col 46.
20  write('\nfunc placeholder - with (string*int): ')
21  write(filter(VFS(SL,"aFunc",0)('aoeu'*2))) # generated from 
        # "$aFunc('aoeu'*2)" at line 10, col 39.
22  write('\nfunc placeholder - with (int*float): ')
23  write(filter(VFS(SL,"aFunc",0)(2*2.0))) # generated from '$aFunc(2*2.0)' 
        # at line 11, col 38.
24  write('\nPython builtin values: ')
25  write(filter(None)) # generated from '$None' at line 12, col 24.
26  write(' ')
27  write(filter(True)) # generated from '$True' at line 12, col 30.
28  write(' ')
29  write(filter(False)) # generated from '$False' at line 12, col 36.
30  write('\nfunc placeholder - with ($arg=float): ')
31  write(filter(VFS(SL,"aFunc",0)(arg=4.0))) # generated from 
        # '$aFunc($arg=4.0)' at line 13, col 40.
32  write('\ndeeply nested argstring: ')
33  write(filter(VFS(SL,"aFunc",0)(  
        arg = VFS(SL,"aMeth",0)( arg = VFS(SL,"aFunc",0)( 1 ) ) ))) 
	# generated from '$aFunc(  $arg = $aMeth( $arg = $aFunc( 1 ) ) )' 
	# at line 14, col 26.
34  write(':\nfunction with None: ')
35  write(filter(VFS(SL,"aFunc",0)(None))) # generated from '$aFunc(None)' at 
        # line 15, col 21.
36  write('\nautocalling: ')
37  write(filter(VFS(SL,"aFunc",1))) # generated from '$aFunc' at line 16, 
        # col 14.
38  write('! ')
39  write(filter(VFS(SL,"aFunc",0)())) # generated from '$aFunc()' at line 16, 
        # col 22.
40  write('.\nnested autocalling: ')
41  write(filter(VFS(SL,"aFunc",0)(VFS(SL,"aFunc",1)))) # generated from 
        # '$aFunc($aFunc)' at line 17, col 21.
42  write('.\nlist subscription: ')
43  write(filter(VFS(SL,"aList",1)[0])) # generated from '$aList[0]' at line 
        # 18, col 20.
44  write('\nlist slicing: ')
45  write(filter(VFS(SL,"aList",1)[:2])) # generated from '$aList[:2]' at 
        # line 19, col 15.
46  write('\nlist slicing and subcription combined: ')
47  write(filter(VFS(SL,"aList",1)[:2][0])) # generated from '$aList[:2][0]' 
        # at line 20, col 40.
48  write('\ndict - NameMapper style: ')
49  write(filter(VFS(SL,"aDict.one",1))) # generated from '$aDict.one' at line
        # 21, col 26.
50  write('\ndict - Python style: ')
51  write(filter(VFS(SL,"aDict",1)['one'])) # generated from "$aDict['one']" 
        # at line 22, col 22.
52  write('\ndict combined with autocalled string method: ')
53  write(filter(VFS(SL,"aDict.one.upper",1))) # generated from 
        # '$aDict.one.upper' at line 23, col 46.
54  write('\ndict combined with string method: ')
55  write(filter(VFN(VFS(SL,"aDict.one",1),"upper",0)())) # generated from 
        # '$aDict.one.upper()' at line 24, col 35.
56  write('\nnested dict - NameMapper style: ')
57  write(filter(VFS(SL,"aDict.nestedDict.two",1))) # generated from 
        # '$aDict.nestedDict.two' at line 25, col 33.
58  write('\nnested dict - Python style: ')
59  write(filter(VFS(SL,"aDict",1)['nestedDict']['two'])) # generated from 
        # "$aDict['nestedDict']['two']" at line 26, col 29.
60  write('\nnested dict - alternating style: ')
61  write(filter(VFN(VFS(SL,"aDict",1)['nestedDict'],"two",1))) # generated 
        # from "$aDict['nestedDict'].two" at line 27, col 34.
62  write('\nnested dict - NameMapper style + method: ')
63  write(filter(VFS(SL,"aDict.nestedDict.two.upper",1))) # generated from 
        # '$aDict.nestedDict.two.upper' at line 28, col 42.
64  write('\nnested dict - alternating style + method: ')
65  write(filter(VFN(VFS(SL,"aDict",1)['nestedDict'],"two.upper",1))) 
        # generated from "$aDict['nestedDict'].two.upper" at line 29, col 43.
66  write('\nnested dict - NameMapper style + method + slice: ')
67  write(filter(VFN(VFS(SL,"aDict.nestedDict.two",1),"upper",1)[:4])) 
        # generated from '$aDict.nestedDict.two.upper[:4]' at line 30, col 50.
68  write('\nnested dict - Python style, variable key: ')
69  write(filter(VFN(VFS(SL,"aDict",1)
        [VFN(VFS(SL,"anObj",1),"meth",0)('nestedDict')],"two",1))) 
	# generated from "$aDict[$anObj.meth('nestedDict')].two" at line 31, 
	# col 43.
70  write('\nobject method: ')
71  write(filter(VFS(SL,"anObj.meth1",1))) # generated from '$anObj.meth1' at 
        # line 32, col 16.
72  write('\nobject method + complex slice: ')
73  write(filter(VFN(VFS(SL,"anObj",1),"meth1",1)
        [0: ((4/4*2)*2)/VFN(VFS(SL,"anObj",1),"meth1",0)(2) ])) 
	# generated from '$anObj.meth1[0: ((4/4*2)*2)/$anObj.meth1(2) ]' 
	# at line 33, col 32.
74  write('\nvery complex slice: ')
75  write(filter(VFN(VFS(SL,"anObj",1),"meth1",1)
        [0: ((4/4*2)*2)/VFN(VFS(SL,"anObj",1),"meth1",0)(2) ] )) 
	# generated from '$( anObj.meth1[0: ((4/4*2)*2)/$anObj.meth1(2) ] )' 
	# at line 34, col 21.
76  write('\n')
\end{verbatim}

For each placeholder lookup, the the innermost level of nesting is a \code{VFS}
call, which looks up the first (leftmost) placeholder component in the
searchList.  This is wrapped by zero or more \code{VFN} calls, which perform
Universal Dotted Notation lookup on the next dotted component of the
placeholder, looking for an attribute or key by that name within the previous
object (not in the searchList).  Autocalling is performed by \code{VFS} and
\code{VFN}: that's the reason for their third argument.

Explicit function/method arguments, subscripts and keys (which
are all expressions) are left unchanged, besides expanding any embedded
\$placeholders in them.  This means they must result in valid Python
expressions, following the standard Python quoting rules.

Built-in Python values (\code{None}, \code{True} and \code{False}) are
converted to \code{filter(None)}, etc.  They use normal Python variable
lookup rather than \code{VFS}.  (Cheetah emulates \code{True} and \code{False}
using global variables for Python < 2.2.1, when they weren't builtins yet.)

% Local Variables:
% TeX-master: "devel_guide"
% End:      
