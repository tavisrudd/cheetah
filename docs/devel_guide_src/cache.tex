%%%%%%%%%%%%%%%%%%%%%%%%%%%%%%%%%%%%%%%%%%%%%%%%%%%%%%%%%%%%%%%%%%%%%%%%%%
\section{Caching placeholders and \#cache}
\label{cache}

%%%%%%%%%%%%%%%%%%%%%%%%%%%%%%%%%%%%%%%%%%%%%%%%%%%%%%%%%%%%%%%%%%%%%%%%%%
\subsection{Dynamic placeholder -- no cache}
\label{cache.dynamic}

The template:
\begin{verbatim}
Dynamic variable:  $voom
\end{verbatim}

The command line and the output:
\begin{verbatim}
% voom='Voom!' python x.py --env
Dynamic variable:  Voom!
\end{verbatim}

The generated code:
\begin{verbatim}
write('Dynamic variable:  ')
write(filter(VFS(SL,"voom",1))) # generated from '$voom' at line 1, col 20.
write('\n')
\end{verbatim}

Just what we expected, like any other dynamic placeholder.

%%%%%%%%%%%%%%%%%%%%%%%%%%%%%%%%%%%%%%%%%%%%%%%%%%%%%%%%%%%%%%%%%%%%%%%%%%
\subsection{Static placeholder}
\label{cache.static}

The template:
\begin{verbatim}
Cached variable:  $*voom
\end{verbatim}

The command line and output:
\begin{verbatim}
% voom='Voom!' python x.py --env
Cached variable:  Voom!
\end{verbatim}

The generated code, with line numbers:
\begin{verbatim}
 1  write('Cached variable:  ')
 2  ## START CACHE REGION: at line, col (1, 19) in the source.
 3  RECACHE = True
 4  if not self._cacheData.has_key('19760169'):
 5      pass
 6  else:
 7      RECACHE = False
 8  if RECACHE:
 9      orig_trans = trans
10      trans = cacheCollector = DummyTransaction()
11      write = cacheCollector.response().write
12      write(filter(VFS(SL,"voom",1))) # generated from '$*voom' at line 1,
            # col 19.
13      trans = orig_trans
14      write = trans.response().write
15      self._cacheData['19760169'] = cacheCollector.response().getvalue()
16      del cacheCollector
17  write(self._cacheData['19760169'])
18  ## END CACHE REGION
    
19  write('\n')
\end{verbatim}

That one little star generated a whole lotta code.  First, instead of an 
ordinary \code{VFS} lookup (searchList) lookup, it converted the
placeholder to a lookup in the \code{.\_cacheData} dictionary.  Cheetah also
generated a unique key (\code{'19760169'}) for our cached item -- this is its
cache ID.

Second, Cheetah put a pair of if-blocks before the \code{write}.  The first
(lines 3-7) determine whether the cache value is missing or out of date, and
sets local variable \code{RECHARGE} true or false.
This stanza may look unnecessarily verbose -- lines 3-7 could be eliminated if
line 8 was changed to
\begin{verbatim}
if not self._cacheData.has_key('19760169'):
\end{verbatim}
-- but this model is expandable for some of the cache features we'll see below.

The second if-block, lines 8-16, do the cache updating if necessary.
Clearly, the programmer is trying to stick as close to normal (dynamic)
workflow as possible.  Remember that \code{write}, even though it looks like a
local function, is actually a method of a file-like object.  So we create a
temporary file-like object to divert the \code{write} object into, then read
the result and stuff it into the cache.

%%%%%%%%%%%%%%%%%%%%%%%%%%%%%%%%%%%%%%%%%%%%%%%%%%%%%%%%%%%%%%%%%%%%%%%%%%
\subsection{Timed-refresh placeholder}
\label{cache.timed}

The template:
\begin{verbatim}
Timed cache:  $*.5m*voom
\end{verbatim}

The command line and the output:
\begin{verbatim}
% voom='Voom!' python x.py --env
Timed cache:  Voom!
\end{verbatim}

The generated method's docstring:
\begin{verbatim}
"""
This is the main method generated by Cheetah
This cache will be refreshed every 30.0 seconds.
"""
\end{verbatim}

The generated code:
\begin{verbatim}
 1  write('Timed cache:  ')
 2  ## START CACHE REGION: at line, col (1, 15) in the source.
 3  RECACHE = True
 4  if not self._cacheData.has_key('55048032'):
 5      self.__cache55048032__refreshTime = currentTime() + 30.0
 6  elif currentTime() > self.__cache55048032__refreshTime:
 7      self.__cache55048032__refreshTime = currentTime() + 30.0
 8  else:
 9      RECACHE = False
10  if RECACHE:
11      orig_trans = trans
12      trans = cacheCollector = DummyTransaction()
13      write = cacheCollector.response().write
14      write(filter(VFS(SL,"voom",1))) # generated from '$*.5m*voom' at 
            # line 1, col 15.
15      trans = orig_trans
16      write = trans.response().write
17      self._cacheData['55048032'] = cacheCollector.response().getvalue()
18      del cacheCollector
19  write(self._cacheData['55048032'])
20  ## END CACHE REGION
    
21  write('\n')
\end{verbatim}

This code is identical to the static cache example except for the docstring
and the first if-block.  (OK, so the cache ID is different and the comment on
line 14 is different too.  Big deal.)

Each timed-refresh cache item has a corrsponding private attribute 
\code{.\_\_cache\#\#\#\#\#\#\#\#\_\_refreshTime} giving the refresh time
in ticks (=seconds since January 1, 1970).  The first if-block (lines 3-9)
checks whether the cache value is missing or its update time has passed, and if
so, sets \code{RECHARGE} to true and also schedules another refresh at the next
interval.

The method docstring reminds the user how often the cache will be refreshed.
This information is unfortunately not as robust as it could be.  Each
timed-cache placeholder blindly generates a line in the docstring.  If all
refreshes are at the same interval, there will be multiple identical lines
in the docstring.  If the refreshes are at different intervals, you get a
situation like this:
\begin{verbatim}
"""
This is the main method generated by Cheetah
This cache will be refreshed every 30.0 seconds.
This cache will be refreshed every 60.0 seconds.
This cache will be refreshed every 120.0 seconds.
"""
\end{verbatim}
The docstring tells only that ``something'' will be refreshed every 60.0
seconds, but doesn't reveal {\em which} placeholder that is.  Only if you
know the relative order of the placeholders in the template can you figure
that out.

%%%%%%%%%%%%%%%%%%%%%%%%%%%%%%%%%%%%%%%%%%%%%%%%%%%%%%%%%%%%%%%%%%%%%%%%%%
\subsection{Timed-refresh placeholder with braces}
\label{cache.timed.braces}

This example is the same but with the long placeholder syntax.  It's here 
because it's a Cheetah FAQ whether to put the cache interval inside or outside
the braces.  (It's also here so I can look it up because I frequently forget.)
The answer is: outside.  The braces go around only the placeholder name (and
perhaps some output-filter arguments.)

The template:
\begin{verbatim}
Timed with {}:  $*.5m*{voom}
\end{verbatim}

The output:
\begin{verbatim}
Timed with {}:  Voom!
\end{verbatim}

The generated code differs only in the comment.  Inside the cache-refresh
if-block:
\begin{verbatim}
write(filter(VFS(SL,"voom",1))) # generated from '$*.5m*{voom}' at line 1, 
    #col 17.
\end{verbatim}

The reason this example is here is because it's a Cheetah FAQ whether to
put the cache interval inside or outside the \verb+{}+.  (Also so I can look
it up when I forget, as I frequently do.)  The answer is: outside.  The
\verb+{}+ go around only the placeholder name and arguments.  If you do it
this way:
\begin{verbatim}
Timed with {}:  ${*.5m*voom}      ## Wrong!
\end{verbatim}
you get:
\begin{verbatim}
Timed with {}:  ${*.5m*voom}
\end{verbatim}
because \verb+${+ is not a valid placeholder, so it's treated as ordinary text.

%%%%%%%%%%%%%%%%%%%%%%%%%%%%%%%%%%%%%%%%%%%%%%%%%%%%%%%%%%%%%%%%%%%%%%%%%%
\subsection{\#cache}
\label{cache.directive}

The template:
\begin{verbatim}
#cache
This is a cached region.  $voom
#end cache
\end{verbatim}

The output:
\begin{verbatim}
This is a cached region.  Voom!
\end{verbatim}

The generated code:
\begin{verbatim}
 1  ## START CACHE REGION: at line, col (1, 1) in the source.
 2  RECACHE = True
 3  if not self._cacheData.has_key('23711421'):
 4      pass
 5  else:
 6      RECACHE = False
 7  if RECACHE:
 8      orig_trans = trans
 9      trans = cacheCollector = DummyTransaction()
10      write = cacheCollector.response().write
11      write('This is a cached region.  ')
12      write(filter(VFS(SL,"voom",1))) # generated from '$voom' at line 2, 
            # col 27.
13      write('\n')
14      trans = orig_trans
15      write = trans.response().write
16      self._cacheData['23711421'] = cacheCollector.response().getvalue()
17      del cacheCollector
18  write(self._cacheData['23711421'])
19  ## END CACHE REGION
\end{verbatim}

This is the same as the \code{\$*voom} example, except that the plain text
around the placeholder is inside the second if-block.  

%%%%%%%%%%%%%%%%%%%%%%%%%%%%%%%%%%%%%%%%%%%%%%%%%%%%%%%%%%%%%%%%%%%%%%%%%%
\subsection{\#cache with timer and id}
\label{cache.directive.timer}

The template:
\begin{verbatim}
#cache timer='.5m', id='cache1'
This is a cached region.  $voom
#end cache
\end{verbatim}

The output:
\begin{verbatim}
This is a cached region.  Voom!
\end{verbatim}

The generated code is the same as the previous example except the first
if-block:
\begin{verbatim}
RECACHE = True
if not self._cacheData.has_key('13925129'):
    self._cacheIndex['cache1'] = '13925129'
    self.__cache13925129__refreshTime = currentTime() + 30.0
elif currentTime() > self.__cache13925129__refreshTime:
    self.__cache13925129__refreshTime = currentTime() + 30.0
else:
    RECACHE = False
\end{verbatim}

%%%%%%%%%%%%%%%%%%%%%%%%%%%%%%%%%%%%%%%%%%%%%%%%%%%%%%%%%%%%%%%%%%%%%%%%%%
\subsection{\#cache with test: expression and method conditions}
\label{cache.directive.test}

The template:
\begin{verbatim}
#cache test=$isDBUpdated
This is a cached region.  $voom
#end cache
\end{verbatim}

(Analysis postponed: bug in Cheetah produces invalid Python.)

%The output:
%\begin{verbatim}
%\end{verbatim}

%The generated code:
%\begin{verbatim}
%\end{verbatim}


The template:
\begin{verbatim}
#cache id='cache1', test=($isDBUpdated or $someOtherCondition)
This is a cached region.  $voom
#end cache
\end{verbatim}

The output:
\begin{verbatim}
This is a cached region.  Voom!
\end{verbatim}

The first if-block in the generated code:
\begin{verbatim}
RECACHE = True
if not self._cacheData.has_key('36798144'):
    self._cacheIndex['cache1'] = '36798144'
elif (VFS(SL,"isDBUpdated",1) or VFS(SL,"someOtherCondition",1)):
    RECACHE = True
else:
    RECACHE = False
\end{verbatim}
The second if-block is the same as in the previous example.  If you leave
out the \code{()} around the test expression, the result is the same, although
it may be harder for the template maintainer to read.

You can even combine arguments, although this is of questionable value.

The template:
\begin{verbatim}
#cache id='cache1', timer='30m', test=$isDBUpdated or $someOtherCondition
This is a cached region.  $voom
#end cache
\end{verbatim}

The output:
\begin{verbatim}
This is a cached region.  Voom!
\end{verbatim}

The first if-block:
\begin{verbatim}
RECACHE = True
if not self._cacheData.has_key('88939345'):
    self._cacheIndex['cache1'] = '88939345'
    self.__cache88939345__refreshTime = currentTime() + 1800.0
elif currentTime() > self.__cache88939345__refreshTime:
    self.__cache88939345__refreshTime = currentTime() + 1800.0
elif VFS(SL,"isDBUpdated",1) or VFS(SL,"someOtherCondition",1):
    RECACHE = True
else:
    RECACHE = False
\end{verbatim}

We are planning to add a \code{'varyBy'} keyword argument in the future that
will allow a separate cache instances to be created for a variety of conditions,
such as different query string parameters or browser types. This is inspired by
ASP.net's varyByParam and varyByBrowser output caching keywords.  Since this is
not implemented yet, I cannot provide examples here.

% Local Variables:
% TeX-master: "devel_guide"
% End:      
